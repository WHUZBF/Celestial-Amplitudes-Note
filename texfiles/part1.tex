\part{A quick review on SR \& GR}
\section{Basics conceptions in SR}\label{sec:1}
我们生活的空间是一个四维局部平坦的\textbf{Lorentz流形},也就是一个四维微分流形配备一个非正定、非退化的度规$g=g_{\mu\nu}dx^\mu\otimes dx^\nu$,这是一个$(0,2)$张量。局部平坦意思是说任何一点处都可以选取一个坐标系\sn{这个坐标系称为\textbf{局部惯性系},由于可以从指数映射结合测地线来构造这个惯性系,所以也称为\textbf{自由下落参考系}}使得$g_{\mu\nu}=\eta_{\mu\nu},\nabla_\rho g_{\mu\nu}$\sn{这里符号约定为$\eta_{\mu\nu}=(-,+,+,+)$}。而Lorentz体现在$\eta$有一个指标是负数,而且根据惯性定理,无论你选取什么坐标系将度规对角化,最终负数的个数都是一样的,这样一来我们便可以严格的区分时间和空间\sn{如果这里把$\eta$中的$-1$变成$+1$,我们称为\textbf{Riemann流形}}。

参数化流形后,时空上的每一点(事件)都将对应一个坐标$x^\mu$,时空中的曲线(世界线)可以参数化为$x^\mu(\tau)$,其可以看作是由矢量场$X=X^\mu\partial_\mu=\frac{x^\mu(\tau)}{d\tau}\partial_\mu$诱导的。考虑世界线上相邻的两点,我们可以定义线长为
\[dl=\sqrt{|g(X,X)|}=\int d\tau\sqrt{g_{\mu\nu}\frac{x^\mu}{d\tau}\frac{x^\nu}{d\tau}}\]
很多时候也把线长记为$ds^2=g_{\mu\nu}dx^\mu dx^\nu$但是在微分几何的严格意义下,$dx^\mu$是对偶矢量,并不是初等微积分中的微分,所以这个式子只能理解为$ds^2=g_{\mu\nu}dx^\mu\otimes dx^\nu$,也就是说$ds^2$只是张量$g$的另一个叫法而已!后面我们为了方便可能牺牲严谨性,使用$ds^2$表示世界线长。

GR中最重要的基本假设便是在坐标变换下物理定律是不变的,这说明作用量必须是标量,几乎唯一确定了真空引力场作用量为:
\begin{margintable}\footnotesize 
	\begin{tabularx}{\marginparwidth}{|X}
		$\Lambda$是宇宙学常数\\
		$g\equiv\det g_{\mu\nu}$\\
		$R$是Ricci标量\\
		取自然单位制$c=\hbar=1$
	\end{tabularx}
\end{margintable}
\begin{equation}
	S_{\mathrm{EH}}=\frac{1}{16\pi G}\int \mathrm{d}^4x\sqrt{-g} (R-2\Lambda)
\end{equation}
由于度规是张量,所以其在坐标变换下分量变换为:\sn{注意两边对应的自变量,因为张量都是关于流形上点的场,所以这里坐标变换后流形上某点对应的坐标也变了。}
\[\tilde{g}_{\mu\nu}(\tilde{x})=\frac{\partial \tilde{x}^\rho}{\partial x^\mu}\frac{\partial \tilde{x}^\sigma}{\partial x^\nu}g_{\rho\sigma}(x)\]
在SR中我们仅研究平直的时空,或者说只研究惯性系之间的变换,这些惯性系中的变换满足$\tilde{\eta}=\eta$,可以一般的记为:
\[\tilde{x}^\mu=\Lambda^{\mu}_{\ \nu}x^\nu+a^\mu\]
那么$\Lambda$满足:
\begin{equation}
	\Lambda^{\operatorname{T}}\eta\Lambda=\eta
\end{equation}
本笔记主要考虑SR时空。
\section{Poincar\'e group and Lorentz group}
显然所有的$\Lambda$构成了一个群,称之为\textbf{Lorentz群}:
\begin{equation}
	L\equiv O(3,1)\equiv\left\{\Lambda\in M(4,\mathbb{R})|	\Lambda^{\operatorname{T}}\eta\Lambda=\eta\right\}
\end{equation}
而所有的保度规变换还要加入$a^mu$,构成\textbf{Poincar\'e群}:$O(3,1)\ltimes \mathbb{R}$,群乘法为:
\begin{equation}
	(\Lambda,a)\cdot(\Lambda',a')=(\Lambda\cdot\Lambda',a+\Lambda\cdot a')
\end{equation}
利用Poincar\'e群的不等价不可约表示可以对基本粒子进行分类,见本部分末尾。后面我们将主要关注Lorentz群。
不难验证$\det \Lambda=\pm 1$,它将Lorentz群分成两个分支,其中$\det \Lambda=1$的部分含有单位元,构成子群\textbf{正规Lorentz群}。记为$SO(3,1)$或$L_+$

另外$(\Lambda^0_{\ 0})^2\geq 1$也将Lorentz群分成两个分支,其中$\Lambda^0_{\ 0}\geq1$的部分含有单位元,构成子群\textbf{正时Lorentz群}。记为$O(3,1)^\uparrow$或$L^\uparrow$。

最后$L^\uparrow_+\equiv L^\uparrow\cap L_+$也构成了$L$的一个子群。这些子群之间可以用时间反演和空间反演算符相联系:
\begin{equation}
	\mathcal{T}=\begin{pmatrix}
		-1&  &  & \\
		& 1 &  & \\
		&  & 1 & \\
		&  &  &1
	\end{pmatrix},\quad\quad\mathcal{P}=\begin{pmatrix}
	1&  &  & \\
	& -1 &  & \\
	&  & -1 & \\
	&  &  &-1
	\end{pmatrix}
\end{equation}
显然$L_+=\{L_+^\uparrow,\mathcal{T}\},L^\uparrow=\{L_+^\uparrow,\mathcal{P}\},L=\{L_+^\uparrow,\mathcal{T},\mathcal{P}\}$

\subsection{Poincar\'e algebra}
现在考虑群的局部性质,考虑无穷小坐标变换$x^\mu\mapsto x^\mu+\xi^\mu$,保度规条件为:
\begin{equation}
	\tilde{\eta}_{\mu\nu}(\tilde x)-\eta_{\mu\nu}(x)=\partial_\mu\xi_\nu+\partial_\mu\xi_\nu=0
\end{equation}
$\xi^\mu$可以用$\omega_{\ \nu}^\mu$和$b^\mu$两个无穷小参数标记:
\[\xi^\mu={\omega^\mu}_\nu x^\nu+b^\mu,\quad \omega_{\mu\nu}=-\omega_{\nu\mu}\]
平移生成元为:
\[P_\mu=-i\partial_\mu\Rightarrow T(b)=\exp(-i b^\mu P_\mu)\]
boost和转动生成元为:
\[M_{\mu\nu}=i\left(x_\mu\partial_\nu-x_\nu\partial_\mu\right)\Rightarrow \Lambda(\omega)=\exp(-\frac{i}{2} \omega^{\mu\nu} M_{\mu\nu})\]
生成共同构成Poincar\'e代数:
\begin{equation}
	\begin{aligned}
		&[P_\mu,P_\nu]=0,\quad[P_\rho,M_{\mu\nu}]=i(\eta_{\mu\rho}P_\nu-\eta_{\nu\rho} P_\mu)\\
		&[M_{\mu\nu},M_{\rho\sigma}=i\left(\eta_{\mu\rho}M_{\nu\sigma}-\eta_{\mu\sigma}M_{\nu\rho}-\eta_{\nu\rho}M_{\nu\rho}+\eta_{\nu\sigma}M_{\mu\rho}\right)]
	\end{aligned}
\end{equation}

\section{Boost and Rapidity}
我们把Lorentz群记为$O(3,1)$强烈暗示了其与$O(n)$群的类似性,其实Lorentz变换完全可以看作是四维时空中的旋转。三维空间旋转有三个自由度,分别是绕着$x,y,z$轴的旋转,这些轴都是由另外两个轴张成的平面所确定的,总数为$C_3^2=3$。那么对于高维空间旋转,比如四维空间可以推广为共$C_4^2=6$个自由度。其中有3个是单纯的$\mathbb{R}^3$中的旋转,还有三个是混合了时间轴的旋转,也就是初等SR介绍中的两个相对速度为$v$的惯性系之间的变换,称为\textbf{boost}。比如$x$方向上的boost就可以显式表达出来为:
\begin{margintable}\footnotesize 
	\begin{tabularx}{\marginparwidth}{|X}
		$\gamma(v)=1/\sqrt{(1-\beta(v)^2)}$\\
		$\beta(v)=v/c$
	\end{tabularx}
\end{margintable}
\begin{equation}
	\Lambda(v)=\begin{pmatrix}
		\gamma(v)&-\gamma(v)\beta(v)  &0  & 0\\
		-\gamma(v)\beta(v) & \gamma(v) &0  &0 \\
		0&0  & 1 &0 \\
		0&  0&  0&1
	\end{pmatrix}
\end{equation}

$\Lambda(v)$是一个boost,或者说四维转动,类比三维转动$R_x(\theta)R_x(\phi)R_x(\theta+\phi)$,很容易想到$\Lambda(v)\Lambda(w)\overset{?}{=}\Lambda(v+w)$,即绕着某个轴的转动是一个单参数Abel子群。但实际上以$v$为参数并不能看出这一点。可以定义\sn{$\chi\in(-\infty,+\infty)$,所以是非紧致的Lie群}:
\begin{equation}
	\chi(v)\equiv\operatorname{arctanh}(\frac{v}{c})
\end{equation}
称为\textbf{rapidity}这样便有$\Lambda(\chi_2)\Lambda(\chi_1)=\Lambda(\chi_2+\chi_1)$。

rapidity其实有非常明显的物理含义,回忆一下速度的定义:
\[\text{velocity}=\frac{\text{displacement}}{\text{time}}\]
由于右边的分式分子分母都是依赖于参考系的,所以如果B相对于A运动\sn{方便起见假设沿$x$轴作直线运动,但不要求匀速},实际上可以对于B定义三种不同的速度。
\begin{definition}
	$v=\frac{dx}{dt}$,这里$x,t$都是在A的参考系下测得的。
\end{definition}
\begin{definition}
	$u=\frac{dx}{d\tau}$,这里$x$是在A的参考系下测的,$\tau$是B的固有时。
\end{definition}
这个定义是关于参考系协变的,也就是通常的$\mathfrak{4}$\mbox{-}速度的定义。
\begin{definition}
	$\tilde(v)=\frac{dx_B}{d\tau}$,分子分母都是B自己测得的。
\end{definition}
但是这个定义有个很大的问题,B自己测量时间没问题,但是B自己测量自己的位移始终是0,所以上面这个定义必须重新审视。首先我们看如何对应B的加速度,假设B在固有时$\tau$的时刻相对于地面系的速度\sn{第一个定义}为$v$,这个时候考虑一个与B速度相同的瞬时惯性系,也就是说过一段时间$d\tau$之后B相对于这个瞬时惯性系会有个速度$d\tilde{v}$,加速度也便定义为$d\tilde{v}/d\tau$。假设这段时间内,相对于地面系B速度增加了$dv$,那么根据速度叠加法则:
\begin{equation}
	\frac{v+d\tilde{v}}{1+vd\tilde{v}/c^2}=v+dv\Rightarrow d\tilde{v}\left(1+v^2/c^2\right)dv
\end{equation}
现在对加速度进行积分:
\begin{equation}
		I(\tau)=\int^\tau_0 d\tau\frac{d\tilde{v}}{d\tau}=\int^{\tilde{v}}_0d\tilde{v}=\int^{v}_0\frac{dv}{1+v^2/c^2}=c\cdot\operatorname{arctanh}\frac{v(\tau)}{c}=c\cdot\chi(v(\tau))
\end{equation}
所以在这个速度的定义下,自然导出了rapidity,如果取自然单位制$c=1$,那么两者完全一致。
\section{Connected components of Lorentz group}
对于任何正规且正时的Lorentz群中的元素都可以做标准分解:
\begin{theorem}
	$\forall \Lambda\in L_+^\uparrow, \exists R_1,R_2\in SO(3)$,使得
	\begin{equation}\label{eq:4.1}
		\Lambda = R_1 L_x(\chi) R_2
	\end{equation}
\end{theorem}
而Lorentz群只需要再加上$\mathcal{T}$和$\mathcal{P}$即可,而且这种分解对于$d>2$维时空都是适用的。从物理上很好理解,一般的Lorentz变换无非就是绕着任意轴的boost,我们都可以先进行转动,将boost方向转为$x$轴,进行boost之后再转回原来的方向。

任何一个Lie群实际上都是一个微分流形,而连通性这个概念正是建立在此之上从拓扑观点来看的。作为一个流形,$G$不一定是连通的,可以有很多个连通分支,其中只有含有$e$的连通分支才能构成子群,我们记为$G_e$,有下面的定理成立:
\begin{theorem}
	$G_e\lhd G$且所有连通分支构成商群$G/G_e$.
\end{theorem}
这里不做严格证明,下面我们将此定理用于Lorentz群。根据\ref{eq:4.1},由于其中的每个因子都与$e$道路连通,所以$L^\uparrow_+\subseteq L_e$,而$L$中的其它群元为了与$e$相连,必须通过离散变换$\mathcal{T},\mathcal{P}$,所以实际上$L^\uparrow_+=L_e$,那么连通分支构成商群$L/L^\uparrow_+\cong \mathbb{Z}_2\times\mathbb{Z}_2$,这里同构成立是因为商群实际上由$\{\mathcal{T},\mathcal{P}\}$生成。
\begin{margintable}\footnotesize 
	\begin{tabularx}{\marginparwidth}{|X}
		后面我们都用$\cong$表示同构,$\simeq$表示同态
	\end{tabularx}
\end{margintable}

可见Lorentz群确实包含了4个连通分支,可以根据$\Lambda^0_{\ 0}$以及$\det \Lambda$的符号进行分类。
\section{Poincar\'e group and particles}
这一部分是最为精妙的部分,我们将会利用Poincar\'e群的不可约表示对场和粒子进行分类,本节论述主要参考Weinberg\cite{Weinberg}和董无极\cite{WKT}


to be continue\ldots

