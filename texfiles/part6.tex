\part{A Crush Course on CFT}
% 计数器清零,每个part都要引用,除了part1
\setcounter{theorem}{0}
\setcounter{definition}{0}
\setcounter{lemma}{0}
\setcounter{sidenote}{1}

CFT经典教材是大黄书\cite{DiFrancesco:1997nk},教材\cite{Blumenhagen:2009zz,ito}也是不错的选择\sn{伊藤克司先生的书还没有官方中文译本,不过这里有民间译本的重排本\url{https://github.com/WHUZBF/CFT-book}},还有一些比较好的讲义\cite{Nawata:2022lsw,Ginsparg:1988ui,Qualls:2015qjb},共形场论的自举(bootstrap)方法也是非常重要的,可以见讲义\cite{Ribault:2014hia}的专门介绍。CFT也可以当作一门数学课来学习,偏重于数学的阅读材料有\cite{Schottenloher:2008zz}。

首先是一些Overview性质的介绍。CFT无非就是一种特殊的QFT,但是这个时候理论具有比Poincar\'e对称性更大的对称性,在二维的情况下甚至提升为无穷维的对称性,这种对称性能让我们不通过微扰场论直接确定关联函数。一般的QFT中我们用Poincar\'e的不同的不可约表示来标记不同的场,或者说,我们用自旋来标记场,到了CFT这边,我们还需要使用共形维数$\Delta$来标记,对于$s=0$的标量场,共形维数定义为在Dilation $x\mapsto\lambda x$ 下场变换为:
\begin{equation}
	\phi^\prime(\lambda \vec{x})=\lambda^{-\Delta}\phi(\vec x)
\end{equation}
但是这只是对Dilation要求场在共形变换下“协变”,进一步要求对任意共形变换“协变”就给出了\textbf{初级场}(primary field)的定义。\sn{注意我们遵从大黄书的符号约定,和\cite{Blumenhagen:2009zz,Ginsparg:1988ui}的定义恰巧相反,对最终CFT中的一些结论没有任何影响,仅仅只是中间推导过程有些细微的差别。}\sn{$s=1/2$是何形式?}
\begin{definition}
	共形维数为$\Delta$的初级场(s=0)定义为在任意共形变换下满足:
	\begin{equation}
		\phi^\prime(\vec{x^\prime})=\left|\frac{\partial \vec {x^\prime}}{\partial \vec {x}}\right|^{\Delta/d}\phi(\vec {x})
	\end{equation}
\end{definition}

初级场将是后面研究的主要对象。二维的共形对称性比较特殊,分为global和local的,如果上面的“协变性”只对global的共形变换适用,那我们称之为\textbf{准初级场}(quasi-primary field),显然初级场一定是准初级场,反过来却不一定。二维情况下我们还使用复平面为坐标\footnote{这是欧氏空间CFT的主要选取,也是后文研究的主要内容,Wick转动到闵氏时空之后选取所谓光锥坐标。},但是我们为了一些地方的方便,并不是考虑的$\mathbb{C}$,而是$\mathbb{C}^2$,也就是说我们把$z,\bar z$看作是完全独立的变量,进在一些特殊情况下认为$z^*=\bar z$。而且把$\Delta$拆分为共形权$(h,\bar h)=\frac{1}{2}\left(\Delta+s,\Delta-s\right)$在这一符号约定下,共形权定义变为:
\begin{equation}
	\boxed{
	\phi^{\prime}(\lambda z,\overline{\lambda}\overline{z})=\lambda^{-h}\overline{\lambda}^{-\overline{h}}\phi(z,\overline{z})
	}
\end{equation}
(准)初级场定义变为:
\begin{equation}
	\boxed{
		\phi^{\prime}\left(f(z),\overline{f}(\overline{z})\right)=\left(\frac{\partial f}{\partial z}\right)^{-h}\left(\frac{\partial\overline{f}}{\partial\bar{z}}\right)^{\overline{-h}}\phi(z,\bar{z})
	}
\end{equation}
如果$\phi$全纯我们称为\textbf{chiral},反全纯称为\textbf{anti-chiral}。无穷小共形变换$x\mapsto x+\epsilon$下:
\begin{equation}\label{ict}
	\boxed{
	\delta_{\epsilon,\bar{\epsilon}}\phi(z,\bar{z})\equiv\phi^\prime(x^\prime)-\phi(x)=-\left(h\mathrm{~}\partial_z\epsilon+\epsilon\mathrm{~}\partial_z+\overline{h}\mathrm{~}\partial_{\bar{z}}\bar{\epsilon}+\overline{\epsilon}\mathrm{~}\partial_{\bar{z}}\right)\phi(z,\overline{z})
	}
\end{equation}
\begin{remark}
	初级场的定义可以看作是一种拓宽的张量定义,考虑一个带$s$个协变指标的张量,在任意坐标变换$x\mapsto x+\epsilon(x)$下:
	\[-\delta\Phi_{\mu_1\cdots\mu_s}=\epsilon^\nu\partial_\nu\Phi_{\mu_1\cdots\mu_s}+(\partial_{\mu_1}\epsilon^\nu)\Phi_{\nu\mu_2\cdots\mu_s}+\cdots+(\partial_{\mu_s}\epsilon^\nu)\Phi_{\mu_1\cdots\mu_{s-1}\nu}\]
	换到复平面,简单起见只考虑全纯部分,那么$h=s$,上面的定义局限于$h$是整数,现在考虑任意取值,那些指标也没必要写出了,便得到:
	\[-\delta\Phi=\epsilon\partial\Phi+s\partial \epsilon\Phi\]
	这就是前面得到的无穷小变换形式。
\end{remark}

同QFT一样,这些场都会量子化成算符。不像QFT中我们研究的场是有限多个的,比如说QED就是正负电子对应的Dirac场和一个U(1)规范场耦合,CFT中我们研究的场很多情况下会是无穷多个的,因为我们把$\phi,\partial\phi$看作是不同的场,因为他们是不同共形权的初级场。定义一个CFT第一步就是告诉我们理论中有哪些初级场,也就是一个\textbf{谱}(Spectrum)$\{\mathcal{O}_{h,\bar h}\}$。CFT中我们并非按照微扰场论那一套来建立关联函数的计算方法的,而会去关注场之间的\textbf{算符乘积展开(OPE)},后面将会看到自举给出了OPE的绝大多数信息,还有一些系数是自举无法确定的,需要CFT的定义给定,这是定义CFT的第二个data。我们这样做是在算符的观点下看问题,或者说是在海森堡表象下看问题,那CFT的态是什么呢?这其实被所谓\textbf{态算符对应}联系起来。

最后想强调一点,正是因为CFT的思考方式和一般的QFT有比较大的不同,所以CFT的建立甚至是不需要已知理论的拉氏量的,我们需要知道的只是理论拥有的对称性,然后去找对称性的表示构造谱,能动量张量则刻画了CFT在共形变换下的性质,是构建OPE必须的,如果强行利用拉格朗日量进行分析反而会变得非常复杂,有的CFT甚至是没办法写下一个拉氏量的,但是通过自洽性分析我们是知道这种理论的存在性的,而且可以根据自举方法走得很远。

\section{Virasoro Algebra}
讨论共形场论,都是在量子层面上已经消去共形反常后的理论,比如YM理论就是量子化后存在共形反常的QFT,从而不能看作一个CFT。二维CFT的共形代数是$\mathrm{Witt}\times\overline{\mathrm{Witt}}$,后面我们讨论CFT其实都是在其中心扩张$\mathrm{Vir_c}\times\overline{\mathrm{Vir_c}}$下进行\sn{明确一下convention,扩张前的代数用小写$l$标记,扩张后的用大写$L$标记}。这里我不打算讨论过多中心荷的物理意义,这应当是弦论课的内容,为何要引入中心荷可以从群表示的观点来看。

首先如果一个CFT具有某个对称性,这个对称性构成了一个群,那么体系的谱就生活在这个群的群表示之中\sn{严格说是表示的最高权是那些初级场,也就是CFT的谱,由这些谱生成的次级态最终张成整个CFT的希尔伯特空间,也就是对称代数的表示。}。也就是说假设对称代数为$\mathfrak{V}\times\bar{\mathfrak{V}}$\sn{注意在二维CFT中我们都会将对称代数复化为两个独立的部分。两个独立的李代数可以作为线性空间考虑他们的直和$\mathfrak{V}\oplus\bar{\mathfrak{V}}$,这里写成$\times$也没错,是将他们考虑成集合,然后做卡氏积,由于两部分独立,所以这两者本质上没区别。},则:
\begin{equation}
	\boxed{\mathcal{S}=\bigoplus_{(\mathcal{R},\mathcal{R}^{\prime})\in\mathrm{Rep}(\mathfrak{V})^2}m_{\mathcal{R},\mathcal{R}^{\prime}}\mathcal{R}\otimes\bar{\mathcal{R}^{\prime}}}
\end{equation}
但是这个表示只需要是个射影表示就好,而射影表示对应的是李代数的中心扩张的表示。前面处理洛伦兹群我们不用考虑那么多,因为我们这证明了理论总是可以redefine来消去中心荷,但是一般的共形场论至少都有Virasoro对称性,而这个代数的中心荷是non-trivial的,一般是不能消除的,所以我们考虑对称性并非直接考虑$\mathrm{Vir}$,而是$\mathrm{Vir}_c$。
\begin{theorem}[Virasoro Algebra]
	\begin{equation}
		\boxed{
		\left[L_m,L_n\right]=\left(m-n\right)L_{m+n}+\frac c{12}\left(m^3-m\right)\delta_{m+n,0}
		}
	\end{equation}
\end{theorem}
\begin{remark}
	如果我们考虑$\mathfrak{sl}(2,\mathbbm{C})$子代数,会发现中心扩张是trivial的。
\end{remark}
\section{Radial Quantization and Hilbert Space}
\subsection{Radial Quantization}
对于$1+1$维时空,可以把空间方向一点紧化为圆,这样平面就会变为一个圆柱面,这样就可以引入圆柱面上的复坐标$w=x^0+ix^1$,这样$w\sim w+2\pi i$自动满足周期性条件,这也是弦论worldsheet的图像。而圆柱面又可以通过共形变换$w\mapsto e^w$变到另一个复平面上:
\begin{figure}[htbp]
	\centering
	\includegraphics{figs/fig9.pdf}
	\caption{径向量子化}
\end{figure}
在这一变换下,时间方向变为了径向,空间方向变成了角向,而且把整个无穷远映射到了原点这一个点上。这下时间平移对应的哈密顿量$H$到复平面这边就变成了dilation算子$D$,而空间平移对应旋转$e^{i\theta}$:
\begin{equation}
		\boxed{H=L_0+\bar {L}_0,\quad P=i\left(=L_0-\bar {L}_0\right)}
\end{equation}
也正是因为时间方向变成了径向,时序积就应当变成“径向顺序积”,我们要去计算的就是径向顺序积的真空期望值,对于玻色子定义为:\sn{费米子把下面的情况改成负号就好}
\begin{equation}
	R\bigl(A(z)B(w)\bigr)\equiv\begin{cases}{}A(z)B(w)&\text{for }|z|>|w|,\\B(w)A(z)&\text{for }|w|>|z|.\end{cases}
\end{equation}

圆柱上的场我们用$\varphi(w,\bar w)$表示,复平面上的场用$\phi(z,\bar z)$表示,那么根据共形变换:
\begin{equation}
	\phi(z,\bar z)=z^{-h}{\bar {z}}^{-\bar h}\varphi(w,\bar w)
\end{equation}
在圆柱那边的CFT中我们知道$\varphi(w,\bar w)$可以做平面波展开为$e^{-ip^0x^0+ip^1x^1}$,由于我们考虑的是欧氏空间的场论,Wick转动后得到$e^{-p^0x^0+ip^1x^1}$,即$(e^{-w})^n(e^{-\bar w})^{\bar m}$的形式\sn{$n+m=p^0,n-m=p^1$}。而这恰恰就是洛朗展开的每一项!所以复平面上的CFT的模式展开为洛朗展开:
\begin{theorem}[Mode Expansion]
	\begin{equation}
		\boxed{\phi(z,\overline{z})=\sum_{n,\overline{m}\in\mathbb{Z}}z^{-n-h}\overline{z}^{-\overline{m}-\overline{h}}\phi_{n,\overline{m}}}
	\end{equation}
\end{theorem}
量子化场就是把$\phi_{n,\bar m}$量子化为算符,这些算符具有$(n,\bar m)$的权。

QFT微扰计算中一个非常重要的概念是渐近态,我们考虑的都是无穷远处的平面波入射经过复杂的相互作用后的平面波出射,微扰的角度看入射态就是在无穷远处的过去在真空附近施加一个扰动,也就是插入场算符,出射态也类似的这样做,这样就会自然出来时序积,然后LSZ约化这些的去计算振幅。同样的CFT这边也有这样的物理图景,只是现在无穷远处过去被映射到原点这一个点上了,所以渐近态应该定义为:
\begin{equation}
	ket{\phi}=\lim_{z,\overline{z}\to0}\phi(z,\overline{z})\ket{0}
\end{equation}
在原点处上式不奇异自然要求:\sn{这似乎告诉我们$n>-h,\bar m>-\bar h$对应的是CFT中的湮灭算符,后面讲到对易子会回到这里}
\begin{equation}\label{eq:30.6}
	\phi_{n,\overline{m}}\ket{0} =0\quad\text{for}\quad n>-h,\quad\overline{m}>-\overline{h}
\end{equation}
完整得到渐近态的公式:
\begin{equation}
	\boxed{|\phi\rangle=\lim\limits_{z,\overline{z}\to0}\phi(z,\overline{z})\left|0\right\rangle=\phi_{-h,-\overline{h}}\left|0\right\rangle}
\end{equation}
这里有个非常有趣的地方,本来一个算符里面有很多模,最后只剩下了一个,所以在CFT中,态和\textbf{局域}算符是唯一对应的!这就是态\mbox{-}算符对应。这和一般的QFT形成了鲜明的对比,那里一个场对应是不同“方向”平面波的组合,没有做到一一对应,但是CFT做到正是因为他把一个无穷大的空间部分在无穷远的过去全部紧化到了原点这一处,为了加深理解我们用路径积分的观点来看一下。

回忆量子力学中的传播子可以用路径积分表达为:
\begin{equation}
	G(x_f,x_i)=\int_{x(\tau_i)=x_i}^{x(\tau_f)=x_f}\mathcal{D}xe^{iS}
\end{equation}
那么演化后的末态波函数可以用初态波函数传播得到:
\begin{equation}
	\psi_{f}(x_{f},\tau_{f})=\int dx_{i}G(x_{f},x_{i})\psi_{i}(x_{i},\tau_{i})
\end{equation}
QFT这边“波函数”是一个泛函$\Psi[\phi(\sigma)]$,其模方表示在固定时间,空间中每个点$\sigma$上出现场构型$\phi(\sigma)$的概率,可以用路径积分得到末态:
\begin{equation}
	\Psi_f[\phi_f(\sigma),\tau_f]=\int\mathcal{D}\phi_i\int_{\phi(\tau_i)=\phi_i}^{\phi(\tau_f)=\phi_f}\mathcal{D}\phi e^{-S[\phi]}\Psi_i[\phi_i(\sigma),\tau_i]
\end{equation}
现在从柱面换到复平面CFT:
\begin{equation}
	\Psi_f[\phi_f(\sigma),r_f]=\int\mathcal{D}\phi_i\int_{\phi(r_i)=\phi_i}^{\phi(r_f)=\phi_f}\mathcal{D}\phi e^{-S[\phi]}\Psi_i[\phi_i(\sigma),r_i]
\end{equation}
不难发现初始态的作用是对$|z|=r_i$的积分进行加权,构造初始态需要在无穷远的过去,$r_i\to 0$,从而初始态只对原点处积分进行加权,等价于在原点处插入一个算子,即:
\begin{equation}
	\Psi[\phi_f;r]=\int^{\phi(r)=\phi_f}\mathcal{D}\phi e^{-S[\phi]}\mathcal{O}(z=0)
\end{equation}

\subsection{BPZ Conjugate}
为了构建关联函数,我们需要“左矢”,也就是厄米共轭来构造出射态从而有内积,这里谈的BPZ共轭是厄米共轭在Wick转动后的径向量子化欧氏空间CFT上的自然推广。不想扯过多物理上的考量,\cite{ito}从Wick转动后厄米共轭必须与时间反演一同定义BPZ共轭来解释了这一定义。
\begin{definition}[BPZ Conjugate]
	\begin{equation}
		\phi^\dagger(z,\overline{z})=\overline{z}^{-2h}z^{-2\overline{h}}\phi\left(\frac{1}{\overline{z}},\frac{1}{z}\right)
	\end{equation}
\end{definition}
另一方面直接从洛朗展开得到:
\begin{equation}
	\phi^{\dagger}(z,\overline{z})=\overline{z}^{-2h}z^{-2\overline{h}}\sum_{n,\overline{m}\in\mathbb{Z}}\overline{z}^{+n+h}z^{+\overline{m}+\overline{h}}\phi_{n,\overline{m}}=\sum_{n,\overline{m}\in\mathbb{Z}}\overline{z}^{+n-h}z^{+\overline{m}-\overline{h}}\phi_{n,\overline{m}}
\end{equation}
从而有:
\begin{equation}\label{BPZ}
	\boxed{
	\left(\phi_{n,\overline{m}}\right)^\dagger=\phi_{-n,-\overline{m}}
	}
\end{equation}
根据场的BPZ共轭定义,出射态应该定义为:
\begin{equation}
	\bra{\phi}=\lim\limits_{z,\overline{z}\to0}\bra{ 0  }              \phi^\dagger(z,\overline{z})=\lim\limits_{\overline{w},w\to\infty}w^{2h}\overline{w}^{2\overline{h}}\bra{0}\phi(w,\overline{w})=\bra{0}\phi_{h,\bar h}
\end{equation}
这里根据非奇异性要求了:
\begin{equation}\label{eq:30.17}
	\left<0\right|\phi_{n,\overline{m}}=0\quad\text{for}\quad n<h,\quad\overline{m}<\overline{h}
\end{equation}
当然,这里是根据定义严格来构造了一遍,实际计算疯狂使用\ref{BPZ}就好。
\section{Ward Identity and OPE}
\subsection{Infinitesimal Conformal Ward Identity}
前面$\S 4$已经看到共形对称性对应的守恒流可以使用$T_{\mu\nu}\epsilon^\nu$来构造,利用前面得到的最一般的Ward恒等式\ref{eq:19.11},对整个时空进行积分:
\begin{equation}
	\int dx^0dx^1\frac\partial{\partial x^{\mu}} \left\langle j_{a}^{\mu}(x)\Phi(x_{1})\cdots\Phi(x_{n})\right\rangle  =\sum_{i=1}^{n}\left<\Phi(x_{1})\delta_{\epsilon,\bar \epsilon}\cdots\Phi(x_{i})\cdots\Phi(x_{n})\right>
\end{equation}
$z=e^{x^0+ix^1}=x+iy$,进一步利用积分换元得到:\sn{别忘了雅可比行列式}
\begin{equation}
	\begin{aligned}
		\int dx^0dx^1&\partial_\mu\left\langle j_{a}^{\mu}(x)\Phi(x_{1})\cdots\Phi(x_{n})\right\rangle\\
		&=4\int dzd\bar z\left[\bar\partial(T_{\bar z\bar z}\epsilon)+\partial(T_{zz}\bar\epsilon)\right]\\
		&=2\int dxdy\left[\bar\partial(T_{\bar z\bar z}\epsilon)+\partial(T_{zz}\bar\epsilon)\right]
	\end{aligned}
\end{equation}
上面取了约定$\epsilon^z\equiv\epsilon(z),\partial_z\equiv\partial$,反全纯部分类似。现在考虑二元积分的格林公式:
\begin{equation}
	\begin{aligned}\oint_{\partial\Omega}P(x,y)dx+Q(x,y)dy=\int_\Omega dxdy\left(\frac{\partial Q}{\partial x}-\frac{\partial P}{\partial y}\right)\end{aligned}
\end{equation}
利用复变量$z=x+iy$可得下面的形式:\sn{就像是Wick转动一样,我们总假设这样随意复化是可行的}
\begin{equation}
	\begin{aligned}\oint_{\partial_\Omega}frac{1}{2}(P-iQ)dz+\frac{1}{2}(P+iQ)d\bar{z}=\int_\Omega dxdy(\partial(Q-iP)+\bar{\partial}(Q+iP))\end{aligned}
\end{equation}
上式中所有的围道积分都假设方向相对于$z$而言是逆时针,那么在$\bar z$平面上也就是逆时针,不过为了方便,而且CFT中我们也只考虑全纯部分,反全纯部分只要做个Conjugate就好了,后文中$\oint d\bar z$,表示围道方向相对于$\bar z$逆时针。
\begin{equation}
	\int_\Omega dxdy\left(\partial \bar f+\bar\partial f\right)=\frac{1}{2i}\oint_{\partial\omega}fdz+\bar f d\bar z
\end{equation}
再取:
\begin{equation}
	T(z)\equiv-2\pi iT_{zz},\bar T(\bar z)\equiv-2\pi iT_{\bar z\bar z}
\end{equation}
不过这个归一化因子并不重要,只是为了让OPE有一般的convention,后面比如自由玻色子这些具体模型我们都是先将归一化因子待定,再根据OPE的形式来确定它,只需要记住二维共形场论中$T_{zz}$始终全纯,$T_{\bar z\bar z}$始终反全纯就好了。归一化能动张量后,得到初级场的Ward恒等式:
\begin{equation}\label{ward}
	\boxed{
	\sum_{i=1}^{n}\left<\Phi(x_{1})\cdots\delta_{\epsilon,\bar \epsilon}\Phi(x_{i})\cdots\Phi(x_{n})\right>=\frac{1}{2\pi i}\oint\left\langle\left(T(z)\epsilon(z)dz+\bar T(\bar z)\bar \epsilon(\bar z)d\bar z\right)\Phi(x_{1})\cdots\Phi(x_{i})\cdots\Phi(x_{n})\right\rangle
	}
\end{equation}
这里积分围道包围所有$w_i$。根据上式我们可以定义\textbf{Conformal charge}:
\begin{equation}
	Q\equiv\frac{1}{2\pi i}\oint_{\mathcal{C}}\Bigl(dzT(z)\epsilon(z)+d\overline{z}\overline{T}(\overline{z})\overline{\epsilon}(\overline{z})\Bigr)
\end{equation}
这样便有:
\begin{equation}\label{eq:31.9}
	-\delta_{\epsilon,\bar\epsilon}\Phi=\left[Q,\Phi\right]
\end{equation}
\subsection{Operator Product Expansion}
但是\ref{eq:31.9}表面上有个非常大的漏洞,那就是因为全纯,直接导致围道积分始终为0。解决这一问题是注意到Q是一个算符,需要作用到其它场上面,而两个算符靠的非常近时真空涨落发散\sn{类似于$a_p,a_p^\dagger\sim\delta(0)$},上式积分中出现极点。也正是极点的存在导致我们定义局域算符之间的对易子的时候需要做正规化:
\begin{equation}
	[A(z,\bar{z}),B(w,\bar{w})]=\lim_{\begin{smallmatrix}|z|\rightarrow|w|\\|z|>|w|\end{smallmatrix}}A(z,\bar{z})B(w,\bar{w})-\lim_{\begin{smallmatrix}|z|\rightarrow|w|\\|w|>|z|\end{smallmatrix}}B(w,\bar{w})A(z,\bar{z})
\end{equation}
这直接导致对易子的计算可以等价于对径向顺序积的计算:\sn{后文中$\mathcal{C}(w)$表示只绕$w$的逆时针围道}
\begin{equation}
	\boxed{\begin{gathered}
			\oint dz\bigl[A(z),B(w)\bigr] =\oint_{|z|>|w|}dzA(z)B(w)-\oint_{|z|<|w|}dzB(w)A(z) \\
			=\oint_{\mathcal{C}(w)}dzR\Big(A(z)B(w)\Big), 
	\end{gathered}}
\end{equation}
现在利用只含一个初级场的无穷小变换Ward恒等式,或者说\ref{eq:31.9},注意到\ref{ict},以及:\sn{回忆$n$阶极点留数公式:\[\begin{aligned}
		\operatorname{Res}&[f(z),a]\\=&\lim\limits_{z\to a}\frac{1}{(n-1)!}\frac{d^{n-1}}{dz^{n-1}}\left[(z-a)^nf(z)\right]
	\end{aligned}\]}
\begin{equation}\label{eq:31.12}
	\begin{gathered}
		h\left(\partial_{w}\epsilon(w)\right)\phi(w,\overline{w}) =\frac{1}{2\pi i}\oint_{\mathcal{C}(w)}dzh\frac{\epsilon(z)}{(z-w)^{2}}\phi(w,\overline{w}), \\
		\epsilon(w)\left(\partial_{w}\phi(w,\overline{w})\right) =\frac1{2\pi i}\oint_{\mathcal{C}(w)}dz\frac{\epsilon(z)}{z-w}\partial_{w}\phi(w,\overline{w}).
	\end{gathered}
\end{equation}
得到全纯部分:
\begin{equation}
	R\left(T(z)\phi(w,\overline{w})\right)=\frac{h}{(z-w)^2}\phi(w,\overline{w})+\frac{1}{z-w}\partial_w\phi(w,\overline{w})+\ldots 
\end{equation}
\textbf{后面的省略号表示非奇异的项},大多数情况下我们要用的是OPE的奇异部分,因为它完全包含了和对易子一样的信息。像这样的把算符展开成其它算符和的形式称为\textbf{OPE}。在一般的QFT中我们计算时序积需要先写下路径积分,然后进行微扰展开算费曼图,但是CFT中我们只需要去考虑OPE就好了,对称性为我们简化了许多。后面写OPE会省略$R(\cdot)$,利用算符和能动张量之间的OPE我们还能给出一个初级场的等价定义:
\begin{theorem}
	一个场是共形权为$(h,\bar h)$的初级场\textbf{当且仅当}它满足:
	\begin{equation}\label{31.14}
		\boxed{\begin{gathered}
				T(z)\phi(w,\overline{w}) =\frac{h}{(z-w)^{2}}\phi(w,\overline{w})+\frac{1}{z-w}\partial_{w}\phi(w,\overline{w})+\ldots  \\
				\overline{T}(\overline{z})\phi(w,\overline{w}) =\frac{\overline{h}}{(\overline{z}-\overline{w})^{2}}\phi(w,\overline{w})+\frac{1}{\overline{z}-\overline{w}}\partial_{\overline{w}}\phi(w,\overline{w})+\ldots  
		\end{gathered}}
	\end{equation}
\end{theorem}
再次强调,之后写下的所有的OPE都应当认为是在在时序关联函数中插入算符时成立的关系!而且因为是时序积之中的关系:
\begin{equation}
	\langle\mathcal{O}_i(z,\bar{z})\mathcal{O}_j(w,\bar{w})\ldots\rangle=\sum_kC_{ij}^k(z-w,\bar{z}-\bar{w})\langle\mathcal{O}_k(w,\bar{w})\ldots\rangle 
\end{equation}
所以OPE结合律和交换律都自动满足。上式中$\ldots$表示其它任意算符的插入,当然既然是级数总有一个收敛半径,何时OPE的插入是有效的是我们需要关心的,从前面对OPE的推导可以看出我们要求积分围道中只有$w$这一个奇点,所以OPE是精确的当且仅当其它算符的插入点$\{w_i\}$满足$|w-w_i|>|z-w|$。

现在再回到Ward恒等式\ref{ward},但是现在考虑有多个$\Phi$的插入,从前面对OPE的分析得知围道内的极点是$\{w_i\}$,然后利用柯西定理将等式右边的积分换成对绕每个极点的围道积分求和得到:
\begin{equation}
	\begin{aligned}0=\oint\frac{dz}{2\pi i}\epsilon(z)&\Bigg[\Big\langle T(z)\phi_1(w_1,\overline{w}_1)\ldots\phi_N(w_N,\overline{w}_N)\Big\rangle\\
		&-\sum_{i=1}^N\left(\frac{h_i}{(z-w_i)^2}+\frac{1}{z-w_i}\partial_{w_i}\right)\Big\langle\phi_1(w_1,\overline{w}_1)\ldots\phi_N(w_N,\overline{w}_N)\Big\rangle\Bigg]\end{aligned}
\end{equation}
等式左边再次利用\ref{eq:31.12},得到所谓共形Ward恒等式(仅适用于初级场):\sn{注意这并不是代入$T,Phi$之间OPE得到的,因为前面说过OPE收敛半径等于与OPE处最近的其他算符插入处的距离。而我们下面要导出的等式与算符插入点完全无关,推导过程中看到我们将围道变形为单独绕每个$\{w_i\}$的围道来使得OPE的插入是精确的。}
\begin{equation}
	\boxed{\begin{aligned}\big\langle T\left(z\right)&\phi_{1}(w_{1},\overline{w}_{1})\ldots\phi_{N}(w_{N},\overline{w}_{N})\big\rangle\\&=\sum_{i=1}^{N}\left(\frac{h_{i}}{(z-w_{i})^{2}}+\frac{1}{z-w_{i}}\partial_{w_{i}}\right)\Big\langle\phi_{1}(w_{1},\overline{w}_{1})\ldots\phi_{N}(w_{N},\overline{w}_{N})\Big\rangle\end{aligned}}
\end{equation}
\subsubsection{TT OPE}
CFT$_{2}$中非常重要的OPE是两个能动张量之间的OPE,就平面上的二维共形场论而言,chiral和anti-chiral部分是完全解耦的,所以:
\begin{equation}
	\boxed{
		T(z)\bar T(\bar w)=0
	}
\end{equation}
回忆能动张量$T\sim\frac{\delta S}\delta{g}$,现在来数质量量纲,由于$S$最终要放到指数上,所以$[S]=0$,再根据泛函导数定义$\frac{\delta f(x)}{\delta f(y)}=\delta^D(x-y)$,所以$[f]+\left[\frac{\delta }{\delta f}\right]=[\delta^D]=D$\sn{最后一个等号是因为$\int{dx^D}\delta^D(x-y)=1$,$[x]=-1$}。由于$[g]=0$,所以$[T]=2$,这暗示着在dilation下$T$的共形权为$(2,0)$,根据\ref{31.14}得到:
\begin{equation}
	T(z)T(w)=\cdots+\frac{2T(w)}{(z-w)^2}+\frac{\partial T(w)}{z-w}+\cdots
\end{equation}
第一个省略号是因为$T$不一定是初级场!第二个省略号表示奇异项。但是由于$[TT]=4$,所以最奇异的项只能是$(z-w)^{-4}$,而且分子只能是一个常数,我们取归一化为$c/2$得到:
\begin{theorem}[TT OPE]
	\begin{equation}\label{TT}
		\boxed{
			T(z)T(w)=\frac{c/2}{(z-w)^4}+\frac{2T(w)}{(z-w)^2}+\frac{\partial_wT(w)}{z-w}+\ldots 
		}
	\end{equation}
	$\bar T\bar T$的OPE取个conjugate就好。
\end{theorem}
有两点需要解释一下,首先是为何没有$(z-w)^{-3}$项的加入?这可以解释为由于OPE的交换律,所以$z\leftrightarrow w$对称性会紧闭所有奇数次数项。但是有一个例外是$(z-w)^{-1}$:
\begin{equation}
	\begin{aligned}T(w)T(z)&=\frac{c/2}{(z-w)^4}+\frac{2T(w)+2(z-w)\partial T(w)}{(z-w)^2}-\frac{\partial T(w)}{z-w}+\ldots=T(z)T(w)\end{aligned}
\end{equation}
可见$(z-w)^{-1}$的对称性被更低一阶项分子的泰勒展开保护,但是$(z-w)^{-3}$的更低一阶分子是常数,泰勒展开是trivial的。

虽然$T$并非初级场,但是是准初级场,因为$\{L_0,L_{\pm1}\}$的中心扩张是trivial的。虽然不是初级场,但是能动张量在$z\mapsto f(z)$的共形变换下的行为可以又下式描述:
\begin{theorem}
	\begin{equation}\label{schwarzian}
		\boxed{
			T^\prime(f(z))=\left(\frac{\partial f(z)}{\partial z}\right)^{-2}\begin{bmatrix}T(z)-\frac{c}{12}S(f(z),z)\end{bmatrix}
		}
	\end{equation}
	其中$S(f,z)$是Schwarzian导数,定义为:
	\begin{equation}
		\boxed{
			S(w,z)=\frac1{(\partial_zw)^2}\Big(\big(\partial_zw\big)\big(\partial_z^3w\big)-\frac32\big(\partial_z^2w\big)^2\big)
		}
	\end{equation}
\end{theorem}
观察一下无穷小共形变换下能动张量的性质:
\begin{equation}
	\begin{aligned}
		-\delta_{\epsilon}T(z)&=[Q,T]\begin{aligned}=\frac{1}{2\pi i}\oint_{\mathcal{C}(z)}dw\epsilon(w)T(w)T(z)\end{aligned}  \\
		&=\frac{1}{2\pi i}\oint_{\mathcal{C}(z)}dw\epsilon(w)\left(\frac{c/2}{(w-z)^{4}}+\frac{2T(z)}{(w-z)^{2}}+\frac{\partial_{z}T(z)}{w-z}+\ldots\right) \\
		&=\frac{c}{12}\partial_{z}^{3}\epsilon(z)+2T(z)\partial_{z}\epsilon(z)+\epsilon(z)\partial_{z}T(z).
	\end{aligned}
\end{equation}
不难证明在无穷小变换下\ref{schwarzian}的正确性。现在来考虑$T$模展开的对易子:
\begin{equation}
	T(z)=\sum_{n\in\mathbb{Z}}z^{-n-2}L_n\quad\text{where}\quad L_n=\frac{1}{2\pi i}\oint dzz^{n+1}T(z)
\end{equation}
\begin{proof}
	\begin{equation}
		\begin{aligned}
			\begin{bmatrix}L_{m},L_{n}\end{bmatrix}=&\oint\frac{dz}{2\pi i}\oint\frac{dw}{2\pi i}z^{m+1}w^{n+1}\bigl[T(z),T(w)\bigr]  \\
			=&\oint_{\mathcal{C}(0)}\frac{dw}{2\pi i}w^{n+1}\oint_{\mathcal{C}(w)}\frac{dz}{2\pi i}z^{m+1}R\Big(T(z)T(w)\Big) \\
			=&\oint_{\mathcal{C}(0)}\frac{dw}{2\pi i}w^{n+1}\oint_{\mathcal{C}(w)}\frac{dz}{2\pi i}z^{m+1}\bigg(\frac{c/2}{(z-w)^4}+\frac{2T(w)}{(z-w)^2}+\frac{\partial_wT(w)}{z-w}\bigg) \\
			=&\oint_{\mathcal{C}(0)}\frac{dw}{2\pi i}w^{n+1}\Big((m+1)m(m-1)w^{m-2}\frac c{2\cdot3!} \\
			&+2\left(m+1\right)w^m\left.T(w)+w^{m+1}\partial_wT(w)\right) \\
			=&\oint\frac{dw}{2\pi i}\left(\frac c{12}(m^3-m)\mid w^{m+n-1}\right. \\
			&\left.+2(m+1)w^{m+n+1}T(w)+w^{m+n+2}\partial_{w}T(w)\right) \\
			=&\frac c{12}\bigl(m^3-m\bigr)\delta_{m,-n}+2(m+1)L_{m+n} \\
			&+0-\underbrace{\oint\frac{dw}{2\pi i}(m+n+2)T(w)w^{m+n+1}}_{=(m+n+2)L_{m+n}}\\
			=&(m-n)~L_{m+n}+\frac c{12}\Big(m^3-m\Big)~\delta_{m,-n}
		\end{aligned}
	\end{equation}
	上面这一套是非常一般的做法。
\end{proof}
所以能动张量的Laurent模满足的是Virasoro代数。将这一套做法放到\ref{31.14}得到:
\begin{equation}\label{31.27}
	\boxed{\left[L_m,\phi_n\right]=\left((h-1)m-n\right)\phi_{m+n}}
\end{equation}
\subsection{n-point correlators (n<4)}
利用共形对称性实际上可以完全确定四点以下关联函数的形式。先来考虑准初级chiral场的关联函数,前面考虑的都是无穷小变换的Ward恒等式,但实际上在这里我们要使用的只是有限global共形变换的Ward恒等式,一般形式如下:
\begin{equation}\label{global ward}
	\boxed{
	\langle\phi_1'(x_1')\cdots\phi_N'(x_N')\rangle=\langle\phi_1(x_1')\cdots\phi_N(x_N')\rangle
	}
\end{equation}
上式等价于:
\begin{equation}
	\left\langle\phi_{1}\left(z_{1},\bar{z}_{1}\right)\cdots\phi_{n}\left(z_{n},\bar{z}_{n}\right)\right\rangle=\prod_{i=1}^{n}\left(\frac{\partial z'}{\partial z}\right)^{h_{i}}\left(\frac{\partial\bar{z}'}{\partial\bar{z}}\right)^{\bar{h}_{i}}\Bigg|_{z=z_{i},\bar{z}=\bar{z}_{i}}\left\langle\phi_{1}\left(z'_{1},\bar{z}'_{1}\right)\cdots\phi_{n}\left(z'_{n},\bar{z}'_{n}\right)\right\rangle 
\end{equation}
可以通过考察路径积分轻易得到。利用这一等式就可以得到一点关联函数必定是trivial的,这也可以根据前面的态算符对应要求的\ref{eq:30.6}\ref{eq:30.17}得到。
\begin{theorem}[2-pt]
	\begin{equation}\label{31.19}
		\left\langle\phi_i(z)\phi_j(w)\right\rangle=\frac{d_{ij}\delta_{h_i,h_j}}{(z-w)^{2h_i}}
	\end{equation}
\end{theorem}
\begin{theorem}[3-pt]
	\begin{equation}\label{31.20}
		\left\langle\phi_1(z_1)\phi_2(z_2)\phi_3(z_3)\right\rangle=\frac{C_{123}}{z_{12}^{h_1+h_2-h_3}z_{23}^{h_2+h_3-h_1}z_{13}^{h_1+h_3-h_2}}
	\end{equation}
	其中$z_{ij}\equiv z_i-z_j$
\end{theorem}
\begin{remark}
	$d_{\phi\phi},C_{123}$可以用关联函数写成下面的形式:
	\begin{equation}
		d_{\phi\phi}=\braket{\phi}{\phi}
	\end{equation}
	\begin{equation}
		C_{123}=\langle0|\phi_{(1)+h_1}\phi_{(2)h_3-h_1}\phi_{(3)-h_3}|0\rangle =\bra{\phi_1}\phi_{(2)h_3-h_1}\ket{\phi_3}
	\end{equation}
	这里$\phi_{(i)n}$表示$\phi_i$的第$n$个Laurent模。
\end{remark}
另外根据关联函数的单值性,还进一步要求\sn{后面考虑Virasoro代数表示的时候我们会先暂时忘掉这个要求。}\textbf{Chiral场的共形权只能是整数或者半整数}。把反全纯部分也加进来考虑得到:
\begin{itemize}
	\item 2-pt:
	\begin{equation}
		\left\langle\phi_1(z_1,\overline{z}_1)\phi_2(z_2,\overline{z}_2)\right\rangle=\frac{d_{12}}{z_{12}^{h_1+h_2}\overline{z}_{12}^{\overline{h}_1+\overline{h}_2}}\delta_{h_1,h_2}\delta_{\overline{h}_1,\overline{h}_2}
	\end{equation}
	\item 3-pt:
	\begin{equation}
		\begin{aligned}\big\langle\phi_1(z_1,\overline{z}_1)&\phi_2(z_2,\overline{z}_2)\phi_3(z_3,\overline{z}_3)\big\rangle\\&=\frac{C_{123}}{z_{12}^{h_1+h_2-h_3}z_{23}^{h_2+h_3-h_1}z_{13}^{h_1+h_3-h_2}\overline{z}_{12}^{\vec{h}_1+\vec{h}_2-\vec{h}_3}\overline{z}_{23}^{\vec{h}_2+\vec{h}_3-\vec{h}_1}\overline{z}_{13}^{\vec{h}_1+\vec{h}_3-\vec{h}_2} }\end{aligned}
	\end{equation}
\end{itemize}
根据OPE的交换性和结合性可以知道$C_{123}=C_{231}=C_{312}$。

关于这些公式的详细推导\cite{ito}上面都有,这里主要考虑用无穷小Ward恒等式的方法,但是我们考虑的毕竟是准初级场,所以只需要考虑$\epsilon(z)=\epsilon_{-1}+\epsilon_0z+\epsilon_1z^2$。将\ref{global ward}在无穷小变换下展开得到全纯部分:
\begin{equation}
	\displaystyle\sum_{i=1}^{n}\left(h_{i}\partial_{i}\epsilon\left(z_{i}\right)+\epsilon\left(z_{i}\right)\partial_{i}\right)\left<\phi_{1}\left(z_{1}\right)\cdots\phi_{n}\left(z_{n}\right)\right>=0+\mathcal{O}(\epsilon^2)
\end{equation}
这是一个微分方程,对于$\epsilon(z)=\epsilon_{-1}+\epsilon_0z+\epsilon_1z^2$恒成立,所以我们可以通过求解这个微分方程来确定关联函数的形式,不难验证\ref{31.19}和\ref{31.20}就有这一特征。这种将关联函数计算转化为微分方程的求解的方法后面我们还会碰到。

但是四点及以上函数就不能确定到只差一个待定系数了,不过我们仍旧可以利用所谓\textbf{交叉对称性}(Crossing Symmetry)走的很远。

\subsection{General Form of the OPE}
为了简化讨论,先从chiral场开始,$\{\phi_i\}$表示CFT$_2$的Spectrum,下标$i$表示共形权为$h_i$,则OPE有下面的一般形式:
\begin{theorem}
	\begin{equation}
		\boxed{
		\phi_i(z)\phi_j(w)=\sum_{k,n\geq0}C_{ij}^k\frac{a_{ijk}^n}{n!}\frac{1}{(z-w)^{h_i+h_j-h_k-n}}\partial^n\phi_k(w)
		}
	\end{equation}
	其中:
	\begin{equation}\label{31.26}
		\begin{aligned}
			&a_{ijk}^{n} =\binom{2h_k+n-1}n^{-1}\binom{h_k+h_i-h_j+n-1}n,  \\
			&C_{ijk} =\sum_l{C_{ij}^{l}d_{lk}}
		\end{aligned}
	\end{equation}
\end{theorem}
由于1-pt关联函数始终为0,所以上面的OPE对两点关联函数的贡献只能是$n=0$且$h=0$的常值初级场贡献,也即$\left\langle\phi_1\phi_2\right\rangle\propto\frac{\delta_{h_1,h_2}}{z_{12}^{h_1+h_2}}$。
\begin{proof}
	考虑下面的关联函数:
	\begin{equation}
		\Big\langle\left(\phi_i(z)\phi_j(1)\right)\phi_k(0)\Big\rangle=\sum_{l,n\geq0}C_{ij}^l\frac{a_{ijl}^n}{n!}\frac{1}{(z-1)^{h_i+h_j-h_l-n}}\Big\langle\partial^n\phi_l(1)\phi_k(0)\Big\rangle 
	\end{equation}
	这里OPE的插入有效已经暗示$|z-1|<|0-1|=1$。利用两点关联函数的一般形式得到:
	\begin{equation}
		\left.\left\langle\partial_{z}^{n}\phi_{l}(z)\phi_{k}(0)\right\rangle\right|_{z=1}=\partial_{z}^{n}\left.\left(\frac{d_{lk}\delta_{h_{l},h_{k}}}{z^{2h_{k}}}\right)\right|_{z=1}=(-1)^{n}n!\left(\begin{matrix}2h_{k}+n-1\\n\end{matrix}\right)d_{lk}\delta_{h_{l},h_{k}}
	\end{equation}
	而3-pt关联函数的一般形式前面已经给出:
	\begin{equation}
			\Big\langle\left(\phi_i(z)\phi_j(1)\right)\phi_k(0)\Big\rangle=\frac{C_{ijk}}{(z-1)^{h_i+h_j-h_k}z^{h_i+h_k-h_j}}
	\end{equation}
	自洽性直接要求:
	\begin{equation}\label{31.30}
		\sum\limits_{l,n\geq0}C_{ij}^ld_{lk}a_{ijk}^n\binom{2h_k+n-1}{n}(-1)^n(z-1)^n\overset{!}{=}\frac{C_{ijk}}{\left(1+(z-1)\right)^{h_i+h_k-h_j}}
	\end{equation}
	由于$|z-1|<1$,所以可对右边利用下面的广义二项式定理进行展开:
	\begin{equation}
		\frac1{(1+x)^H}=\sum_{n=0}^\infty(-1)^n\binom{H+n-1}nx^n
	\end{equation}
	比较\ref{31.30}两边就得到了系数\ref{31.26},这其实就是Bootstrap的思想。
\end{proof}
Laurent模之间的一般对易关系形式为:
\begin{theorem}
	\begin{equation}
		\boxed{\left[\phi_{(i)m},\phi_{(j)n}\right]=\sum_kC_{ij}^kp_{ijk}(m,n)\phi_{(k)m+n}+d_{ij}\delta_{m,-n}\binom{m+h_i-1}{2h_i-1}}
	\end{equation}
	其中:
	\begin{equation}
		\begin{aligned}p_{ijk}(m,n)&=\sum_{\substack{r,s\in\mathbb{Z}_0^+\\r+s=h_i+h_j-h_{k-1}}}C_{r,s}^{ijk}\cdot\binom{-m+h_i-1}{r}\cdot\binom{-n+h_j-1}{s},\\C_{r,s}^{ijk}&=(-1)^r\frac{(2h_k-1)!}{(h_i+h_j+h_k-2)!}\prod_{t=0}^{s-1}(2h_i-2-r-t)\prod_{u=0}^{r-1}(2h_j-2-s-u)\end{aligned}
	\end{equation}
	这里$\mathbb{Z_0^+}$的意思是正整数以及0,后面的乘积$\prod$在$s=0(\text{or }r=0)$时定义为1,上面的系数形式还有一个非常重要的信息,就是只有当$h_k<h_i+h_j$时才有贡献。
\end{theorem}
通过共形对称性得到了OPE非常多的信息,唯一没有确定下来的只是一些系数,这就必须要CFT本身的一些性质了\sn{比如能动张量的形式,以及其它的对称性}。

更一般的非手征(准)初级场OPE的一般表达式更加复杂,形式为:
\begin{equation}
	\boxed{
		\phi_i(z,\overline{z})\phi_j(w,\overline{w})=\sum_{p}\sum_{\{k,\overline{k}\}}C_{ij}^p\frac{\beta_{ij}^{p,\{k\}}\overline{\beta}_{ij}^{p,\{\overline{k}\}}\phi_{p}^{\{k,\overline{k}\}}(w,\overline{w})}{(z-w)^{h_i+h_j-h_p-K}(\overline{z}-\overline{w})^{\overline{h}_i+\overline{h}_j-\overline{h}_p-\overline{K}}}
	}
\end{equation}
这里:
\begin{equation}
	\{\hat{L}_{-k_1}\ldots\hat{L}_{-k_n}\hat{\overline{L}}_{-k_1}\ldots\hat{\overline{L}}_{-k_n}\phi_p(z,\overline{z})\},\quad K\equiv\sum_i k_i
\end{equation}
$\beta_{ij}^{p,\{k\}}$和前面的$a^n_{ijk}$一样是之和初级场的共形权有关的,是与理论本身无关的可以计算出来的系数,我们唯一不知道的就是$C_{ij}^p$。
\subsubsection{Current Algebra}
\begin{definition}
	共形权为1的chiral(anti-chiral)场$j(z)$称为流。
\end{definition}
考虑一个谱为$N$个流的理论,根据前面Laurent模的一般形式得到:
\begin{equation}\label{current}
	\begin{bmatrix}j_{(i)m},j_{(j)n}\end{bmatrix}=\sum_kC_{ij}^kp_{111}(m,n)j_{(k)m+n}+d_{ij}m\delta_{m,-n}
\end{equation}
$d_{ij}$在$i\neq j$时为0,但是不同的$ij$对应的比例系数不同。我们总是可以通过rescal这些流,使得$d_{ij}=k\delta_{i,j}$,然后再通过对流进行重新组合得到:
\begin{equation}
	\boxed{
		\begin{bmatrix}j_m^i,j_n^j\end{bmatrix}=i\sum_lf^{ijl}j_{m+n}^l+km\delta^{ij}\delta_{m,-n}
	}
\end{equation}
这些流的模构成了一个Lie代数,显然中心荷non-trivial,后面会进一步详细讨论。它实际上是仿射化之后得到的{\itshape Ka\v{c}-Moody}代数。
\section{Crossing Symmetry}
本节的目的是最大程度得到四点关联函数$\langle\phi_1(z_1,\bar{z}_1)\phi_2(z_2,\bar{z}_2)\phi_3(z_3,\bar{z}_3)\phi_4(z_4,\bar{z}_4)\rangle $的信息。$n<4$的可以完全确定从数学上看是因为黎曼球面上任意两个或三个点,都可以通过共形变换映射为固定的$\{0,1,\infty\}$,但是四点我们最多只能将他们映射到$\{0,x,1,\infty\}$,其中$x$是\textbf{交比(Crossing Ratio)}。
\begin{equation}
	x=\frac{(z_1-z_2)(z_3-z_4)}{(z_1-z_3)(z_2-z_4)},\quad\bar{x}=\frac{(\bar{z}_1-\bar{z}_2)(\bar{z}_3-\bar{z}_4)}{(\bar{z}_1-\bar{z}_3)(\bar{z}_2-\bar{z}_4)}
\end{equation}
所以四点关联函数用前面的方法只能确定到一个与交比有关的函数。利用\ref{global ward},我们知道任意四点关联函数可以与下面的关联函数联系起来:\sn{$z\mapsto \frac{(z_1-z)(z_3-z_4)}{(z-z_4)(z_1-z_3)}$}
\begin{equation}
	\propto \langle\phi_i(0,0)\phi_j(x,\bar x)\phi_l(1,1)\phi_m(\infty,\infty)\rangle\equiv G^{ij}_{lm}(x,\bar x)
\end{equation}
由于OPE由结合律,所以可以前两个先做OPE,后两个在做OPE,注意到$\braket{\phi_p}{\phi_q}\sim\delta_{h_p,h_q}$,我们可以将上式重写为:
\begin{equation}\label{32.3}
	G^{ij}_{lm}(x,\bar x)=\sum_pC_{ij}^pC_{lm}^p\mathcal{F}_{ij}^{lm}(p\mid x)\overline{\mathcal{F}}_{ij}^{lm}(p\mid\overline{x})
\end{equation}
这里的$\mathcal{F}_{ij}^{lm}(p\mid x)$称为\textbf{Conformal Block},关联函数就是由这些Block线性组合而来的。再次利用\ref{global ward},考虑$z\mapsto 1-z$和$z\mapsto 1/z$,有:
\begin{equation}
	G^{ij}_{lm}(x,\bar x)=\langle\phi_i(1,1)\phi_j(1-x,1-\bar x)\phi_l(0,0)\phi_m(\infty,\infty)\rangle= G^{lj}_{im}(1-x,1-\bar x)
\end{equation}
\begin{equation}
	G^{ij}_{lm}(x,\bar x)=x^{2h_j}\bar x^{2\bar h_j}\langle\phi_i(1,1)\phi_j(\frac{1}{x},\frac{1}{\bar x})\phi_l(\infty,\infty)\phi_m(0,0)\rangle= x^{2h_j}\bar x^{2\bar h_j}G^{mj}_{il}(\frac{1}{x},\frac{1}{\bar x})
\end{equation}
这让我们得到另外两个$G(x,\bar x)$的表达式:
\begin{align}
	\label{32.6} G(x,\bar x)&=\sum_pC_{im}^pC_{jl}^p\mathcal{F}_{im}^{jl}(p\mid1-x)\overline{\mathcal{F}}_{im}^{jl}(p\mid1-\overline{x})\\
	\label{32.7} G\left(x,\bar x\right)&=x^{-2h_j}\overline{x}^{-2\overline{h}_j}\sum_pC_{il}^pC_{jm}^p\mathcal{F}_{il}^{jm}\left(p\mid\frac1x\right)\overline{\mathcal{F}}_{il}^{jm}\left(p\mid\frac1{\overline{x}}\right)
\end{align}
这三个等式对应不同顺序的OPE缩并,就像是QFT中费曼图的$s,t,u$三个channel,$p$就像是中间的传播子,共形不变性要求这三个$G(x,\bar x)$的等式殊途同归,这就给出了$C^p_{ij}$必须要满足的一套等式,共形自举的思想就是通过这套方程完全确定OPE系数。
\begin{figure}[htbp]
	\centering
	\includegraphics{figs/fig10.pdf}
	\caption{$G(x)=G(1-x)$ Crossing Symmetry的图示}
\end{figure}
\subsection{Fusing and Braiding Matrices}
考虑一个比较简单的情形,CFT的谱是个有限谱\sn{比如RCFT},这样Conformal Blocks构成了一个有限维线性空间,其中的某个线性组合得到的元素就是我们要的四点关联函数\sn{当然还要把全纯反全纯组合起来}。这其实就是在说\ref{32.3},\ref{32.6}和\ref{32.7}中的Conformal Blocks相差的仅仅是一个线性变换,他们是同一个线性空间的不同基底!
\begin{equation}
	\mathcal{F}_{ij}^{kl}(p\mid x)=\sum_qB{\left[\begin{smallmatrix}j&k\\i&l\end{smallmatrix}\right]}_{p,q}\mathcal{F}_{ik}^{jl}\!\left(q\mid\frac1x\right)
\end{equation}
\begin{equation}
	\mathcal{F}_{ij}^{kl}(p\mid x)=\sum_qF{\left[\begin{array}{cc}j&k\\i&l\end{array}\right]}_{p,q}\mathcal{F}_{il}^{jk}(q\mid1-x)
\end{equation}
这里$B$称作\textbf{Braiding Matrices},$F$称作\textbf{Fusing Materices},$p,q$是标记矩阵元,$\{i,j,l,m\}$标记矩阵本身。我们用类似轮换序图的形式来表达上面的两个方程:
\begin{figure}[H]
	\centering
	\includegraphics[width=0.8\linewidth]{figs/fig11.pdf}
\end{figure}
\begin{figure}[H]
	\centering
	\includegraphics[width=0.8\linewidth]{figs/fig12.pdf}
\end{figure}
利用下面图表的交换性:
\begin{figure}[H]
	\centering
	\includegraphics{figs/fig13.pdf}
\end{figure}
得到hexgon恒等式:
\begin{equation}
	\boxed{
		\sum_{p} B\left[\begin{array}{cc}
			j & k \\
			i & s
		\end{array}\right]_{r p} B\left[\begin{array}{cc}
			j & l \\
			p & m
		\end{array}\right]_{s t} B\left[\begin{array}{cc}
			k & l \\
			i & t
		\end{array}\right]_{p u}=\sum_{q} B\left[\begin{array}{cc}
			k & l \\
			r & m
		\end{array}\right]_{s q} B\left[\begin{array}{cc}
			j & l \\
			i & q
		\end{array}\right]_{r u} B\left[\begin{array}{cc}
			j & k \\
			u & m
		\end{array}\right]_{q t}
	}
\end{equation}
上面的方程非常像Yang-Baxter方程,所以CFT和可积性有很大的关联。但是注意上面这不是证明!就和YBE的图形化规则推导一样,只是便于计算和记忆。同样根据下面的图表交换:
\begin{figure}[H]
	\centering
	\includegraphics{figs/fig14.pdf}
\end{figure}
得到pentagon恒等式:
\begin{equation}
	\boxed{
		 F\left[\begin{array}{cc}
			j & k \\
			i & s
		\end{array}\right]_{r t} F\left[\begin{array}{cc}
			t & l \\
			i & m
		\end{array}\right]_{s u} =\sum_{p} F\left[\begin{array}{cc}
			k & l \\
			r & m
		\end{array}\right]_{s p} F\left[\begin{array}{cc}
			j & p \\
			i & m
		\end{array}\right]_{r u} F\left[\begin{array}{cc}
			j & k \\
			u & l
		\end{array}\right]_{p t}
	}
\end{equation}
\section{NOPs and Conformal family}
\subsection{Normal Ordered Products}
两个算符的OPE在两个算符的插入位置一样时会发散,这种发散就类似于一般QFT中的红外发散,因为空间无限大,所以真空零点能发撒,这种发散可以通过取算符的正规乘积(NOP)得到,共形场论也有这样的做法。我们先考虑\textbf{自由场},也就是$R(\phi\partial^n\phi)$最多只有一项是奇异的,\sn{对到QFT那边就是场的模式展开可以完全分离成产生算符和湮灭算符之和}那么我们直接减去真空期望值就去除了所有的发散项,所以NOP可以定义为:
\begin{definition}[NOPs]
	\begin{equation}
		:\phi_1(z)\phi_2(w):\equiv\left(\phi_1(z)\phi_2(w)-\left\langle{\phi_1(z)\phi_2(w)}\right\rangle\right):
	\end{equation}
	这里省略了所有的径向排序,记收缩:
	\begin{equation}
		\wick{\c \phi(z)\c \phi(w)}\equiv \left\langle{\phi_1(z)\phi_2(w)}\right\rangle
	\end{equation}
	缩并后是一个数,可以提出去。显然在$z\to w$后OPE的所有奇异性都被减除:
	\begin{equation}
		:\phi_1\phi_2:(w)\equiv\lim_{z\to w}\left(:\phi_1(z)\phi_2(w):\right)
	\end{equation}
\end{definition}
\begin{theorem}[Wick]
	\begin{equation}
		\boxed{
			\begin{aligned}
				\mathsf R[\Phi_{a_1}(x_1)\Phi_{a_2}(x_2)\cdots\Phi_{a_n}(x_n)]=\mathsf N\big[&\Phi_{a_1}(x_1)\Phi_{a_2}(x_2)\cdots\Phi_{a_n}(x_n)\\&+\left(\Phi_{a_1}\Phi_{a_2}\cdots\Phi_{a_n}\text{的所有可能缩并}\right)\big]
			\end{aligned}
		}
	\end{equation}
	由于正规序插入到真空态之间恒为0,所以任何没有完全缩并的项对真空期望值都没有贡献。
\end{theorem}
\begin{proof}
	关于Wick定理的证明可见任何一本正则量子化的QFT教材,比如余钊焕老师的讲义\footnote{\url{http://yzhxxzxy.github.io/cn/teaching.html}},这里只想强调定理本身依赖于前面对自由场的NOP定义。
\end{proof}
\begin{example}
	\begin{equation}
		\begin{aligned}\mathsf{R}(\Phi_a\Phi_b\Phi_c\Phi_d)&=\mathsf{N}\big(\Phi_a\Phi_b\Phi_c\Phi_d+\wick{\c \Phi_a\c\Phi_b\Phi_c\Phi_d}+\wick{\c\Phi_a\Phi_b\c\Phi_c\Phi_d}+\wick{\c\Phi_a\Phi_b\Phi_c\c\Phi_d}\\&\quad\quad+\wick{\Phi_a\c\Phi_b\c\Phi_c\Phi_d}+\wick{\Phi_a\c\Phi_b\Phi_c\c\Phi_d}+\wick{\Phi_a\Phi_b\c\Phi_c\c\Phi_d}\\&\quad\quad+\wick{\c\Phi_a\c\Phi_b}\wick{\c\Phi_c\c\Phi_d}+\wick{\c1\Phi_a\c2\Phi_b\c1\Phi_c\c2\Phi_d}+\wick{\c1\Phi_a\c2\Phi_b\c2\Phi_c\c1\Phi_d}\big)\end{aligned}
	\end{equation}
	计算的下一步是将所有收缩的项放在一起,未收缩的项放在一起,对于玻色场由于满足对易关系可以随便移动场的位置,这一步是trivial的,不过费米场因为满足反对易关系,所以交换两个场位置时会产生一个负号需要额外注意。
	
	左边减去右边,并取$x_a,x_b\to x$和$x_c,x_d\to y$的极限后得到两个NOP的时序积的Wick定理:
	\begin{equation}\label{33.6}
		\begin{aligned}
			\mathsf{R}\left[:\Phi_a\Phi_b:(x):\Phi_c\Phi_d:(y)\right]&=\mathsf{N}\big(\Phi_a\Phi_b\Phi_c\Phi_d+\wick{\c\Phi_a\Phi_b\c\Phi_c\Phi_d}+\wick{\c\Phi_a\Phi_b\Phi_c\c\Phi_d}\\&\quad\quad+\wick{\Phi_a\c\Phi_b\c\Phi_c\Phi_d}+\wick{\Phi_a\c\Phi_b\Phi_c\c\Phi_d}\\&\quad\quad+\wick{\c1\Phi_a\c2\Phi_b\c1\Phi_c\c2\Phi_d}+\wick{\c1\Phi_a\c2\Phi_b\c2\Phi_c\c1\Phi_d}\big)
		\end{aligned}
	\end{equation}
	这个计算实际上是非常一般的,他告诉我们计算NOP的时序积时跟没有NOP时一样计算,但是认为NOP内部的场之间的收缩为0。
\end{example}
\begin{remark}
	下面是一些比较有用的缩并等式:
	\begin{equation}
		\wick{\c A(z)\c{:B^{n}:}}(w)=n\wick{\c A(z)\c B(w)}:B^{n-1}(w):
	\end{equation}
	\begin{equation}
		\wick{\c A(z):\c {e}^{B(w)}:}=\wick{\c A(z)\c B(w)}:e^{B(w)}:
	\end{equation}
	\begin{equation}\label{33.9}
		\begin{aligned}
			\wick{:\c{e}^{A(z)}::\c{e}^{B(z)}:}&=\sum_{m,n,k}\frac{k!}{m!n!}\begin{pmatrix}m\\k\end{pmatrix}\begin{pmatrix}n\\k\end{pmatrix}[\wick {\c A(z)\c B(w)}]^k:A^{m-k}(w)B^{n-k}(w):\\&=\exp\left\{\wick{\c A(z)\c B(w)}\right\}:e^{A(w)}e^{B(w)}:\end{aligned}
	\end{equation}
\end{remark}
但是对于非自由场,NOP不能这么定义,比如$TT$的NOP照此定义,仍然发散,所以对于一般场的NOP应当定义成减去所有的奇异项。
\begin{definition}
	\begin{equation}
		\left(A(z)B(w)\right)\equiv A(z)B(w)-\wick{\c A(z)\c B(w)}
	\end{equation}
	这里$\wick{\c A(z)\c B(w)}$表示OPE的\textbf{所有}奇异部分。注意我们引入了文献中另外一种表示NOP的方式:$\left(\cdot\right)$。同样,定义:
	\begin{equation}
		\left(AB\right)(w)=\lim_{z\to w} \left(A(z)B(w)\right)
	\end{equation}
\end{definition}
将NOP其中一个场做洛朗展开,然后可以将OPE用NOP表达为:
\begin{theorem}[NOPs $\iff$ OPEs]
	\begin{equation}
		\phi(z)\chi(w)=\operatorname{sing.}+\sum_{n=0}^\infty\frac{(z-w)^n}{n!}N\big(\partial^n\phi\chi\big)(w)
	\end{equation}
	同样可以用OPE表达NOP:
	\begin{equation}
		N\left(\phi\chi\right)(w)=\oint_{\mathcal{C}(w)}\frac{dz}{2\pi i}\frac{\phi(z)\chi(w)}{z-w}
	\end{equation}
\end{theorem}
考虑NOP的Laurent模:
\begin{equation}
	\begin{aligned}N\left(\phi\chi\right)&(w)=\sum_{n\in\mathbb{Z}}w^{-n-h^\phi-h^\chi}N\left(\phi\chi\right)_n,\\N\left(\phi\chi\right)_n&=\oint_{\mathcal{C}(0)}\frac{dw}{2\pi i}w^{n+h^\phi+h^\chi-1}N\left(\phi\chi\right)(w)\end{aligned}
\end{equation}
\begin{theorem}
	\begin{equation}
		\boxed{
		N\left(\phi\chi\right)_{n}=\sum_{k>-h^{\phi}}\chi_{n-k}\phi_{k}+\sum_{k\leq-h^{\phi}}\phi_{k}\chi_{n-k}
		}
	\end{equation}
\end{theorem}
\begin{proof}
	\begin{equation}
		\begin{aligned}N\left(\phi\chi\right)_n&=\oint_{C(0)}\frac{dw}{2\pi i}w^{n+h^\phi+h^{\lambda}-1}\oint_{C(w)}\frac{dz}{2\pi i}\frac{\phi(z)\chi(w)}{z-w}\\&=\underbrace{\oint\frac{dw}{2\pi i}w^{n+h^\phi+h^{\lambda}-1}\Big(\underbrace{\oint_{|z|>|w|}\frac{dz}{2\pi i}\frac{\phi(z)\chi(w)}{z-w}}_{\mathcal{I}_1}}_{\mathcal{I}_2}-\oint_{|z|<|w|}\frac{dz}{2\pi i}\frac{\chi(w)\phi(z)}{z-w}\Big)\end{aligned}
	\end{equation}
	计算$\mathcal{I}_1$得到:
	\begin{equation}
		\begin{aligned}
			\mathcal{I}_{1}& =\oint_{|z|>|w|}\frac{dz}{2\pi i}\frac{1}{z-w}\sum_{r,s}z^{-r-h^{\phi}}w^{-s-h^{\chi}}\phi_{r}\chi_{s}  \\
			&=\oint_{|z|>|w|}\frac{dz}{2\pi i}\left.\frac1z\sum_{p\geq0}\left(\frac wz\right)^p\sum_{r,s}z^{-r-h^\phi}w^{-s-h^\chi}\phi_r\chi_s\right.  \\
			&=\oint_{|z|>|w|}\frac{dz}{2\pi i}\sum_{p\geq0}\sum_{r,s}z^{-r-h^\phi-p-1}w^{-s-h^\chi+p}\phi_r\chi_s.
		\end{aligned}
	\end{equation}
	频繁利用Cauchy积分公式得到:
	\begin{equation}
		\begin{aligned}
			\mathcal{I}_2&=\oint\frac{dw}{2\pi i}\sum_{p\geq0}\sum_sw^{-s-h^x+p+n+h^\phi+h^x-1}\phi_{-h^\phi-p}\chi_s\\
			&=\sum_{p\geq0}\phi_{-h^\phi-p}\chi_{h^\phi+n+p}=\sum_{k\leq-h^\phi}\phi_k\chi_{n-k}
		\end{aligned}
	\end{equation}
	对另外一项也同样计算就可以证明这个及其重要的等式了。
\end{proof}
这个等式其实非常好理解,从前面对CFT的希尔伯特空间构造可以看到,一个场的Laurent模当$n>-h$时是湮灭算符,$n\leq-h$时是产生算符,所以这个等式的含义就是相对$\phi$而言将所有的产生算符排在前面,湮灭算符放在后面,这正是在QFT中对NOP最naive的定义!

另外,NOP与OPE最大的不同是他并不满足交换律和结合律,这也意味着对多个算符的NOP定义有歧义\sn{对于自由场,前面的定义还是well-define的},往往选取的是下面的嵌套定义:
\begin{equation}
	(ABC\cdots DE)\equiv(A(B(C(\cdots(DE))))))
\end{equation}
另外,Wick定理也失效了,不过我们有下面的等式:
\begin{theorem}[Generalized Wick]
	\begin{equation}\label{wick}
		\boxed{
			{A(z)(BC)}(w)=\frac1{2\pi i}\oint_{\mathcal{C}(w)}\frac{dx}{x-w}\left\{\wick{\c A(z)\c B}(x)C(w)+B(x)\wick{\c A(z)\c C}(w)\right\}+\text{regular}
		}
	\end{equation}
\end{theorem}
上式是对自由场的下面特殊的Wick定理的相互作用推广:
\begin{equation}
	\phi_1(x):\phi_2\phi_3:(y)=\wick{\c\phi_1(x)\c\phi_2(y)}:\phi_3(y):+\wick{\c\phi_1(x)\c\phi_3(y)}:\phi_2(y):+\underbrace{:\phi_1(x)\phi_2(y)\phi_3(y):}_{\text{regular}}
\end{equation}
还可以证明,虽然结合律失效了,但是有下面等式:
\begin{theorem}[Rearrangement Lemma]
	\begin{equation}
		((AB)E)-(A(BE))=(A([E,B]))+(([E,A])B)+([(AB),E])
	\end{equation}
	$E$一般取为$(CD)$。
\end{theorem}
$T$不是初级场,$N(TT)$也不是初级场,但是下面定义的$\mathcal{N}(TT)$是共形权为$(4,0)$的初级场:
\begin{equation}
	\boxed{
		\mathcal{N}(TT)=N\left(TT\right)-\frac3{10}\partial^2T
	}
\end{equation}
后面会用到这个式子。
\subsection{Verma module}
从真空态出发,将$L_n$作为产生算符作用上去,得到的一系列态成为Verma模:
\begin{definition}[Verma Module]
	\begin{equation}
		\boxed{
			\left\{L_{k_1}\ldots L_{k_n}|0\rangle|k_i\leq-2\right\}
		}
	\end{equation}
	中的元素称为Verma Module。
\end{definition}
根据态算符对应,每一个Verma Module都会对应一个场(不一定是初级场),可以证明这些场一定是能动张量或者其偏导或者它们的NOP:
\begin{equation}
	\boxed{
		F\in\left\{T,\partial T,\ldots,N(\ldots)\right\}
	}
\end{equation}
\subsection{Descendant states}
前面是考虑一个真空态开始往上构造其它态,由于:
\begin{equation}
	L_n\left|\phi\right\rangle=\left[L_n,\phi_{-h}\right]\left|0\right\rangle=\left(h\left(n+1\right)-n\right)\phi_{-h+n}\left|0\right\rangle=0,\quad n>0
\end{equation}
所以初级场作用在真空态上构造的$\ket{\phi}$也比较特殊,和真空态一样被所有的$L_n$湮灭算符湮灭,所以也可以考虑从$\ket{\phi}$开始构造其它态:
\begin{definition}[Descendant States]
	\begin{equation}
		\boxed{
			\left\{L_{k_1}\ldots L_{k_n}|\phi\rangle|k_i\leq-1\right\}
		}
	\end{equation}
	其中$\phi$是任意一个初级场。这些态成为初级场的次级态(descendant states)。
\end{definition}
\begin{table}[H]
\centering
\begin{tabular}{lrc}
	\text { Field } & \text { State } & \text { Level } \\
	\hline \hline$ \phi(z)$& $\phi_{-h}|0\rangle=|h\rangle$ & $0 $\\
	\hline $\partial \phi $& $L_{-1} \phi_{-h}|0\rangle $& $1$ \\
	\hline$ \partial^{2} \phi $& $L_{-1} L_{-1} \phi_{-h}|0\rangle $& $2 $\\
	$N(T \phi)$ &$ L_{-2} \phi_{-h}|0\rangle$ & 2 \\
	\hline $\partial^{3} \phi$ & $L_{-1} L_{-1} L_{-1} \phi_{-h}|0\rangle$ & $3$ \\
	$N(T \partial \phi)$ &$ L_{-2} L_{-1} \phi_{-h}|0\rangle $&$ 3$ \\
	$N(\partial T \phi) $& $L_{-3} \phi_{-h}|0\rangle $& $3$ \\
	\hline \ldots & \ldots & \ldots
\end{tabular}
\end{table}
同样根据态算符对应,这些态也会对应到一些场,根据上面的表就可以猜到是:
\begin{equation}
	\boxed{
		\left[\phi(z)\right]:=\left\{\phi,\partial\phi,\partial^2\phi,\ldots,N(T\phi),\ldots\right\}
	}
\end{equation}
我们成为$\phi$的Conformal Family。Verma Module是直接真空态生成来的,真空态对应$h=0$,可以用$[1]$来标记这个特殊的Conformal Family。

注意到在上面的表中我们写下来对应的Level,注意到$L_0$可看作体系的哈密顿量,真空态对应能量0,初级场$\ket{h}$对应能量$h$,而作用$L_{-k}$上去会把能量加$k$,所以这就跟量子力学中谐振子问题作用产生算符得到更高能级的态一样。只不过这里能级都是简并的,而简并态数目的计数在后面的环面CFT配分函数构造上有很大的作用,对于Verma module(真空态毕竟唯一),或者$\ket{h}$本身不简并的Conformal Family来说,设$P(N)$是第$N$个Level简并度,则其有下面的生成函数:
\begin{equation}
	\boxed{
		\prod_{n=1}^\infty\frac1{1-q^n}=\sum_{N=0}^\infty P(N)q^N
	}
\end{equation}

\section{Representations of the Virasoro Algebra}
现在来从群表示的观点看下我们前面的Verma Module和Conformal Family实在做什么。前面反复强调如果我们有了一个理论的对称代数,那么确定其希尔伯特空间就是在找群表示。最简单的CFT就是只有Virasoro代数作为其对称代数,由于这时左右模完全分离,所以我们可以先分开讨论,然后$\bigoplus_{\{h,\bar h\}} V_{c,h}\otimes V_{c,\bar h}$就构成了整个表示空间\sn{也正因为完全分离,这里$h,\bar h$直积时无特殊要求,后面玻色子我们会看到一个有要求的最简单的例子。}。

我们要找的表示是所谓最高权表示,从量子力学的角度去想,构造谱的第一步是去找CSCO,由于理论没有其他对称性,也就是说没有其他力学量和$L_0$对易,所以我们期望构造的谱是$L_0$的本征值。进一步,由于$\{L_n,n\geq0\}$构成了理论的湮灭算符,所以我们先找到被所有湮灭算符湮灭的那些态:
\begin{equation}
	\begin{aligned}L_n\left|h\right\rangle&=0\quad\text{for}\quad n>0,\\L_0\left|h\right\rangle&=h\left|h\right\rangle\end{aligned}
\end{equation}
找到了所有的最高权之后,下一步就是将产生算符$\{L_{n},n<0\}$作用到上面构造能量更高的态:\sn{这些态都是$L_0$的本征态,但是我们并不用$\ket{h+k}$这样的记号标记他们,因为要和最高权区分。}
\begin{equation}
	L_{-1}\ket{h},\quad L_{-2}\ket{h},\quad L_{-1}L_{-1}\ket{h},\quad L_{-3}\ket{h},\quad\ldots
\end{equation}
不难发现,最高权就是真空态和理论中的初级态,而其他用产生算符构造的态就是Verma module和descendant states。

但是这只是从数学上构造出了这些态,真正物理的态最大的特点就是可归一化,所以那些模方为0的态要被剔除出去,比如说前面在构造Verma Module时,就没有把$L_{-1}\ket{0}$算进去,后面我们会进一步阐明如何找到其它的模方为0的态。进一步如果要求理论是幺正的,那些模方为负数的表示还要整个剔除掉。\sn{注意,幺正性要求那些有模方为负数的次级态的表示是被禁闭的,而不是像模方为0的态稍微剔除一下就好。}

如果理论还存在更大的对称性,这个时候CSCO就会加入其他的和${L_0}$对易的算符,谱将会同时对角化CSCO中的算符,即最高权现在由更多量子数标记$\ket{h,q}$。而且产生湮灭算符也不只有$\{L_{n},n<0\}$了,构造次级态还需要用理论中其他的产生湮灭算符去作用。这在数学上本身就是个非常麻烦的事情,后面我们并不追求数学上的一般性和严谨性,只是给些例子。

\section{Free CFT}
学习CFT最好的方式是给出一些具体的例子,首先我们考虑最简单的自由CFT。
\subsection{Free Bosons}
自由玻色场可以从圆柱上定义的标量场量子化后得来,这里直接给出二维CFT的版本\sn{已经做了Wick转动,$\kappa$是弦论来的耦合常数}:
\begin{equation}
	\begin{aligned}
		\text{S}& =\frac1{4\pi\kappa}\int dzd\overline{z}\sqrt{|g|}g^{ab}\partial_aX\partial_bX  \\
		&\xleftarrow{\kappa=1}\frac1{4\pi}\int dzd\overline{z}\partial X\cdot\overline{\partial}X
	\end{aligned}
\end{equation}
其中度规是从圆柱上的平直度规转到复平面上的度规:
\begin{equation}
	\left.g_{ab}=\left[\begin{array}{cc}0&\frac{1}{2z\overline{z}}\\\frac{1}{2z\overline{z}}&0\end{array}\right.\right],\quad g^{ab}=\left[\begin{array}{cc}0&2z\overline{z}\\2z\overline{z}&0\end{array}\right]
\end{equation}
再次强调CFT的定义根本不需要拉氏量,这里只是更场论一点的出发方式。直接读出运动方程:\sn{这也说明$X(z,\bar z)$总是可以写成全纯和反全纯部分之和}
\begin{equation}
	\partial\overline{\partial}X(z,\overline{z})=0
\end{equation}
可以定义下面两个手征场:
\begin{equation}\label{35.4}
	j(z)=i\partial X(z,\overline z),\quad\bar j(\overline z)=i\overline\partial X(z,\overline z)
\end{equation}
还可以读出传播子$K\equiv\left\langle{X(z,\bar z)X(w,\bar w)}\right\rangle$:
\begin{equation}
	\partial_z\partial_{\overline{z}}K(z,\overline{z},w,\overline{w})=-2\pi\delta^{(2)}(z,-w)\Rightarrow K=-\log\left|z-w\right|^2
\end{equation}
这直接说明了$X$不是初级场,但是:
\begin{equation}
	\left\langle j(z)j(w)\right\rangle=\frac{1}{(z-w)^2}
\end{equation}
说明了$j(z)$是$h=1$的初级场,也得到了OPE:\sn{由于OPE的最奇异项只能是二次的,而一次的又破坏了$z\leftrightarrow w$的对称性。}
\begin{equation}\label{35.7}
	\boxed{
		j(z)j(w)=\partial_z X(z)\partial_w X(w)=\frac{1}{(z-w)^2}+\ldots
	}
\end{equation}
根据OPE很容易算出对易关系:
\begin{equation}
	\boxed{
		\begin{bmatrix}j_m,j_n\end{bmatrix}= m\delta_{m+n,0}
	}
\end{equation}
根据$j$是共形权为1的初级场很容易证明作用量的确是共形不变的\sn{不少文献数量纲之后根据尺度不变性直接说明共形不变,绝大多数情况下这没问题,甚至一度让人们觉得尺度不变蕴含共形不变。但是上世纪八十年代Polchinski就研究了反例\cite{Polchinski:1987dy}}。下面演示如何通过作用量确定CFT中最关键的能动张量:
\begin{equation}
	T_{ab}=-4\pi~\gamma\frac1{\sqrt{|g|}}\frac{\delta\mathcal{S}}{\delta g^{ab}}
\end{equation}
这里$\gamma$是待定的归一化系数,利用上式计算得到:\sn{$\delta\sqrt{|g|}=-\frac12\sqrt{|g|}g_{ab}\delta g^{ab}$}
\begin{equation}
	T_{zz}=-\gamma\mathrm{~}\partial X\partial X\mathrm{~},\quad\quad T_{z\overline{z}}=T_{\overline{z}z}=0
\end{equation}
下面最重要的一步就是取正规乘积,这相当于把理论的零点能shift掉:
\begin{equation}
	T(z)=-\gamma N(\partial X\partial X)(z)=\gamma N(jj)(z)
\end{equation}
定$\gamma$的方式有很多,重点就是其洛朗模$L_n$要满足Virasoro代数,$L_n$和初级场的洛朗模之间对易关系要有正确的形式\ref{31.27},或者是能动张量和初级场的OPE有\ref{31.14}的形式,我们主要来看这种方法:\sn{\ref{35.7}得知$\wick{\c j(z)\c j(w)}=\frac{1}{(z-w)^2}$}
\begin{equation}
	\begin{aligned}
		\wick{\c T(z)\c j(w)}&=\gamma \wick{\c N(jj)(z)\c j(w)}=\gamma \wick{\c j(w)\c N(jj)(z)}\\
		&=\frac{\gamma}{2\pi i}\oint_{\mathcal{C}(z)}\frac{dx}{x-z}\left\{\wick{\c j(w)\c j(x)}j(z)+j(x)\wick{\c j(w)\c j(z)}\right\}\\
		&=\frac{2\gamma}{(z-w)^2}j(z)\\
		&=\frac{2\gamma}{(z-w)^2}j(w)+\frac{2\gamma}{z-w}\partial j(w)
	\end{aligned}
\end{equation}
最后一步使用泰勒展开并丢掉Regular部分。和\ref{31.14}对比便知道$\gamma=\frac{1}{2}$:
\begin{equation}
	\boxed{
		T(z)=\frac{1}{2}N(jj)(z)
	}
\end{equation}
\begin{equation}\label{35.14}
	L_n=\frac{1}{2}\left.N(jj)_n=\frac{1}{2}\sum_{k>-1}j_{n-k}\left.j_k+\frac{1}{2}\right.\sum_{k\leq-1}j_k\left.j_{n-k}\right.\right. 
\end{equation}
下面再来阐明如何计算中心荷,比较简单的方式是利用$\langle0|L_{+2}L_{-2}|0\rangle=\langle0|[L_2,L_{-2}]|0\rangle=\frac c2$,为了继续熟悉Wick定理,我们根据$TT$ OPE来计算:
\begin{equation}
	\begin{aligned}
		T(z)T(w)&=\frac{1}{4}R\left[:jj:(z):jj:(w)\right]+\cdots\\
		&=\frac{1}{4}\wick{\c j(z)\c j(w)}:j(z)j(w):\times 2+\frac{1}{4}\wick{\c j(z)\c j(w)}\cdot\wick{\c j(z)\c j(w)}\times 2+\cdots\\
		&=\frac{1}{(z-w)^2}:j(z)j(w):+\frac{\frac{1}{2}}{(z-w)^4}+\cdots\\
		&=\frac{1}{(z-w)^2}:jj:(w)+\frac{1}{z-w}\partial_w :jj:(w)\times\frac{1}{2}+\frac{\frac{1}{2}}{(z-w)^4}+\cdots\\
		&=\frac{\frac{1}{2}}{(z-w)^4}+\frac{2T(w)}{(z-w)^2}+\frac{\partial_w T(w)}{z-w}+\cdots
	\end{aligned}
\end{equation}
这里利用了自由场的Wick定理\ref{33.6}进行计算,倒数第二个等号利用了$j(z)$泰勒展开。与\ref{TT}对比得到:
\begin{theorem}
	自由玻色场CFT的中心荷$\boxed{c=1}$
\end{theorem}
如果理论是$N$个无相互作用的Boson场,那么中心荷是$c=N$。所以中心荷某种程度上可以理解为理论的自由度个数。
\subsubsection{Vertex Operator}
虽然从CFT的角度来看我们更关心$j$,但是从弦的角度来看$X$是更基本的,利用\ref{35.7}洛朗展开后积分得到:
\begin{equation}\label{35.16}
	\left.X(z,\overline{z})=x_0-i\left(\begin{array}{c}j_0\ln z+\overline{j}_0\ln\overline{z}\\\end{array}\right.\right)+i\sum_{n\neq0}\frac1n\left(\begin{array}{c}j_nz^{-n}+\overline{j}_n\overline{z}^{-n}\\\end{array}\right)
\end{equation}
根据$X(e^{2\pi i}z,e^{-2\pi i}\bar z)=X(z,\bar z)$有:
\begin{equation}\label{35.17}
	\boxed{
		j_0=\bar j_0
	}
\end{equation}
这个等式是后面构造正确的希尔伯特空间的关键。从自由场模式展开的观点来看,这些$z^n$模对到圆柱那边就是傅里叶展开,所以$j_n$就是理论的产生湮灭算符。而$j_0$实际上对应质心的正则动量$\pi_0$:
\begin{equation}
	\pi_0=\frac1{4\pi}\int_0^{2\pi}dx^1\frac{\partial X{\left(x^0,x^1\right)}}{\partial{\left(-ix^0\right)}}=\frac{j_0+\overline{j}_0}2=j_0
\end{equation}
$x_0$对应弦的质心。按照正则量子化应当有$[x_0,\pi_0]=i$。这样来看\ref{35.14}就分成了质心的动能和弦上激发能,相对质心能量两部分。
\begin{equation}
	L_0=\frac12~j_0~j_0+\frac12~\sum_{k\geq1}j_{-k}~j_k+\frac12~\sum_{k\leq-1}j_k~j_{-k}
\end{equation}
\begin{definition}[Vertex Operator]
	虽然$X$本身不是初级场,但是总可以定义下面的顶点算符:
	\begin{equation}
		\boxed{
			V(z)=:e^{i\alpha X(z)}:
		}
	\end{equation}
	他是共形权为$\left(\frac{\alpha^2}{2},0\right)$的初级场,反全纯部分也可以类似定义。
\end{definition}
\begin{proof}
	利用Wick定理得到下面的OPE:
	\begin{equation}
		\partial\varphi(z):\varphi^n:(w)=\frac{-n}{z-w}:\varphi^{n-1}:(w)
	\end{equation}
	将顶点算符泰勒展开得到:
	\begin{equation}\label{35.21}
		\begin{aligned}
			\wick{\c j(z)\c V_{\alpha}(w)}& \begin{aligned}=\sum_{n=0}^\infty\frac{-in}{z-w}\frac{(i\alpha)^n}{n!}:\varphi^{n-1}:(w)\end{aligned}  \\
			&=\frac{\alpha}{z-w} V_\alpha(w)
		\end{aligned}
	\end{equation}
	再计算与$T$的OPE:\footnote{计算中并没有直接去用\ref{35.21},想想这是为什么?可以参考\url{https://physics.stackexchange.com/questions/398365/ope-double-contractions-between-t-and-eikx}}
	\begin{equation}
		\begin{aligned}
			\wick{\c T(z)\c V_{\alpha}(w)}
			=&\frac{1}{2}\sum_{n=0}^\infty\frac{(i\alpha)^n}{n!}:\partial\varphi(z)\partial\varphi(z)::\varphi(w,\bar{w})^n: \\
			=&\frac1{2}\frac1{(z-w)^2}\sum_{n=2}^\infty\frac{(i\alpha)^n}{n!}\cdot n(n-1):\varphi(w,\bar{w})^{n-2}: \\
			&+\frac1{z-w}\sum_{n=1}^\infty\frac{(i\alpha)^n}{n!}\cdot n:\partial\varphi(z)\varphi(w,\bar{w})^{n-1}: \\
			=&\frac{\alpha^2}{2}\frac{V_\alpha(w,\bar{w})}{(z-w)^2}+\frac{\partial_wV_\alpha(w,\bar{w})}{z-w}
		\end{aligned}
	\end{equation}
	最后一个等号利用了$\partial\phi(z)$在$w$处的泰勒展开。
\end{proof}
根据\ref{35.21}可以得到下面的对易关系:
\begin{equation}
	[j_0,V_\alpha]=\alpha V_\alpha\Rightarrow j_0\ket{\alpha}=\alpha \ket{\alpha}
\end{equation}
所以$\alpha$的物理含义是质心动量!手征的部分终究不是物理的,我们需要把全纯反全纯的顶点算符按一定的要求组合起来得到体系真正的初级场。这个要求就是\ref{35.17},左模动量要等于右模动量,所以体系中的初级场应当是:
\begin{equation}
	V_\alpha(z,\bar z)=V_{\alpha}(z)\bar V_{\alpha}(\bar z)=:e^{i\alpha X(z,\bar z)}:
\end{equation}
对应共形权为$\left(\frac{\alpha^2}{2},\frac{\alpha^2}{2}\right)$。利用\ref{33.9}得到:
\begin{equation}
	V_\alpha(z,\bar z)V_\beta(w,\bar w)=|z-w|^{2\alpha\beta}V_{\alpha+\beta}(w,\bar w)
\end{equation}
显然$\left\langle V_{\alpha}(z,\overline{z})V_{\beta}(w,\overline{w})\right\rangle \sim\delta(|\alpha|-|\beta|)$,但是$\alpha=\beta\Rightarrow\alpha\beta>0$,根据上面的OPE这将导致体系呈现出长程关联,超距作用。所以唯一的可能就是$\alpha+\beta=0$,所以顶点算符的两点关联函数\textbf{唯一不为零的}只能是:
\begin{equation}
	\left\langle V_{-\alpha}(z,\overline{z})V_{\alpha}(w,\overline{w})\right\rangle=\frac1{(z-w)^{\alpha^2}(\overline{z}-\overline{w})^{\alpha^2}}
\end{equation}
而$\alpha$的含义是质心动量,所以这其实是动量守恒的体现。实际上自由玻色子对称性比共形对称性更大,从作用量可以看出$X(z,\overline{z})\mapsto X(z,\overline{z})+a$并不会对作用量产生任何影响,根据Noether定理这实际上对应的是\ref{35.4}这两个守恒流,自由玻色子真正的对称性是$\hat{\mathfrak{u}}(1)_1$的流对称性。守恒流对应的守恒荷正是$Q=\oint\frac{dz}{2\pi i}j(z)=j_{0}$,所以有动量守恒。这一点可以推广到顶点算符的n点关联函数,其不为0当且仅当$\sum_i \alpha_i=0$。

现在我们构造出了理论里所有的谱$\{j,V_\alpha\}$,构造希尔伯特空间的产生湮灭算符这里用的是流所对应的$j_n$而不是$L_n$\sn{当然这里$L_n$本身就是$j_n$构造来的}:
\begin{theorem}
	自由玻色体系的希尔伯特空间由下面的态张成:
	\begin{equation}
		j_{-1}^{n_1}j_{-2}^{n_2}\cdots\bar{j}_{-1}^{m_1}\bar{j}_{-2}^{m_2}\cdots|\alpha\rangle\quad(n_i,m_j\geq0)
	\end{equation}
\end{theorem}
这其实蕴含了相当大的物理意义,我们并不从群表示的观点来看,而是从物理本身上来看。这里$\alpha$的不同标记了不同质心动量的真空,它是在$\alpha=0$的“绝对”真空上作用$V_\alpha(z,\bar z)$得到的。我们也提到了$j_n$就是$X$模式展开之后的产生湮灭算符,所以用它在这些真空上面作用得到希尔伯特空间也是非常合理的。
\subsubsection{Compactified Bosons}
既然玻色体系是允许$X(z,\overline{z})\mapsto X(z,\overline{z})+a$的shift的,那我们何不干脆把$X$的位形空间粘合起来,也就是认为:
\begin{equation}
	X\sim X+2\pi nR,\quad n\in\mathbb{Z}
\end{equation}
把$X$看作是像角度变量一样的场,这样的紧致化显然是可行的,$n$称为\textbf{winding number}。但要注意,我们这里是把位形空间从$\mathbb{R}$紧致化成$\mathbb{R}\cup\infty\cong\mathcal{S}^1_R$,并没有对场本身所定义在的时空流形做任何的拓扑变形,这一点要与后面环面CFT区分。

这样一来,场的单值性被放宽为$X(e^{2\pi i}z,e^{-2\pi i}\bar z)=X(z,\bar z)+2\pi n R$,从而\ref{35.17}变为:
\begin{equation}\label{35.29}
	\boxed{
		j_0-\bar j_0=nR,\quad n\in \mathbb{Z}
	}
\end{equation}
顶点算符(手征)也只允许那些$\alpha=\frac{m}{R},m\in\mathbb{Z}$的谱。这种紧致化对场论的希尔伯特空间会有比较大的变化,由于\ref{35.17}改变了,所以左右模的组合方式也要随之改变,而且体系的谱也从连续谱变成了离散谱。

$\{j_0,\bar j_0,L_0\}$构成体系的CSCO,初级态应该同时对角化它们,除了\ref{35.29},还有质心动量给的约束:
\begin{equation}
	\boxed{
		\pi_0=\frac{j_0+\bar j_0}{2}=\alpha=\frac{m}{R},\quad m\in\mathbb{Z}
	}
\end{equation}
初级场原先由$\ket{\alpha}$标记,现在由两个整数标记为$\ket{m,n}$:
\begin{equation}
	j_0\left|m,n\right\rangle=\left(\frac mR+\frac{Rn}2\right)\left|m,n\right\rangle,\quad\quad\overline{j}_0\left|m,n\right\rangle=\left(\frac mR-\frac{Rn}2\right)\left|m,n\right\rangle 
\end{equation}
$m$和质心动量有关,$n$和卷绕数有关,$m\neq 0$的态称为Kaluza\mbox{-}Klein态。这个最高权的共形权显然为:
\begin{equation}
	h=\frac{1}{2}\left(\frac mR+\frac{Rn}2\right)^2, \quad \bar h=\frac{1}{2}\left(\frac mR-\frac{Rn}2\right)^2
\end{equation}
希尔伯特空间的剩下部分就是把$j_n,\bar j_n$作用到$\ket{m,n}$上面去了。
\begin{remark}
	前面推导有关顶点算符的对易关系都是使用的wick定理,这样推起来虽然方便,但是物理上看比较好的方式是直接代入$X$的显式表达式\ref{35.16},再根据正则对易关系$[x_0,\pi_0]=i$得到$[\pi_0,V_\alpha]=\alpha V_\alpha$,前面只是在$j_0=\pi_0$的特例下去推导\footnote{这也解释了$j_0$作用到$\ket{m,n}$真空上得到的并非$\alpha$}。所谓不同的真空,应当就是$\pi_0$的不同模,同样也是$j_0,\bar j_0$的本征态。然后在上面作用$j_n$就得到了次级态。这样来看前面的一系列讨论物理图像就很清晰。
\end{remark}
\subsubsection{Current Algebra Realization}
现在回到非紧致化的Bosons,除了$j$这一个流,顶点算符还确定了两个流:
\begin{equation}
	j^\pm(z)=:e^{\pm i\sqrt{2}X}:
\end{equation}
利用前面得到的$jV$以及$VV$ OPE我们可以得到下面的对易关系:\sn{Blumenhagen\cite{Blumenhagen:2009zz}从自举的角度,也就是从\ref{current}考虑了这个问题}
\begin{equation}
	\boxed{
		\begin{aligned}
			&\left[j_m,j_n\right]=m\delta_{m+n,0},&&\left[j_m^\pm,j_n^\pm\right]=0,\\
			&\left[j_m,j_n^\pm\right]=\pm\sqrt{2}j_{m+n}^\pm,&&\left[j_m^+,j_n^-\right]=\sqrt{2}j_{m+n}+m\delta_{m+n,0}
		\end{aligned}
	}
\end{equation}
重定义:
\begin{equation}
	j^1=\frac1{\sqrt{2}}\left(j^++j^-\right),\quad\quad j^2=\frac1{\sqrt{2}i}\left(j^+-j^-\right),\quad\quad j^3=j
\end{equation}
得到:
\begin{equation}
	\boxed{
		\left[j_m^i,j_n^j\right]=+i\mathrm{~}\sqrt{2}\sum_k\epsilon^{ijk}\mathrm{~}j_{m+n}^k+m\mathrm{~}\delta^{ij}\mathrm{~}\delta_{m,-n}
	}
\end{equation}
这具有流代数的形式,这实际上是个$\mathfrak{su}(2)$的流代数。
\begin{remark}
	注意,我们只是说明了在自由玻色子体系中具有流代数代数结构,并不是说自由玻色子有$\mathfrak{su}(2)$的对称性!它的对称性永远是$\mathfrak{u}(1)$,问题的关键就在于在$N(jj)$的能动张量所描述的玻色体系下,这三个流真正守恒的也就$i\partial X$一个,后面会发现如果要让其对称性提升为$\mathfrak{su}(2)$,必须重新构造能动张量,这相当于在作用量中加入其他的项,构成所谓WZW模型。紧致化为半径为$\frac{1}{\sqrt{2}}$圆的玻色体系也会提升为$\mathfrak{su}(2)$对称性。
\end{remark}
\subsection{Free Fermions}
这里我们考虑Majorana型费米子,也就是下面作用量中:
\begin{equation}
	\mathcal{S}=\frac1{4\pi\kappa}\int dzd\overline{z}\left(\psi\overline{\partial}\psi+\overline{\psi}\partial\overline{\psi}\right)
\end{equation}
$\psi$这个给wyle旋量是一个实值标量场,而且是个Grassmann数。运动方程为:
\begin{equation}
	\partial\overline{\psi}=\overline{\partial}\psi=0
\end{equation}
传播子$K\equiv\left\langle\psi(z)\psi(w)\right\rangle$为:
\begin{equation}
	\partial_{\overline{z}}K(z,w)=2\pi\kappa\delta^{(2)}(z-w)\Rightarrow K(z,w)=\frac{1}{z-w}
\end{equation}
这表明$\psi(z)$是共形权为$(\frac{1}{2},0)$的初级场,利用这一点也可证明作用量确实共形不变。上式也是OPE的正则部分,注意到上式并非关于$z,w$对称的,因为费米子的反对易性,所以对径向序的定义为:
\begin{equation}
	R\big(\Psi(z)\Theta(w)\big):=
	\begin{cases}+\Psi(z)\Theta(w)&\mathrm{for}\quad|z|>|w|\\-\Theta(w)\Psi(z)&\mathrm{for}\quad|w|>|z|\end{cases}
\end{equation}
这直接导致费米子的OPE是反交换律的,这也解释了为什么OPE中没有$(z-w)^{-\frac{1}{2}}$的项。费米体系我们比较关注下面的两类边界条件:
\begin{equation}
	\begin{aligned}\psi(e^{2\pi i}z)&=+\psi(z)&&\text{Neveu-Schwarz sector (NS)}\\\psi(e^{2\pi i}z)&=-\psi(z)&&\text{Ramond sector (R)}\end{aligned}
\end{equation}
这就像是对多值函数$\sqrt{z}$考虑其哪一支。两类边界条件也导致了不同的洛朗展开:
\begin{equation}
	\psi(z)=\sum_r\psi_r\mathrm{~}z^{-r-\frac12}=\begin{cases}r\in\mathbb{Z}+\frac1{2}&NS\\r\in\mathbb{Z}&R\end{cases}
\end{equation}
反对易关系同样包含于OPE:
\begin{equation}
	\begin{aligned}
		\left\{\psi_{r},\psi_{s}\right\}=&\oint\frac{dz}{2\pi i}\oint\frac{dw}{2\pi i}\left\{\psi(z),\psi(w)\right\}z^{r-\frac{1}{2}}w^{s-\frac{1}{2}} \\
		=&\oint\frac{dw}{2\pi i}w^{s-\frac12}\Big(\oint_{|z|>|w|}\frac{dz}{2\pi i}\psi(z)\psi(w)z^{r-\frac12} \\
		&-\oint_{|z|<|w|}\frac{dz}{2\pi i}\left.-\psi(w)\psi(z)z^{r-\frac12}\right) \\
		=&\oint\frac{dw}{2\pi i}w^{s-\frac12}\oint_{\mathcal{C}(w)}\frac{dz}{2\pi i}\underbrace{R\big(\psi(z)\psi(w)\big)}_{\frac{\kappa}{z-w}}z^{r-\frac12} \\
		=&\kappa\oint\frac{aw}{2\pi i}w^{r+s-1} \\
		=&\kappa~\delta_{r+s,0}~
	\end{aligned}
\end{equation}
费米子体系我们用下面的式子来确定能动张量:
\begin{equation}
	T_{\mu\nu}^c=8\pi\kappa\left.\gamma\left(-\eta_{\mu\nu}\mathcal{L}+\sum_i\frac{\partial\mathcal{L}}{\partial\left(\partial^\mu\phi_i\right)}\partial_\nu\phi_i\right)\right. 
\end{equation}
\begin{equation}
	T_{zz}=\gamma~\psi~\partial\psi~,\quad T_{z\overline{z}}=-\gamma~\overline{\psi}~\partial\overline{\psi}~,\quad\quad T_{\overline{z}z}=-\gamma~\psi~\overline{\partial}\psi~=0,\quad T_{\overline{z}z}=\gamma~\overline{\psi}~\overline{\partial}\overline{\psi}=0
\end{equation}
然后取NOP去除零点能:
\begin{equation}
	T(z)=\gamma\mathrm{~}N\left(\mathrm{~}\psi\mathrm{~}\partial\psi\mathrm{~}\right)
\end{equation}
费米子情况,由于反对易性,NOP洛朗模要修正为:
\begin{equation}
	\boxed{
		N\left(\psi_0\theta\right)_r=-\sum_{s>-h^\theta}\psi_{r-s}\theta_s+\sum_{s\leq-h^\phi}\theta_s\psi_{r-s}
	}
\end{equation}
\subsubsection{Bosonization}
\subsection{$b,c$ ghost}


\section{Unitary Representations of the Virasoro Algebra}
\section{Fusion Rules}
\section{Ka\v{c}\mbox{–}Moody Symmetry}
\subsection{Ka\v{c}\mbox{–}Moody Algebras}
\subsection{Sugawara Construction}
\subsection{Knizhnik\mbox{–}Zamolodchikov Equation}
\section{Example: Highest Weight Representations of $\widehat{\mathfrak{su}}(2)_{k}$ }
\section{Coset Construction}
\section{$\mathcal{W}$ Algebras}
本节主要参考\cite{Pope:1991ig}。
