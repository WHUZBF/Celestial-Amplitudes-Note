\part{Lorentz group \& special linear groups}
% 计数器清零,每个part都要引用,除了part1
\setcounter{theorem}{0}
\setcounter{definition}{0}
\setcounter{lemma}{0}
\setcounter{sidenote}{1}

本部分我们的目的是建立一系列同构关系,基本思路就是先找到一个同态,然后利用同态核定理构造同构
\begin{theorem}[同态核定理]
	如果$f:G\to H$是一个群同态,那么有$\ker(f)\lhd G$且$G/\ker(f)\cong\Im(f)$
\end{theorem}
而且本部分会充分利用李群是微分流形这一拓扑性质进行说明,很多证明没有数学上的严谨,重在直观的感性认知。
\section{$SO(3)\cong SU(2)/\mathbb{Z}_2$}
$SU(2)$是所有行列式为1的酉矩阵构成的群,其群元素可以一般的写为:
\[
U=
\begin{pmatrix}
	\alpha & \beta\\
	-\beta^* &\alpha^*
\end{pmatrix},\quad |\alpha|^2+|\beta|^2=1
\]
这意味着描述一个群元需要四个实参,且满足$\alpha_1^2+\alpha_2^2+\beta_1^2+\beta_2^2=1$,显然这意味着$SU(2)$的拓扑结构为$\mathcal{S}^3$。另外一个需要用到的概念是\textbf{群中心},也就是与所有群元都对易的群元\sn{注意与Casimir算符的区别}。不难看出$SU(2)$的群中心构成子群$\mathbb{Z}_2$:
\[
\mathbb{I}_2=\begin{pmatrix}
	1 & 0\\
	0 &1
\end{pmatrix},\quad
-\mathbb{I}_2=\begin{pmatrix}
	-1 &0\\
	0 &\-1
\end{pmatrix}
\]
\begin{theorem}
$
		SO(3)\cong SU(2)/\mathbb{Z}_2
$
\end{theorem}

\begin{proof}
	首先考虑$2\times2$无迹厄米矩阵构成的线性空间$\mathbb{V}$,显然其中任意一个元素都可以写为$X=x^i\sigma_i$,其中$\sigma_i$是三个Pauli矩阵,这实际上建立起了同构$\mathbb{V}\cong\mathbb{R}^3$。$\mathfrak{su}(2)$李代数作为线性空间显然是与$\mathbb{V}$同构的,而$SU(2)$在李代数\footnote{所谓代数,就是线性空间赋予一个封闭的乘法结构}上诱导出一个所谓伴随表示:
	\[
	\mathcal{U}(U)X=UXU^\dagger=Ux^i\sigma_i U^\dagger\equiv f(U)^i_j x^j\sigma_i
	\]
	容易验证这个表示是保范数$\left\|x\right\|$的,那么$f(U)\in O(3)$,也就是说我们建立了一个同态:
	\[f:SU(2)\to O(3),U\mapsto f(U)\]
	
	
	\setlength\parindent{2em}为了利用同态核定理,首先计算$\Im(f)$。由于$f$是个连续映射,而且$SU(2)$单连通\footnote{$\mathcal{S}^n$的基本群在$n=1$时为自由群$\mathbb{Z}$,其它时候都为平凡群},所以$f(U)$也应当包含在$O(3)$的单连通子群中,即$\Im(f)\subseteq SO(3)$。反过来$SO(3)\subseteq \Im(f)$也成立,可以看作是Euler角和Caylay\mbox{-}Klein参数之间的对应,所以$\Im(f)=SO(3)$
	
	
	\setlength\parindent{2em}现在来计算$\ker(f)$,$f(U)=\mathbb{I}_{3\times 3}$说明$\forall X\in \mathbb{V}$,都有$UXU^\dagger=X$,也就是说要找的$U$与任意$X$对易,那么其也与任意的$e^{i X}$对易,然而$\text{Lie Group}=\mathrm{e}^{\textbf{Lie Algebra}}$,所以$U$就是群中心的元素,所以$\ker(f)\cong\mathbb{Z}_2$,根据同态核定理便有$SO(3)\cong SU(2)/\mathbb{Z}_2$。
\end{proof}
\begin{remark}
	这其实说明了$SU(2)$是$SO(3)$群的双覆盖,$SO(3)$群对应流形是对径认同实心球$\mathbb{R}\mathbf{P}^2\times [0,\pi]$,从流形上也能感受一下。最后我们显式给出这个同态:
	\[
	f\left[\begin{pmatrix}
		a& b\\
		c &f
	\end{pmatrix}\right]=\begin{pmatrix}
	\Re(\bar a d+\bar b c) & \Im(a\bar  d-b\bar c)  &\Re(\bar a c-\bar bd) \\
	\Im(\bar a d+\bar b c) & \Re(a\bar  d-b\bar c)   & \Im(\bar a c-\bar bd)\\
	\Re(a\bar b-c\bar d) & \Im(a\bar b-c\bar d) &\frac{1}{2}\left(|a|^2-|b|^2-|c|^2+|d|^2\right)
	\end{pmatrix}
	\]
	上式直接从$U,-U$对应同一个$SO(3)$中元素也可看出双覆盖性。
\end{remark}

\section{$SO(2,1)^\uparrow\cong SL(2,\mathbb{R})/\mathbb{Z}_2$}
\begin{lemma}[QR分解]
	任意复矩阵都可以分解为一个酉矩阵$Q$和一个上三角矩阵$R$的乘积,且$R$主对角元全为正数。如果这个矩阵是实矩阵,那么$Q$为正交矩阵。如矩阵可逆,则分解唯一。
\end{lemma}
\begin{lemma}
	$SL(2,\mathbb{R})$的拓扑结构为$\mathcal{S}^1\times \mathbb{R}\times\mathbb{R}^+$,基本群为$\mathbb{Z}$
\end{lemma}
\begin{proof}
	根据QR分解,以及$\det S=1\neq 0$,任意$SL(2,\mathbb{R})$中的矩阵都可以唯一的分解为$S=QR$,而且要求$|Q|\cdot|R|=1$,而$R$主对角元全为正数以及$|Q|=\pm 1$实际上给出$Q\in SO(3)$且$R$的对角线上元素有$a\cdot b=1$且为正数的限制,而另一个元素不做限制。这其实就是在对$SL(2,\mathbb{R})$做直积分解,由于$SO(2)$对应的流形为$\mathcal{S}^1$,所以$SL(2,\mathbb{R})$对应的流形为$\mathcal{S}^1\times \mathbb{R}\times\mathbb{R}^+$,后两者基本群平凡,所以$SL(2,\mathbb{R})$对应的基本群为$\mathbb{Z}$。\footnote{这里用了乘积空间基本群为各自基本群的直积。}
\end{proof}
了解了$SL(2)$的拓扑性质后就可以开始证明本节的核心结论。
\begin{theorem}
	$SO(2,1)^\uparrow\cong SL(2,\mathbb{R})/\mathbb{Z}_2$
\end{theorem}
\begin{proof}
	与上一节同样,我们先构造$\mathfrak{sl}(2,\mathbb{C})$李代数,其由二维实无迹矩阵构成,生成元为:
	\[t_0=\begin{pmatrix}
		0 &1 \\
		-1&0
	\end{pmatrix},\quad t_1=\begin{pmatrix}
	0 &1 \\
	1&0
	\end{pmatrix},\quad t_2=\begin{pmatrix}
	1 &0 \\
	0&-1
	\end{pmatrix} \]
	张成的线性空间中任一元素可以表达为$X=x^\mu t_\mu$,显然$\det X=-\eta_{\mu\nu}x^\mu x^\nu=-x^2$,同样我们对$S\in SL(2,\mathbb{C})$构造伴随表示$X\mapsto SXS^{-1}$,其保事件间隔不变,所以诱导了一个同态:
	\[f:SL(2,\mathbb{R})\to O(2,1),S\mapsto f(S)\]
	其中
	\[S t_\mu x^\mu S^{-1}=t_\mu {f(S)^\mu}_\nu x^\nu,\forall x^\mu \in\mathbb{R}^3\iff S t_\mu  S^{-1}=t_\nu {f(S)^\nu}_\mu \]
	根据$f$连续,从拓扑上得知$\Im f\in SO(2,1)^\uparrow$,反过来,要论证任何$\Lambda \in SO(2,1)^\uparrow$都可以用$f(S)$表示,根据$\Lambda=R_1 L(\chi) R_2$,我们只需要找到$S_1,S_2,S(\chi)$使得
	\[f(S_1)=R_1,f(S_1)=R_2,f(S(\chi))=L(\chi)\]
	不难验证前两个等式只需要选取
	\[S=\begin{pmatrix}
		\cos\theta &\sin\theta \\
		-\sin\theta&\cos\theta
	\end{pmatrix}\in SO(2)\subseteq SL(2,\mathbb{R})\]
	并合适选取$\theta$参数即可,而后面一个只需选取:
	\[S(\chi)=\begin{pmatrix}
		e^{-\chi/2} &0 \\
		0&e^{\chi/2} 
	\end{pmatrix}\]
	最后证明$\ker f\cong \mathbb{Z}_2$的方法就和上一节一样了。
\end{proof}
\begin{remark}
	下面显式给出同态:
	\begin{equation}
		f\left[\begin{pmatrix}
			a &b \\
			c&d
		\end{pmatrix}\right]=\begin{pmatrix}
		\frac{1}{2}\left(a^2+b^2+c^2+d^2\right) &\frac{1}{2}\left(a^2-b^2+c^2-d^2\right) &-ab-cd\\
		\frac{1}{2}\left(a^2+b^2-c^2-d^2\right)&\frac{1}{2}\left(a^2-b^2-c^2-d^2\right)&-ab+cd\\
		-ac-bd&bd-ac&ad+bc
		\end{pmatrix}
	\end{equation}
\end{remark}
\section{$SO(3,1)^\uparrow\cong SL(2,\mathbb{C})/\mathbb{Z}_2$}
这一节的证明与上一节非常类似,证明细节会适当省略。还是首先关注一下$SL(2,\mathbb{C})$$SL(2,\mathbb{C})$的拓扑性质。
\begin{lemma}
	$SL(2,\mathbb{C})$单连通
\end{lemma}
\begin{proof}
	证明依旧是使用QR分解,现在$R$需要一个正实数和一个复数来描述,所以对应流形为$\mathbb{R}^+\times\mathbb{C}$,基本群平凡。而$Q\in SU(2)$对应流形为$\mathcal{S}^3$,基本群也平凡,所以$SL(2,\mathbb{C})$基本群平凡,即单连通。
\end{proof}
下面证明本节核心定理:
\begin{theorem}
	$SO(3,1)^\uparrow\cong SL(2,\mathbb{C})/\mathbb{Z}_2$
\end{theorem}
\begin{proof}
	证明完全仿造上一节,只是$t\to\tau$,$\tau_0=\mathbb{I}_{2\times 2}$,$\tau_1=-\sigma_1$\footnote{这里符号约定上比一般定义多了个负号,是为了后文处理天球符号更自洽。},$\tau_2=\sigma_2$,$\tau_3=\sigma_3$。后面的证明也是用伴随表示诱导同态后计算$\Im f$,这里根据$SL(2,\mathbb{C})\supseteq SU(2)/\mathbb{Z}_2\cong SO(3)$可以得到$f(S)=R$,剩下的一个只用取:
	\[S(\chi)=\begin{pmatrix}
		\cosh\frac{\chi}{2} &\sinh\frac{\chi}{2} \\
		\sinh\frac{\chi}{2} &\cosh\frac{\chi}{2} 
	\end{pmatrix}\]
	最后计算$\ker f$也是同样的思路说明$\ker f\cong\mathbb{Z}_2$
\end{proof}
\begin{remark}
	下面显式给出同态:
	\begin{equation}\label{eq:8.1}
		\begin{aligned}
		&f\left[\begin{pmatrix}
			a &b \\
			c&d
		\end{pmatrix}\right]=\\
		&\begin{pmatrix}
			\frac{1}{2}\left(|a|^2+|b|^2+|c|^2+|d|^2\right) &-\Re(a\bar b +c\bar d) &\Im(a\bar b +c\bar d)&\frac{1}{2}\left(|a|^2-|b|^2+|c|^2-|d|^2\right)\\
			-\Re(\bar ac +\bar bd)&\Re(\bar ad +\bar bc)&-\Im(a\bar d- b\bar c)&-\Re(\bar ac -\bar bd)\\
			\Im(\bar ac +\bar bd)&-\Im(\bar ad +\bar bc)&\Re(a\bar d- b\bar c)&\Im(\bar ac -\bar bd)\\
			\frac{1}{2}\left(|a|^2+|b|^2-|c|^2-|d|^2\right)&-\Re(a\bar b -c\bar d)&\Im(a\bar b -c\bar d)&\frac{1}{2}\left(|a|^2-|b|^2-|c|^2+|d|^2\right)
		\end{pmatrix}
		\end{aligned}
	\end{equation}
\end{remark}
下面给出两个例子:
\begin{example}
	$z$轴旋转
	\begin{equation}
		\begin{pmatrix}
			1 & 0 &0  &0 \\
			0&\cos\theta  &-\sin\theta  & 0\\
			0&-\sin\theta  & \cos\theta &0 \\
			0&  0&  0&1
		\end{pmatrix}\sim\pm\begin{pmatrix}
		e^{-i\theta/2}&0\\
		0&e^{i\theta/2}
		\end{pmatrix}
	\end{equation}
	$z$方向boost
	\begin{equation}
		\begin{pmatrix}
			\cosh\chi& 0 &0  &-\sinh\chi \\
			0&1  &0  & 0\\
			0&0 & 1 &0 \\
			-\sinh\chi&  0&  0&\cosh\chi
		\end{pmatrix}\sim\pm\begin{pmatrix}
			e^{-\chi/2}&0\\
			0&e^{\chi/2}
		\end{pmatrix}
	\end{equation}
\end{example}
\section{Higher dimensions}
\begin{definition}[赋范可除代数]
	首先考虑$\mathbb{R}$或者$\mathbb{C}$上的线性空间我们可以赋予乘法结构将其提升为代数,如果除了$0$元其它元素都有逆元,我们称为\textbf{可除代数},进一步我们可以赋予范数,而且要求范数满足:
	\[\left\|xy\right\|=\left\|x\right\|\left\|y\right\|,\forall x,y\in V\]
	即\textbf{赋范可除代数}。
\end{definition}
\begin{theorem}[Hurwitz]
	任何赋范可除代数都同构于$\mathbb{R},\mathbb{C},\mathbb{H},\mathbb{O}$中的一种。其中$\mathbb{H}$是四元数,$\mathbb{O}$是八元数。
\end{theorem}
这是一个非常漂亮的结论,告诉我们为什么历史上发现复数之后寻找三元数必然是失败的,而哈密顿的四元数会成功。

不难猜测,对于更高维时空的Lorentz群,会对应四元数和八元数,实际上有:
\begin{equation}
	SO(5,1)^\uparrow\cong SL(2,\mathbb{H})/\mathbb{Z}_2\quad\quad
	SO(9,1)^\uparrow\cong SL(2,\mathbb{O})/\mathbb{Z}_2
\end{equation}
另外,八元数实际上构成的不是一个结合代数,所以$SL(2,\mathbb{O})$的存在并非显然的,这里并不深入讨论。但是实际上这些同构关系式蕴含着很深刻的物理,暗示着最小超对称理论只能构建在3,4,6,10维时空中。更有意思的是,可除代数与超弦理论之间有着非常深刻的联系。\cite{Baez1,Baez2}
\section*{Interlude: Project representation}
{\color{red}\hrule}
\hspace*{\fill} \\

下面对Lorentz群表示的本身做更加细致的考量,主要是为了引进一些必要的数学概念,最终结论主要就是将一些讨论更加数学严格化,不用过分在意。本节讨论主要参照Weinberg\cite{Weinberg}。

对于一个对称群$G$,在考虑量子力学关于这个群的对称性时我们其实是在考虑其表示对Hilbert空间的作用。群表示自然是个同态,所以我们想到去考虑:
\begin{equation}
	U(T)U(\bar T)=U(T\cdot \bar T)
\end{equation}
但是量子力学的希尔伯特空间实际上是一个射影空间,两个相差全局相位的量子态视作等价,所以我们实际上应该去考虑群的\textbf{射影表示}:
\begin{equation}
	U(T)U(\bar T)=e^{i\phi(T,\bar T)}U(T\cdot \bar T)	
\end{equation}
这里利用线性可以证明$\phi(T,\bar T)$与所作用的量子态本身无关,但前提条件是这两个量子态是可加的。比如具有半整数自旋和整数自旋的量子态就是不可加的,最多只能制备出这俩态的直积态。

但是射影表示用起来很麻烦,如果相位具有下面的特殊结构:
\[\phi(T,\bar T)=\alpha(T\bar T)-\alpha(T)-\alpha(\bar T)\]
那我们可以对群表示后的算符重定义:
\begin{equation}\label{redefine1}
	\tilde U(T)=U(T)\exp\left(i\alpha(T)\right)
\end{equation}
这样我们就可以继续考虑普通的表示而不是射影表示了。现在我们必须严格考虑一个群是否存在不能通过重定义消去的射影表示,我们称为\textbf{内禀射影表示}。

既然考虑的是李群,那我们可以把群元用$\theta^a$参数化,并定义:
\[T(\bar\theta)T(\theta)=T\left(f(\bar\theta,\theta)\right)\]
根据$f(0,\theta)=f(0,\theta)=\theta$,我们得到单位元附近群元表示的展开:
\begin{equation}\label{eq:9.3}
	U\left(T(\theta)\right)=1+i\theta^a t_a+\frac{1}{2}\theta^b\theta^c t_{bc}+\mathcal{O}(\theta^3)
\end{equation}
这里$t_a$就是常说的生成元,在这个符号约定下是厄米的,$t_{ab}$关于指标对称,表示更高阶的项。另外$f$有展开:
\begin{equation}\label{eq:9.4}
	f^a(\bar\theta,\theta)=\theta^a+\theta^a+{f^a}_{bc}\bar\theta^b\theta^c+\mathcal{O}(\theta^3)
\end{equation}
类似地,因为$\phi(T,1)=\phi(T,1)=0$,我们有展开:
\begin{equation}\label{eq:9.5}
	\phi\left(T(\theta),T(\bar\theta)\right)=f_{ab}\theta^a\bar\theta^b
\end{equation}
结合\ref{eq:9.3},\ref{eq:9.4}和\ref{eq:9.5}我们得到:
\begin{equation}
	t_{bc}=-t_bt_c-i{f^a}_{bc}t_a-if_{bc}
\end{equation}
再根据$t_{bc}$的对称性有:
\begin{equation}
	[t_b,t_c]=i\eqnmarkbox[blue]{node1}{\left({f^a}_{cb}-{f^a}_{bc}\right)}t_a+i\eqnmarkbox[red]{node2}{\left(f_{cb}-f_{bc}\right)}\cdot\mathbb{I}
\end{equation}
\annotate[yshift=1em]{left}{node1}{denoted by ${C^a}_{bc}$}
\annotate[yshift=-1em]{below,label below}{node2}{denoted by $C_{bc}$}

在相位不为0时,生成元的对易关系之间多了一项$iC_{bc}\cdots\mathbb{I}$,称为\textbf{中心荷}。根据Jacobi恒等式,中心荷要满足方程:
\begin{equation}
	\begin{aligned}
		{C^a}_{bc}{C^e}_{ad}+{C^a}_{cd}{C^e}_{ab}+{C^a}_{db}{C^e}_{ac}&=0\\
		{C^a}_{bc}{C}_{ad}+{C^a}_{cd}{C}_{ab}+{C^a}_{db}{C}_{ac}&=0
	\end{aligned}
\end{equation}
这是与李代数具体结构无关的约束,给出一类特解:
\[C_{ab}={C^e}_{ab}\phi_e,\quad\phi_e\in\mathbb{R}\]
这类解到底存不存在要看李代数具体结构,但如果说李代数恰好是这样的解,那么我们可以通过重定义生成元:
\begin{equation}\label{redefine2}
	\tilde t_a=t_a+\phi_a\Rightarrow [\tilde t_b,\tilde t_c]=i{C^a}_{bc}\tilde t_a
\end{equation}
来消除中心荷。这引出了李群是否存在内禀射影表示的定理。
\begin{theorem}
	如果李群满足下面两个条件:
	\begin{itemize}
		\item[$\bullet$] 可以类似\ref{redefine2}重定义生成元消去所有中心荷。
		\item[$\bullet$] 群的拓扑结构单连通。
	\end{itemize}
	那么我么总可以类似\ref{redefine1}一样令相位为。
\end{theorem}
证明比较复杂,我们重点看在Poincar\'e群上的应用,另外,这个定理告诉我们只有两种方式产生内禀投影表示,一种是代数的,一种是拓扑的。

\begin{theorem}[V.Bargmann\cite{V.Bargmann}]
	半单Lie代数都可以通过重定义生成元消去中心荷
\end{theorem}
很幸运,齐次Lorentz群,也就是$M_{\mu\nu}$张成的代数是半单的,但不幸的是Poincar\'e代数不是半单的,不过更加幸运的是中心荷依旧可以被消除。

前面我们说明Lorentz群的拓扑结构使用了$QR$分解,实际上,使用另一种稍微不同的分解方式——极分解——可以证明拓扑结构其实同胚于$\mathbb{R}^3\times\mathcal{S}^3\times\mathbb{Z}_2$,Poincar\'e群多出来的那一部分,也就是$\mathbb{R}^4$是平凡的,重点在于$\mathcal{S}^3\times\mathbb{Z}_2$,这其实是个双连通结构。也就是说基本群为$\mathbb{Z}_2$。直观但不严谨的说就是初始点固定,转两圈总共回到初始点两次的“双圈”可以连续收缩到一点,但是单圈被分成两种,一种能收缩到一点,另一种必须再重复自己以此才能收缩到一点。

这么来看Lorentz群必须得用射影表示,我们看一下这个射影表示的特点,核心在于双圈可以收缩到单位元,所以$1\to\Lambda\to\Lambda\bar\Lambda\to1$的路径走两次等于单位元:
\begin{equation}
	\left[U(\Lambda)U(\bar\Lambda)U^{-1}(\Lambda\bar\Lambda)\right]^2=1\Rightarrow U(\Lambda)U(\bar\Lambda)=\pm U(\Lambda\bar\Lambda)
\end{equation} 

这里的正负号完全可以解释成自旋!整数自旋取正号,半整数取负号。而完整的描述应该是这个射影表示加上所谓“超选择定则”。前面我们说过相位是不依赖于态的,但前提条件是这些态是可加的,所以我们可以认为整数和半整数自旋存在超选择定则,也就是说它们不可加,那么相位就依赖于作用的态的自旋,我们实验上也确实发现了这种不可加性。这就构成了整个Lorentz群的表示(同样的推理也可以扩张到Poincar\'e群,毕竟它们拓扑结构一致)。

但是这样还是比较繁琐,但是从数学上看似乎引进射影表示是必然的,那我们能否从物理上把超选择定则给去掉呢?其实,我们完全可以把大自然真正的对称群取为$SL(2,\mathbb{C})$而不是Lorentz群$SL(2,\mathbb{C})/\mathbb{Z}$,这个更大的对称群只有普通表示,射影表示里的正负号随着表示本身的不同就自然的冒出来了。就像是$SU(2)$,看作是对$SO(3)$对称性的扩张,奇数维表示是简并表示,会出现负号,而偶数维是忠实表示,就只留下正号了。当然这一做法有代价,那就是抛弃了超选择定则\sn{总之除了超选择定则,其它完全一样。},这样不同自旋态的不可加性就不能从Lorentz不变性直接导出,但并不用担心这一点,毕竟实验上从未制备出这样的叠加态。

所以,对于任何对称群,如果存在中心荷,可以干脆扩张这个李代数,把与一切生成元都对易的生成元包含进来,这样就不存在中心荷了\sn{伽利略群就存在质量$M$这个中心荷,我们可以进行扩张,见A.Zee\cite{A.Zee}},但是同样也会丢弃超选择定则;如果李群不是单连通的,我们可以将其表示成$C/H$,其中$C$单连通,称为$G$的通用覆盖群\sn{类似的论述可以在\cite{PFS}对应章节找到}。然后把对称群取为$C$而不是$G$,这样扩张群之后就可以不用担心射影表示问题,也不用引入超选择定则。