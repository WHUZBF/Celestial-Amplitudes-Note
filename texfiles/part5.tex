\part{Soft Theorem}
% 计数器清零,每个part都要引用,除了part1
\setcounter{theorem}{0}
\setcounter{definition}{0}
\setcounter{lemma}{0}
\setcounter{sidenote}{1}
\section{Review on the quantization of gravity}
本节采用\cite{Badger:2023eqz}中的记号简要回顾引力量子化以及树图顶点项。众所周知微扰量子引力是不可重整化的,所以只能看作是一个有效理论,但好就好在它的低能极限是可以用于计算引力波振幅的\cite{Goldberger:2004jt,Bern:2019crd,Mogull:2020sak}。

取自由引力场拉式量为:
\begin{equation}
	\mathcal{L}_{EH}=\frac{2}{\kappa^2}\sqrt{-g}R,\quad \kappa=\sqrt{32\pi G}
\end{equation}
然后在平直度规附近做微扰:
\begin{equation}
	g_{\mu\nu}=\eta_{\mu\nu}+\kappa h_{\mu\nu}
\end{equation}
代入到$\mathcal{L}_{EH}$进行计算得到:
\begin{equation}
	\begin{aligned}
		\mathcal{L}_{\mathrm{EH}}& =\partial_{\alpha}h\partial_{\beta}h^{\alpha\beta}-\partial_{\alpha}h_{\beta\gamma}\partial^{\beta}h^{\alpha\gamma}-\frac{1}{2}(\partial_{\alpha}h)^{2}+\frac{1}{2}(\partial_{\gamma}h_{\alpha\beta})^{2}  \\
		&+\text{total derivatives}+\mathcal{O}\left(\kappa,h^{3}\right).
	\end{aligned}
\end{equation}
其中$h\equiv h^\alpha_\alpha$。这一堆都是$(\partial h)^2$项,对应动能项,显然由于没有$h^2$项,所以引力子质量为0。后面的高阶项意味着引力子之间是有自相互作用的。

由于广义相对论是微分同胚不变的,所以他其实也是个规范理论,不同的坐标系就意味着不同的规范,度规分量的变换正对应于场的规范变换。那么我们对$h_{\mu\nu}$进行路径积分时就必须在规范下进行,这可以使用F-P鬼场方法来系统的处理。比较常用的规范选取是下面的de-Donder规范:
\begin{equation}
	G_\mu\equiv\partial^\nu h_{\mu\nu}\frac{1}{2}\partial_\mu h=0
\end{equation}
这导致规范固定项:
\begin{equation}
	{\mathcal L}_{\mathrm{GF}}=G_{\mu}G^{\mu}=\partial^{\nu}h_{\mu\nu}\partial^{\rho}h^{\mu}{}_{\rho}+\frac{1}{4}(\partial_{\mu}h)^{2}-\partial^{\nu}h_{\mu\nu}\partial^{\mu}h
\end{equation}
以及鬼场:
\begin{equation}
	\mathcal{L}_{\mathrm{GH}}=-\bar{b}^{\mu}\left(\kappa\frac{\delta G_{\mu}}{\delta\xi^{\nu}}\right)b^{\nu}
\end{equation}
de-Donder规范下:
\begin{equation}
	\kappa\frac{\delta G_{\mu}}{\delta\xi^{\nu}}=\eta_{\mu\nu}\partial^{2}+\kappa\bigl[\partial^{\rho}h_{\mu\nu}\partial_{\rho}+\partial^{\rho}h_{\nu\rho}\partial_{\mu}+\partial^{\rho}(\partial_{\nu}h_{\mu\rho})-\partial_{\mu}h_{\nu\rho}\partial^{\rho}-\frac{1}{2}\partial_{\mu}(\partial_{\nu}h)\bigr]
\end{equation}
进行路径积分时,就不必再考虑规范,取而代之$\mathcal{L}_{EH}\mapsto\mathcal{L}_{EH}+\mathcal{L}_{GF}+\mathcal{L}_{GH}$,由于树图鬼场无贡献,所以这里只考虑前两项:
\begin{equation}
	\begin{aligned}
		\mathcal{L}_{\mathrm{EH}}|_{h^{2}}+\mathcal{L}_{\mathrm{GF}}& =-\frac{1}{2}h_{\alpha\beta}\partial^{2}h_{\alpha\beta}+\frac{1}{4}h\partial^{2}h  \\
		&=-\frac{1}{2}h_{\alpha\beta}\underbrace{\left[\eta^{\alpha(\gamma}\eta^{\delta)\beta}-\frac{1}{2}\eta^{\alpha\beta}\eta^{\gamma\delta}\right]}_{I^{\alpha\beta,\gamma\delta}}\partial^{2}h_{\gamma\delta} 
	\end{aligned}
\end{equation}
给出费曼图顶点和传播子项:
\begin{equation}
	\feynmandiagram [horizontal=a to b] {a[particle=\(\alpha\beta\)] -- [gluon] b[particle=\(\gamma\delta\)]};
	=\dfrac{\text{i}P_{\alpha\beta,\gamma\delta}}{p^2-\text{i}0^+}\quad\text{with}\quad P_{\alpha\beta,\gamma\delta}=\eta_{\alpha(\gamma}\eta_{\delta)\beta}-\dfrac{1}{D-2}\eta_{\alpha\beta}\eta_{\gamma\delta}
\end{equation}
其中$I^{\alpha\beta,\gamma\delta}P_{\gamma\delta,\rho\kappa}=\delta_{(\rho}^{\alpha}\delta_{\kappa)}^{\beta}$。高阶自相互作用顶点可以在\cite{Sannan:1986tz}中找到。

引力子自旋为$2$,但由于规范的限制,其仍然如光子一样只有两个独立的极化矢量(张量),对应引力波的两种偏振模式,只不过这个时候其实是个二阶对称极化张量。满足下面的条件:
\begin{equation} 
	p_\mu \epsilon^{\mu\nu}_{++/--} (p)=0,\quad \eta_{\mu\nu} \epsilon^{\mu\nu}_{++/--} (p)=0
\end{equation}
利用光子的极化矢量,可以构造出一组非常方便的$\epsilon^{\mu\nu}_{++/--}$选取:
\begin{equation}
	\epsilon^{\mu\nu}_{\pm\pm}=\epsilon^{\mu}_{\pm}\epsilon^{\nu}_{\pm}
\end{equation}

现在考虑引力场与其他场的耦合,耦合项拉式量为:
\begin{equation}
	\mathcal{L}_{\text{int}}=\frac{\kappa}{2}h^{\mu\nu}T_{\mu\nu}
\end{equation}
这里$T_{\mu\nu}$是场的能动张量,可以使用下式进行计算:
\begin{equation}
	T_{\mu\nu}(x)=-\frac{2}{\sqrt{-g}}\frac{\delta S_M}{\delta g_{\mu\nu}(x)}
\end{equation}
$S_M$是物质场的作用量,但是是在弯曲时空中的,可以使用最小耦合方法得到\sn{玻色场可以这么做,费米场复杂些}。以自旋为0对应的实标量场为例。
\begin{equation}\label{S_M}
	S_M=\int d^4x\sqrt{-\eta}\left(-\frac{1}{2}\eta^{\mu\nu}\partial_\mu\phi\partial_\nu\phi-V(\phi)\right)
\end{equation}
这里我们把所有暗含$\eta$的地方都显式的写出来了,弯曲时空中的作用量只需要将$\eta\mapsto g,\partial\mapsto\nabla$即可,得到:
\begin{equation}
	\tilde S_M=\int d^4x\sqrt{-g}\left(-\frac{1}{2}g^{\mu\nu}\nabla_\mu\phi\nabla_\nu\phi-V(\phi)\right)
\end{equation}

当然,对称性允许我们在后面添加正比于Ricci标量的项等等,但所谓我们要的就是“最小”耦合。不过在微扰引力框架下,我们其实只需要利用\ref{S_M}对$\eta$求变分导数就好了,对于无质量的自由标量场(对应不动点处的理论),我们得到:
\begin{equation}
	T_{\mu\nu}=\partial_\mu\phi\partial_\nu\phi-\frac{1}{2}\eta_{\mu\nu}\partial^\rho\phi\partial_\rho\phi
\end{equation}
对应的顶点项为:
\begin{equation}
	\feynmandiagram [inline=(d.base), horizontal=d to b] {
		a --[momentum=\(p\)]  b[dot] --[momentum=\(q\)]  c,
		b -- [gluon] d ,
	};
	= i \kappa p_{\mu}q_\nu +\text{正比于}\eta_{\mu\nu}\text{的项}
\end{equation}

\section{Soft theorem from feynman diagrams}