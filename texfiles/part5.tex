\part{Soft Theorem}
% 计数器清零,每个part都要引用,除了part1
\setcounter{theorem}{0}
\setcounter{definition}{0}
\setcounter{lemma}{0}
\setcounter{sidenote}{1}
\section{Review on the quantization of gravity}
本节采用\cite{Badger:2023eqz}中的记号简要回顾引力量子化以及树图顶点项。众所周知微扰量子引力是不可重整化的,所以只能看作是一个有效理论,但好就好在它的低能极限是可以用于计算引力波振幅的\cite{Goldberger:2004jt,Bern:2019crd,Mogull:2020sak}。

取自由引力场拉式量为:
\begin{equation}
	\mathcal{L}_{EH}=\frac{2}{\kappa^2}\sqrt{-g}R,\quad \kappa=\sqrt{32\pi G}
\end{equation}
然后在平直度规附近做微扰:
\begin{equation}
	g_{\mu\nu}=\eta_{\mu\nu}+\kappa h_{\mu\nu}
\end{equation}
代入到$\mathcal{L}_{EH}$进行计算得到:
\begin{equation}
	\begin{aligned}
		\mathcal{L}_{\mathrm{EH}}& =\partial_{\alpha}h\partial_{\beta}h^{\alpha\beta}-\partial_{\alpha}h_{\beta\gamma}\partial^{\beta}h^{\alpha\gamma}-\frac{1}{2}(\partial_{\alpha}h)^{2}+\frac{1}{2}(\partial_{\gamma}h_{\alpha\beta})^{2}  \\
		&+\text{total derivatives}+\mathcal{O}\left(\kappa,h^{3}\right).
	\end{aligned}
\end{equation}
其中$h\equiv h^\alpha_\alpha$。这一堆都是$(\partial h)^2$项,对应动能项,显然由于没有$h^2$项,所以引力子质量为0。后面的高阶项意味着引力子之间是有自相互作用的。

由于广义相对论是微分同胚不变的,所以他其实也是个规范理论,不同的坐标系就意味着不同的规范,度规分量的变换正对应于场的规范变换。那么我们对$h_{\mu\nu}$进行路径积分时就必须在规范下进行,这可以使用F-P鬼场方法来系统的处理。比较常用的规范选取是下面的de-Donder规范:
\begin{equation}
	G_\mu\equiv\partial^\nu h_{\mu\nu}\frac{1}{2}\partial_\mu h=0
\end{equation}
这导致规范固定项:
\begin{equation}
	{\mathcal L}_{\mathrm{GF}}=G_{\mu}G^{\mu}=\partial^{\nu}h_{\mu\nu}\partial^{\rho}h^{\mu}{}_{\rho}+\frac{1}{4}(\partial_{\mu}h)^{2}-\partial^{\nu}h_{\mu\nu}\partial^{\mu}h
\end{equation}
以及鬼场:
\begin{equation}
	\mathcal{L}_{\mathrm{GH}}=-\bar{b}^{\mu}\left(\kappa\frac{\delta G_{\mu}}{\delta\xi^{\nu}}\right)b^{\nu}
\end{equation}
de-Donder规范下:
\begin{equation}
	\kappa\frac{\delta G_{\mu}}{\delta\xi^{\nu}}=\eta_{\mu\nu}\partial^{2}+\kappa\bigl[\partial^{\rho}h_{\mu\nu}\partial_{\rho}+\partial^{\rho}h_{\nu\rho}\partial_{\mu}+\partial^{\rho}(\partial_{\nu}h_{\mu\rho})-\partial_{\mu}h_{\nu\rho}\partial^{\rho}-\frac{1}{2}\partial_{\mu}(\partial_{\nu}h)\bigr]
\end{equation}
进行路径积分时,就不必再考虑规范,取而代之$\mathcal{L}_{EH}\mapsto\mathcal{L}_{EH}+\mathcal{L}_{GF}+\mathcal{L}_{GH}$,由于树图鬼场无贡献,所以这里只考虑前两项:
\begin{equation}
	\begin{aligned}
		\mathcal{L}_{\mathrm{EH}}|_{h^{2}}+\mathcal{L}_{\mathrm{GF}}& =-\frac{1}{2}h_{\alpha\beta}\partial^{2}h_{\alpha\beta}+\frac{1}{4}h\partial^{2}h  \\
		&=-\frac{1}{2}h_{\alpha\beta}\underbrace{\left[\eta^{\alpha(\gamma}\eta^{\delta)\beta}-\frac{1}{2}\eta^{\alpha\beta}\eta^{\gamma\delta}\right]}_{I^{\alpha\beta,\gamma\delta}}\partial^{2}h_{\gamma\delta} 
	\end{aligned}
\end{equation}
给出费曼图顶点和传播子项:
\begin{equation}
	\feynmandiagram [horizontal=a to b] {a[particle=\(\alpha\beta\)] -- [gluon] b[particle=\(\gamma\delta\)]};
	=\dfrac{\text{i}P_{\alpha\beta,\gamma\delta}}{p^2-\text{i}0^+}\quad\text{with}\quad P_{\alpha\beta,\gamma\delta}=\eta_{\alpha(\gamma}\eta_{\delta)\beta}-\dfrac{1}{D-2}\eta_{\alpha\beta}\eta_{\gamma\delta}
\end{equation}
其中$I^{\alpha\beta,\gamma\delta}P_{\gamma\delta,\rho\kappa}=\delta_{(\rho}^{\alpha}\delta_{\kappa)}^{\beta}$。高阶自相互作用顶点可以在\cite{Sannan:1986tz}中找到。

引力子自旋为$2$,但由于规范的限制,其仍然如光子一样只有两个独立的极化矢量(张量),对应引力波的两种偏振模式,只不过这个时候其实是个二阶对称极化张量。满足下面的条件:
\begin{equation} 
	p_\mu \epsilon^{\mu\nu}_{++/--} (p)=0,\quad \eta_{\mu\nu} \epsilon^{\mu\nu}_{++/--} (p)=0
\end{equation}
利用光子的极化矢量,可以构造出一组非常方便的$\epsilon^{\mu\nu}_{++/--}$选取:
\begin{equation}
	\epsilon^{\mu\nu}_{\pm\pm}=\epsilon^{\mu}_{\pm}\epsilon^{\nu}_{\pm}
\end{equation}

现在考虑引力场与其他场的耦合,耦合项拉式量为:
\begin{equation}
	\mathcal{L}_{\text{int}}=\frac{\kappa}{2}h^{\mu\nu}T_{\mu\nu}
\end{equation}
这里$T_{\mu\nu}$是场的能动张量,可以使用下式进行计算:\sn{平直时空中能动量张量是根据Poincar\'e不变性引入的Noether current,可以说明(见\ref{eq:19.8})实际上等价于$$\left.T_{\mu\nu}\right|_{\text{flat}}=-2\left.\frac{\delta S_M}{\delta g^{\mu\nu}}\right|_{g^{\mu\nu}=\eta_{\mu\nu}}$$GR里面的能动量张量就是这个的推广。}
\begin{equation}
	T_{\mu\nu}(x)=-\frac{2}{\sqrt{-g}}\frac{\delta S_M}{\delta g_{\mu\nu}(x)}
\end{equation}
$S_M$是物质场的作用量,但是是在弯曲时空中的,可以使用最小耦合方法得到\sn{玻色场可以这么做,费米场复杂些}。以自旋为0对应的实标量场为例。
\begin{equation}\label{S_M}
	S_M=\int d^4x\sqrt{-\eta}\left(-\frac{1}{2}\eta^{\mu\nu}\partial_\mu\phi\partial_\nu\phi-V(\phi)\right)
\end{equation}
这里我们把所有暗含$\eta$的地方都显式的写出来了,弯曲时空中的作用量只需要将$\eta\mapsto g,\partial\mapsto\nabla$即可,得到:
\begin{equation}
	\tilde S_M=\int d^4x\sqrt{-g}\left(-\frac{1}{2}g^{\mu\nu}\nabla_\mu\phi\nabla_\nu\phi-V(\phi)\right)
\end{equation}

当然,对称性允许我们在后面添加正比于Ricci标量的项等等,但所谓我们要的就是“最小”耦合。不过在微扰引力框架下,我们其实只需要利用\ref{S_M}对$\eta$求变分导数就好了,对于无质量的自由标量场(对应不动点处的理论),我们得到\sn{实际上,考虑弯曲背景时空之后,多出了一些可重整的项可以加入到作用量中,比如$R\phi$,对于平直时空$R=0$,这一项不用考虑}:
\begin{equation}\label{eq:20.10}
	T_{\mu\nu}=\partial_\mu\phi\partial_\nu\phi-\frac{1}{2}\eta_{\mu\nu}\partial^\rho\phi\partial_\rho\phi
\end{equation}
对应的顶点项为:
\begin{equation}
	\feynmandiagram [inline=(d.base), horizontal=d to b] {
		a --[momentum=\(p\)]  b[dot] --[momentum=\(q\)]  c,
		b -- [gluon] d ,
	};
	= i \kappa p_{\mu}q_\nu +\text{正比于}\eta_{\mu\nu}\text{的项}
\end{equation}

\section{Soft theorem from feynman diagrams}
所谓软粒子,指的就是动量趋于0的无质量粒子,因为无质量$E=cp$,所以能量也会趋于0,任意过程都会辐射出期望值为无限多个的软粒子。本节使用费曼图方法讨论辐射软粒子的振幅修正\cite{McLoughlin:2022ljp}。

首先考虑辐射一个软光子,理论背景为旋量QED,散射过程为$ne^-\to me^-+\gamma $,注意考虑的是一般的旋量QED,可以有不同味的电子,电荷为$Q_i$。这里考虑的是电子,正电子的讨论也类似,最终结论是一样的。

辐射软光子可以从入射外线辐射,也可以从出射外线辐射,还可以从内线辐射:
\begin{figure}[H]
	\centering
	\includegraphics[width=0.8\linewidth]{figs/fig3.pdf}
	\caption{辐射软光子的三种情况}
\end{figure}
我们使用$\mathcal{A}(p,q)$表示辐射一个软光子后的振幅$\mathcal{A}(p)$表示没有辐射软光子的振幅(所有阶费曼图求和)。用$T(p_i-q)$表示原先没有辐射软光子的费曼图砍掉一个$p_i$外线但保留外线动量为非在壳的$p_i-q$得到的费曼图振幅,显然$T(p_i)u(p_i)=\mathcal{A}(p)$。

首先计算入射外线辐射软光子:
\begin{equation}
	\begin{aligned}
		Q_iT(p_i-q)\frac{(-\slashed{p}_i^\prime+\slashed{q}+m)\slashed{\epsilon}(q)}{(p_i-q)^2+m^2}u(p_i)&=-Q_iT(p_i-q)\Big[\frac{\epsilon(q)\cdot p_i+i\epsilon(q)_\mu q_\nu S^{\mu\nu}}{q\cdot p_i+\mathrm{i}0^+}\Big]u(p_i)
	\end{aligned}
\end{equation}
其中利用了$\slashed{a}\slashed{b}=-2a\cdot b-\slashed{b}\slashed{a}$,$q\cdot \epsilon_\pm(q)=0$以及狄拉克方程$(\slashed{p}+m)u(p)=0$。其中$S^{\mu\nu}=\frac{\mathrm{i}}{4}[\gamma^\mu,\gamma^\nu]$。唯一做近似的地方就是分母里面的$q^2$我们略去了,这对后面的高阶修正也不会影响。这一注意到$q=0$实际上是一个极点,这是因为在光子无线“软”的时候,多出来的那条内线传播子会无线趋于在壳,导致分母为0。

对于出射粒子辐射软光子也是类似的计算得到:
\begin{equation}\label{eq:20.17}
	\begin{aligned}
		Q^\prime_i\bar u(p_i^\prime)\frac{(-\slashed{p}_i^\prime-\slashed{q}+m)\slashed{\epsilon}(q)}{(p_i-q)^2+m^2}\bar T(p_i+q)&=Q^\prime_i\bar u(p_i^\prime)\Big[\frac{\epsilon(q)\cdot p_i+i\epsilon(q)_\mu q_\nu S^{\mu\nu}}{q\cdot p_i-\mathrm{i}0^+}\Big]\bar T(p_i-q)
	\end{aligned}
\end{equation}

至于内线发射光子,我们用$N^\mu(p,q)$表示,而且注意到$q\to0$时$N$非奇异,而且内线本身就是不在壳的,所以可以认为$N^\mu(p,q)\sim\mathcal{O}(q^0)$。三项加起来得到:
\begin{equation}\label{eq:21.3}
	\begin{aligned}
		\mathcal{A}(p,p^{\prime},q) =&\sum_\text{incoming }{ - Q _ i T ( p _ i - q )}\frac{\epsilon(q)\cdot p_i+i\epsilon(q)_{\mu}q_\nu S^{\mu\nu}}{q\cdot p_i+\mathrm{i}0^+}u(p_i)  \\
		&+\sum_{\mathrm{outgoing}}Q_{i}^{\prime}\bar{u}(p_{i}^{\prime})\frac{\epsilon(q)\cdot p_{i}^{\prime}+i\epsilon(q)_{\mu}q_{\nu}{S}^{\mu\nu}}{q\cdot p_{i}^{\prime}-\mathrm{i}0^+}\bar{T}(p_{i}^{\prime}+q)\\
		&+\epsilon(q)_{\mu}N^{\mu}(p,p^{\prime},q)
	\end{aligned}
\end{equation}
考虑最低阶修正,也就是只保留极点,得到:
\begin{equation}\label{eq:21.4}
	\boxed{
	\mathcal{A}(p,q)=\left[\sum_{i}\eta_iQ_{i}\frac{\epsilon(q)\cdot p_{i}}{q\cdot p_{i}}\right]\mathcal{A}(p)}+\mathcal{O}(1)
\end{equation}
其中对于入射粒子$\eta_i=-1$,出射粒子为$+1$。QED中辐射出光子的振幅都可以拆分成$\epsilon(q)_\mu \mathcal{M}^\mu$,振幅是相对论不变量,但是我们虽然常说$\epsilon(q)_\mu$是极化矢量,但它并不是真正意义上的矢量,因为其在lorentz变换下并不协变,而是会多出来一个正比于$q$的项\sn{我们是在Lorentz变换的意义下理解,其实矢量的这种变换完全可以理解为规范选取不同,从而从振幅的规范不变性导出结果。\cite{srednicki}}\cite{Weinberg}。为了让振幅是相对论不变量,必须有:
\begin{equation}\label{eq:21.5}
	\boxed{
	q_\mu\mathcal{M}^\mu=0}
\end{equation}
也就是Ward恒等式,也可以从Ward-高桥恒等式在U(1)对称性下导出它,实际上也就是利用电荷守恒导出它,但是前面我们的导出仅仅依赖于Lorentz不变性。现在把\ref{eq:21.4}中的$\epsilon(q)$替换为$q$我们得到:
\begin{equation}\label{eq:21.6}
	\left[\sum_i\eta_i Q_i\right]\mathcal{A}(p)=0
\end{equation}
也就是说,如果散射过程不被禁闭,也就是$\mathcal{A}(p)\neq 0$,那么这个过程必然要电荷守恒!另外说一句,我们这里的推导得到的\ref{eq:21.4}对于任意自旋的场都适用,比如对于标量QED,只需要把$S^{\mu\nu}$替换为0就好,不同的自旋对应$S^{\mu\nu}$的不同表示。

现在继续考虑两个光子的情况,两个光子由不同外线发射,在最低阶近似下只是乘上了\ref{eq:21.4}中两个因子,但是由相同外线发射就要考虑发射的先后顺序,因子形式会变化:
\begin{figure}[H]
	\centering
	\includegraphics[width=0.8\linewidth]{figs/fig4.pdf}
	\caption{同一外线辐射两个软光子}
\end{figure}
把两幅图贡献的因子加起来得到:
\begin{equation}
	\begin{aligned}
	&\left[\frac{\eta Q\epsilon(q_1)\cdot p}{p\cdot q_1-\mathrm{i}\eta0^+}\right]\left[\frac{\eta Q\epsilon(q_2)\cdot p}{p\cdot(q_2+q_1)-\mathrm{i}\eta 0^+}\right]+\left[\frac{\eta Q\epsilon(q_2)\cdot p}{p\cdot q_2-\mathrm{i}\eta0^+}\right]\left[\frac{\eta Q\epsilon(q_1)\cdot p}{p\cdot(q_1+q_2)-\mathrm{i}\eta 0^+}\right]\\=&\left[\frac{\eta Q\epsilon(q_1)\cdot p}{p\cdot q_1-\mathrm{i}\eta0^+}\right]\left[\frac{\eta Q\epsilon(q_2)\cdot p}{p\cdot q_2-\mathrm{i}\eta 0^+}\right]	
	\end{aligned}
\end{equation}
得到的结果就是不同外腿上面的辐射两个软光子得到的因子,利用数学归纳法可以证明这一结论对于辐射任意数量的软光子都是适用的,那么最终我们只需要把因子乘起来,然后对所有外腿求和就好了,用式子表示如下:
\begin{equation}
	\boxed{\mathcal{A}(p,q_1,\ldots,q_m)=\prod_{j=1}^m\left[\sum_{i=1}^nQ_i\eta_i\frac{\epsilon(q_j)\cdot p_i}{q_j\cdot p_i}\right]\mathcal{A}(p)++\mathcal{O}(1)}
\end{equation}
前面的因子称为\textbf{eikonal因子}。推广到non-Abelian Y-M场的软定理也早有人计算过了\cite{Berends:1987me,Berends:1988zn,Mangano:1987kp,Mangano:1990by},软定理更多更一般的推广见\cite{McLoughlin:2022ljp}的文献索引。

现在来考虑辐射一个软引力子的情况,考虑第一节引入的标量场模型,传播子为:
\begin{equation}
	-\frac{i }{(p+\eta q)^2+m^2}
\end{equation}
顶点由\ref{eq:20.17}给出,但是由于\ref{eq:20.10},要与$\epsilon^{\mu\nu}$缩并,正比于$\eta_{\mu\nu}$的项没有贡献。以出射外线辐射软引力子为例,给出因子:
\begin{equation}
	i\sqrt{32\pi G}\varepsilon^{\mu\nu}p_{\mu}p_{\nu}\frac{-i}{\left(p+q\right)^{2}+m^{2}}\rightarrow\sqrt{8\pi G}\frac{\varepsilon^{\mu\nu}p_{\mu}p_{\nu}}{p\cdot q}
\end{equation}
求和后得到:
\begin{equation}
	\boxed{
	\mathcal{A}(p,q)=\left[\sum_i\frac{\kappa_i}{2}\eta_i\frac{\epsilon_{\mu\nu}(q)p_i^\mu p_i^\nu}{q\cdot p_i}\right]\mathcal{A}(p)+\mathcal{O}(1)}
\end{equation}

引力子同样有类似于\ref{eq:21.5}的恒等式,由此我们可以得到所有的$\kappa_i$都是相等的\sn{这里有点循环论证了,因为前面为了推导的方便我们同一取$\kappa_i=\sqrt{32\pi G}$},也就是等效原理!而自旋大于二的软粒子的软定理沿用上面的方法给出的限制就太强了,以至于散射$\mathcal{S}$矩阵必须trivial,所以一般认为自旋大于二的无质量粒子在“变软”的时候就脱耦合了。
\section{Subleading and subsubleading order soft theorem}
考虑动量为$\delta q,\delta \to0$的软光子辐射,$\pm$表示软光子的螺旋度,而$\ell_i$表示其它粒子的螺旋度,则更一般的软定理可以用洛朗展开写成:
\begin{equation}
	\mathcal{A}_{\ell_1,...,\ell_n,\pm}(p,\delta q)=\left[\sum_{a=0}^\infty\delta^{-1+a}S_\pm^{(a)}\right]\mathcal{A}_{\ell_1,...,\ell_n}(p)
\end{equation}
次领头阶的计算可以从\ref{eq:21.5}式出发,将\ref{eq:21.3}中所有的$\epsilon$替换为动量,保留到$\mathcal{O}(\delta^0)$,并且令等式左边为0。注意到这一阶近似下可以有$N^\mu(p,q)\approx N^\mu(p,0)$,得到:
\begin{equation}
	-q^{\mu}N_{\mu}(p,0)=\cancelto{0}{\frac{1}{\delta}\sum_{i=1}^{n}\eta_{i}Q_{i}\mathcal{A}(p)}+q^{\mu}\sum_{i=1}^{n}Q_{i}\frac{\partial}{\partial p_{i}^{\mu}}\mathcal{A}(p)
\end{equation}
其中第一项为0是因为电荷守恒\ref{eq:21.6},第二项中的$\partial_{p_i}$只作用于$\mathcal{A}$中的$T$,不作用于$u$\sn{因为这个式子的导出,是对$T(p\pm q)$展开得到的。}。这样就确定出了$N^\mu$\sn{其实只能确定到某个与$q$无关的矢量$v$,满足$q\cdot v=0$,但是这样的$v$实际上不存在。},再带回到\ref{eq:21.3}得到:
\begin{equation}
	A^\mu=\sum_{i=1}^nQ_i\left[\frac{\eta_ip_i^\mu}{\delta q\cdot p_i}+\frac{q^\nu p_i^\mu}{q\cdot p_i}\frac{\partial}{\partial p_i^\nu}-\frac{iq_\nu S_i^{\mu\nu}}{q\cdot p_i}-\frac{\partial}{\partial p_{i\mu}}\right]\mathcal{A}(p)+\mathcal{O}(\delta)
\end{equation}
这个式子里面的$\partial_{p_i}$就是作用于整个$\mathcal{A}$了。定义:
\begin{equation}
	L_i^{\mu\nu}=i\left(p_i^\mu\frac{\partial}{\partial p_{i\nu}}-p_i^\nu\frac{\partial}{\partial p_{i\mu}}\right),\quad J^{\mu\nu}_i+S_i^{\mu\nu}
\end{equation}
这里$S_i^{\mu\nu}$要根据第i个粒子的自旋去选取对应的Lorentz群不可约表示,比如$s=0,S^{\mu\nu}=0;s=\frac{1}{2},S^{\mu\nu}=\frac{i}{4}[\gamma^\mu,\gamma^\nu]$。这样我们就得到了包含subleading修正的软定理(LBK定理\cite{PhysRevLett.20.86,PhysRev.110.974}):
\begin{equation}
	\begin{aligned}S_{\pm}^{(0)}&=\sum_{i=1}^nQ_i\eta_i\frac{\epsilon_{\pm\mu}(q)p_i^\mu}{q\cdot p_i},\quad\text{and}\quad S_{\pm}^{(1)}&=-i\sum_{i=1}^nQ_i\frac{\epsilon_{\pm\mu}(q)q_\nu J_i^{\mu\nu}}{q\cdot p_i}\end{aligned}
\end{equation}
同样的方法可以推导出软引力子定理的subleading和subsubleading的修正项\sn{软定理已经使用不同的方法推了很多遍,文献\cite{McLoughlin:2022ljp,Brandhuber:2022qbk}中有不同证明方法的文献索引,而且还有对于更复杂的非阿贝尔规范理论的软定理讨论。}\cite{PhysRevD.90.084035,PhysRev.168.1623,White:2014qia,Broedel:2014fsa}:
\begin{equation}
	\begin{gathered}
		S_{\pm}^{(0)}=\frac{\kappa}{2}\sum_{i=1}^{n}\eta_{i}\frac{\epsilon_{\pm\mu\nu}(q)p_{i}^{\mu}p_{i}^{\nu}}{q\cdot p_{i}},\quad S_{\pm}^{(1)}=-i\frac{\kappa}{2}\sum_{i=1}^{n}\frac{\epsilon_{\pm\mu\nu}(q)p_{i}^{\mu}q_{\lambda}J_{i}^{\nu\lambda}}{q\cdot p_{i}}, \\
		S_{\pm}^{(2)}=-\frac{\kappa}{4}\sum_{i=1}^{n}\eta_{i}\frac{\epsilon_{\pm\mu\nu}(q)q_{\rho}q_{\sigma}J_{i}^{\mu\rho}J_{i}^{\nu\sigma}}{q\cdot p_{i}}
	\end{gathered}
\end{equation}

\section{Massless QED}
本章用天球的视角来看QED理论,为下一节做准备,本节的讨论都局限于比较简单的,不含磁荷而且电子没有质量的QED理论,但是基于此推出的许多结论实际上是普适的,可以推广到有质量QED上\cite{Strominger:2017zoo}。
\subsection{Classical}
弯曲时空中QED作用量:
\begin{equation}
	S_{EM}=-\frac{1}{2e^2}\int{F}\wedge\star F +S_M=-\frac{1}{4e^2}\int \mathrm{d}^4x\sqrt{-g}F_{\mu]nu}F^{\mu\nu}+S_M
\end{equation}
其中$S_M$由$j^\nu A_\nu$形式给出相互作用项,即$j^\nu=-\frac{\delta S_M}{\delta A^\nu}$。Maxwell场方程有下面简洁形式:
\begin{equation}
	\begin{cases}
		dF=0&\Rightarrow \nabla_\mu F_{\nu\rho}+\nabla_\nu F_{\rho\mu}+\nabla_\rho F_{\mu\nu}=0\\
		d\star F=e^2\star j&\Rightarrow \nabla_\mu F_{\mu\nu}=e^2j_\nu
	\end{cases}
\end{equation}
下式给出了整个空间内的净电荷/磁荷\sn{后文$\star,*$指的都是Hodge dual}:
\begin{equation}
	Q_E=\frac{1}{e^2}\int_{\mathcal{S}^2_\infty}\star F=\int_{\Sigma}\star j,\quad Q_M=\frac{1}{2\pi}\int_{\mathcal{S}^2_\infty} F
\end{equation}
它们是量子化的,只能取整数。QED是$U(1)$规范理论,在$A\mapsto A+d\varepsilon$的规范变换下理论不变,后文不加说明都在所谓\textbf{retarded radial 规范}下进行计算\cite{He:2014cra,Kapec:2015ena}:
\begin{equation}
	\begin{aligned}
		&\mathcal{I}^+: A_r=0,\quad A_u|_{\mathcal{I}^+}=0\\
		&\mathcal{I}^-: A_r=0,\quad A_v|_{\mathcal{I}^-}=0
	\end{aligned}
\end{equation}
在$\mathcal{I}^{\pm}$附近可以把$A,F$展开成$\mathcal{O}(1/r)$的形式,用$A^{(i)},F^{(i)}$表示$r^{-i}$项前的系数,那么在这一规范下有\sn{各项系数都只和$(u,z,\bar z)$相关了}:
\begin{equation}
	F_{ur}^{(0)}=A_u^{(0)}, F_{z\bar z}^{(0)}=\partial_z A_{\bar z}^{(0)}-\partial_{\bar z}A_z^{(0)},F_{uz}^{(0)}=\partial_u A_z^{(0)},F_{rz}^{(0)}=-A_z^{(1)}
\end{equation}
而且我们所关心的场位形在$\mathcal{I}^{+}_{+}$处为真空,即满足:
\begin{equation}
	\left.F_{ur}\right|_{\mathcal{I}_{+}^{+}}=\left.F_{uz}\right|_{\mathcal{I}_{+}^{+}}=0
\end{equation}
$\mathcal{I}^-$处类似。但这个规范选取就如库伦规范$\nabla\cdot \mathbf{A}=0$一样没有完全确定规范,$A^{(0)}_z$还可以进行下面的所谓Large gauge transformation:
\begin{equation}
	A^{(0)}_z\mapsto A^{(0)}_z+\partial_z\varepsilon(z,\bar z),\quad \forall \varepsilon\in C^\infty\left[C\mathcal{S}^2\right]
\end{equation}
规范变换是一种冗余,但是这里的Large gauge transformations 最大的特点就是在无穷远处不归零,这导致很多时候分部积分的边界项不能丢去,确确实实的成为了一个对称性,后面马上会看到会对应无穷多个守恒荷。

运动电荷激发的电磁场即所谓Li\'enard\mbox{-}Wiecher势\cite{Jackson1998ClassicalE3}。这里我们在天球坐标下将advanced和retarded势统一写为:
\begin{equation}
	F_{rt}(\vec{x},t)=\frac{e^2}{4\pi}\sum_{k=1}^n\frac{Q_k\gamma_k\left(r-t\hat{x}\cdot\vec{\beta}_k\right)}{\left|\gamma_k^2\left(t-r\hat{x}\cdot\vec{\beta}_k\right)^2-t^2+r^2\right|^{3/2}},\quad r^2=\vec{x}\cdot\vec{x},\quad\vec{x}=r\hat{x}
\end{equation}
其中我们假设所有电荷之间匀速运动,而且相互作用可忽略,这个假设有点强,但是请\textbf{相信}后面导出的结论是可以推广到一般情形的。现在计算在天球上的极限:
\begin{equation}
	\begin{aligned}
		&\left.F_{rt}\right|_{\mathcal{I}^{+}}=\frac{e^{2}}{4\pi r^{2}}\sum_{k=1}^{n}\frac{Q_{k}}{\gamma_{k}^{2}(1-\hat{x}\cdot\vec{\beta}_{k})^{2}}\\
		&\left.F_{rt}\right|_{\mathcal{I}^{-}}=\frac{e^{2}}{4\pi r^{2}}\sum_{k=1}^{n}\frac{Q_{k}}{\gamma_{k}^{2}(1+\hat{x}\cdot\vec{\beta_{k}})^{2}}
	\end{aligned}
\end{equation}
取极限,得到:
\begin{equation}
	\boxed{
	\left.F_{rt}\right|_{\mathcal{I}^{+}_-}\neq \left.F_{rt}\right|_{\mathcal{I}^{-}_+}}
\end{equation}
正是这个式子说明了要严格区分$\mathcal{I}^{+}_-,\mathcal{I}^{-}_{+}$和$i^0$。但是,因为$F_{ru}=F_{rt}=F_{rv}$:
\begin{equation}
	\boxed{
		\lim\limits_{r\to\infty}r^2F_{ru}(\hat{x})\Big|_{\mathcal{I}_{-}^+}=\lim\limits_{r\to\infty}r^2F_{rv}(-\hat{x})\Big|_{\mathcal{I}_{+}^-}
	}
\end{equation}
也就是说场在两个天球上是对径认同的,这也是为什么前面$\mathcal{I}^{\pm}$天球的角向选取是antipodal的,上边的式子可以写成下面很简洁的形式:
\begin{equation}
	\boxed{
	\left.F_{ru}^{(2)}(z,\bar z)\right|_{\mathcal{I}^{+}_-}=\left.F_{rv}^{(2)}(z,\bar z)\right|_{\mathcal{I}^{-}_+}
	}
\end{equation}
这个式子是普适的,而且非常重要,是后面散射问题的核心。现在考虑任意一个天球上的对径认同的函数:
\begin{equation}
	\varepsilon(z,\overline{z})|_{\mathcal{I}_{-}^{+}}=\varepsilon(z,\overline{z})|_{\mathcal{I}_{+}^{-}}
\end{equation}
定义下面的future charges和past charges:
\begin{equation}
	\boxed{
		Q_{\varepsilon}^{+}=\frac{1}{e^{2}}\int_{\mathcal I_{-}^{+}}\varepsilon*F,\quad Q_{\varepsilon}^{-}=\frac{1}{e^{2}}\int_{\mathcal I_{+}^{-}}\varepsilon*F
	}
\end{equation}
由于Hodge对偶后出来的体积元$\propto r^2$,所以在天球上$r\to\infty$,$F\to F^{(2)}$,再根据对径认同的条件,这样对于任意一个函数$\varepsilon$,我们都给出了一个“电荷”守恒律:
\begin{equation}
	\boxed{
	Q_\varepsilon^+=Q_\varepsilon^-
	}
\end{equation}
对于$\varepsilon$是常函数情形,就得到了一般的电荷守恒律。利用Stokes公式\sn{$\int_{\partial \Omega}\omega =\int_{ \Omega}d\omega$}可以改写$Q_\varepsilon^{\pm}$为\sn{$Q_\varepsilon^-$只用把$+\mapsto -$}:
\begin{equation}
	Q_\varepsilon^+=\frac{1}{e^2}\int_{\mathcal{I}^+}\mathrm{d}\varepsilon\wedge*F+\int_{\mathcal{I}^+}\varepsilon*j+\cancelto{0}{\frac{1}{e^2}\int_{\mathcal{I}_+^+}\varepsilon*F}
\end{equation}
最后一项为0是因为我们假设电子是无质量的,这样电子就是从$\mathcal{I}^-\to\mathcal{I}^+$,而不是$i^{-}\to i^+$,所以$F|_{\mathcal{I}_+^{+}}=0$。第一项是Soft term $Q_S^+$与软光子的产生湮灭有关,第二项由于是与流耦合,所以叫Hard term $Q_H^+$与带电实物粒子有关。将上面的微分形式写成分量形式\sn{其中使用了Bianchi恒等式:$\partial_{u}F_{ru}^{(2)}+D^{z}F_{uz}^{(0)}+D^{\bar{z}}F_{u\bar{z}}^{(0)}+e^2j_{u}^{(2)}=0$}:
\begin{equation}
	Q_{\varepsilon}^{+}=\underbrace{-\frac{1}{e^{2}}\int_{\mathcal{I}^{+}}dud^{2}z\left(\partial_{z}\varepsilon F_{u\bar{z}}^{(0)}+\partial_{\bar{z}}\varepsilon F_{uz}^{(0)}\right)}_{Q_{S}^{+}}+\underbrace{\int_{\mathcal{I}^{+}}dud^{2}z\varepsilon\gamma_{z\bar{z}}j_{u}^{(2)}}_{Q_{H}^{+}}
\end{equation}
定义Soft phonton mode $N_z$:
\begin{equation}\label{eq:23.18}
	N_{z}\equiv \int_{-\infty}^{\infty}duF_{uz}^{(0)}=\lim_{\omega\to0}\int_{-\infty}^{\infty}duF_{uz}^{(0)}e^{i\omega u}
\end{equation}
再次看到$N_z$对应的是电磁场的零频部分,也就是软光子部分,而且有:
\begin{equation}
	\begin{aligned}
		\partial_{\bar{z}}N_{z}-\partial_{z}N_{\bar{z}}& =\int_{-\infty}^{\infty}du\left[\partial_{\bar{z}}F_{uz}^{(0)}-\partial_{z}F_{u\bar{z}}^{(0)}\right]  \\
		&=-\int_{-\infty}^{\infty}du\left.\partial_{u}F_{z\bar{z}}^{(0)}=-F_{z\bar{z}}^{(0)}\right|_{\mathcal{I}_{-}^{+}}^{\mathcal{I}_{+}^{+}}=0
	\end{aligned}
\end{equation}
最后等于0是因为$F_{z\bar z}|_{\mathcal{I}_{+}^{\pm}}=0$,本质上是因为没有磁单极子,磁场不是long-range的,在无穷远处为0\sn{因为$F_{z\bar z}=\frac{\partial x^{\mu}}{\partial z}\frac{\partial x^{\nu}}{\partial\bar{z}}F_{\mu\nu}=\frac{\partial x^{i}}{\partial z}\frac{\partial x^{j}}{\partial\bar{z}}F_{ij}$,而规范场强的空间分量$F_{ij}$只和磁场有关。}。所以$N_z$无旋,可以由此引申出一个实标量场$N$作为其原函数:
\begin{equation}
	N_z\equiv e^2\partial_zN=\int_{-\infty}^{\infty}duF_{uz}^{(0)}=A_{z}^{(0)}|_{\mathcal{I}_{+}^{+}}-A_{z}^{(0)}|_{\mathcal{I}_{-}^{+}}
\end{equation}
利用$N$,future charges可以写成下面简洁形式:
\begin{equation}\label{eq:23.21}
	Q_{\varepsilon}^{+}=2\int_{\mathcal{S}_\infty^2} d^{2}zN\partial_{z}\partial_{\bar{z}}\varepsilon+\int_{\mathcal I^{+}}dud^{2}z\varepsilon\gamma_{z\bar{z}}j_{u}^{(2)}
\end{equation}
对past charge也可以来上面的这一套:
\begin{equation}\label{eq:23.21.2}
	Q_{\varepsilon}^{-}=-2\int_{\mathcal{S}_\infty^2} d^{2}zN^-\partial_{z}\partial_{\bar{z}}\varepsilon+\int_{\mathcal I^{-}}dvd^{2}z\varepsilon\gamma_{z\bar{z}}j_{v}^{(2)}
\end{equation}
\subsection{Quantization}
上面讨论的都是经典场,现在进行量子化,目的是说明前面定义的future charge 和 past charges 都是 Large gauge transformations 对应的生成元算符。量子化方案选用正则量子化方案,这里使用的方法不需要事先对时空进行3+1分解,是协变的方法\cite{Ashtekar1987AsymptoticQ,Frolov:1979ab,1989thyg.book.....H,Lee:1990nz,Wald:1999wa}。

经典力学的正则形式是建立在辛几何上的,也就是给相流形上配备了一个非退化的、闭的二形式\cite{Arnold}:
\begin{equation}\label{eq:23.22}
	\Omega=\frac{1}{2}\omega_{IJ}dx^I\wedge dx^J
\end{equation}
而且可以证明,可以通过选取不同的广义坐标局部的将$\omega_{IJ}$化为下面的矩阵:
\begin{equation}
	\omega_{IJ}=\left.\left[\begin{matrix}0&1_{N\times N}\\-1_{N\times N}&0\end{matrix}\right.\right]
\end{equation}
力学量的Possion括号由下式给定:
\begin{equation}
	\{A,B\}=\Omega^{IJ}\partial_IA\partial_JB
\end{equation}
相空间是运动方程的解,可以与初始值一一对应。

到了场论这边,相空间我们可以选取为某一个柯西面$\Sigma$上的场位形。场$\phi$本身是流形上的微分形式,\ref{eq:21.22}中的$dx$应该替换为$\delta\phi$,这里$\delta$是场位形的在壳变分,也就是说$\phi+\delta \phi$仍然是场方程的解。场本身是定义在某个纤维丛上的,可以粗浅的理解为两个流形拼起来,而$d$是底流形上的微分形式算子,$\delta$事实上是另一个正交的微分形式算子,所以$\delta\phi$在$\delta$的意义上是1\mbox{-}form,在$d$的意义上$\delta$的作用不会改变其是几形式,这样$\Omega$形如$\delta\phi\wedge\delta\varphi$从$\delta$的意义上看就是个二形式。构建的微分形式还应当满足规范不变性,闭且非退化,而且还要对柯西面的不同选取一致,利用\cite{Wald:1999wa}中方法可以得到$U(1)$规范场的辛形式为:
\begin{equation}
	\Omega_{\Sigma}=-\frac{1}{e^{2}}\int_{\Sigma}\delta(*F)\wedge\delta A
\end{equation}
将柯西面选取为$\mathcal{I}^{\pm}$,得到分量为:
\begin{equation}
	\begin{aligned}
		\omega_{uz\bar{z}}|_{\mathcal{I}}+& =-\frac{1}{e^{2}}\left[\delta(*F)_{uz}\wedge\delta A_{\bar{z}}+\delta(*F)_{\bar{z}u}\wedge\delta A_{z}+\delta(*F)_{z\bar{z}}\wedge\delta A_{u}\right]_{\mathcal{I}^{+}}  \\
		&=-\frac{i}{e^{2}}\left[\delta F_{uz}^{(0)}\wedge\delta A_{\bar{z}}^{(0)}+\delta F_{u\bar{z}}^{(0)}\wedge\delta A_{z}^{(0)}\right],
	\end{aligned}
\end{equation}
其中利用了:\sn{注意到Levi-Civita符号不是张量,而是张量密度,$\sqrt{-g}\epsilon$才是张量,然后利用Hodge dual定义爆算即可}
\begin{equation}
	(*F)_{uz}=i\left(F_{uz}-F_{rz}\right),\quad(*F)_{z\bar{z}}=ir^{2}\gamma_{z\bar{z}}F_{ru}
\end{equation}
最终得到辛形式:
\begin{equation}\label{eq:23.29}
	\Omega_{\mathcal{I}^+}=\frac{1}{e^2}\int dud^2z\left[\delta F_{uz}^{(0)}\wedge\delta A_{\bar{z}}^{(0)}+\delta F_{u\bar{z}}^{(0)}\wedge\delta A_{z}^{(0)}\right]
\end{equation}
总是可以把$A_z$的$u\to\pm {\infty}$,也就是$\mathcal{I}^+_{\pm}$的与$u$无关的部分分离出去,这一部分又总是可以写成某个天球上标量场的导数,因为要求$F_{z\bar z}|_{\mathcal{I}_{+}^{\pm}}=0$:
\begin{equation}
	A_z^{(0)}(u,z,\bar{z})=\hat{A}_z(u,z,\bar{z})+\partial_z\phi(z,\bar{z}),\quad\partial_z\phi\equiv\frac{1}{2}\left[\left.A_z^{(0)}\right|_{\mathcal I_+^+}+\left.A_z^{(0)}\right|_{\mathcal I_-^+}\right]
\end{equation}
利用上面的分解以及soft photon mode重写\ref{eq:23.29}:
\begin{equation}
	\Omega_{\mathcal{I}^{+}}=\frac{2}{e^{2}}\int dud^{2}z\partial_{u}\delta\hat{A}_{z}\wedge\delta\hat{A}_{\bar{z}}-2\int d^{2}z\partial_{z}\delta\phi\wedge\partial_{\bar{z}}\delta N
\end{equation}
现在考虑正则量子化,场变成算符,而Poisson括号换成Dirac括号:\sn{$\{\quad,\quad\}\mapsto\frac{1}{i}[\quad,\quad]$}
\begin{equation}
	\begin{aligned}-\frac{2}{e^2}\left[\partial_u\hat A_z(u,z,\bar z),\hat A_{\bar w}(u',w,\bar w)\right]&=i\delta(u-u')\delta^2(z-w),\\2\left[\partial_z\phi(z,\bar z),\partial_{\bar w}N(w,\bar w)\right]&=i\delta^2(z-w).\end{aligned}
\end{equation}
积分得到:\sn{这里$$\Theta(u)=\frac{1}{\pi i}\int\frac{d\omega}{\omega}e^{i\omega u}=\begin{cases}+1&,u>0\\-1&,u<0\end{cases}$$}
\begin{equation}
	\begin{aligned}
		\left[\hat{A}_{z}(u,z,\bar{z}),\hat{A}_{\bar{w}}(u^{\prime},w,\bar{w})\right]& =-\frac{ie^{2}}{4}\Theta(u-u^{\prime})\delta^{2}(z-w),  \\
		[\phi(z,\bar{z}),N(w,\bar{w})]& =-\frac{i}{4\pi}\log|z-w|^{2}+f(z,\bar{z})+g(w,\bar{w}). 
	\end{aligned}
\end{equation}

\subsection{Large guage symmetry}
利用上面发展的对易式可以去计算$Q_\varepsilon^{\pm}$与场算符的对易关系,注意到物质场部分$Q_H$始终和规范场$A$对易,所以:
\begin{equation}
	\begin{aligned}&\left[Q_{\varepsilon}^{+},A_{z}^{(0)}(u,z,\bar{z})\right]=i\partial_{z}\varepsilon(z,\bar{z}),\\&\left[Q_{\varepsilon}^{-},A_{z}^{(0)}(v,z,\bar{z})\right]=i\partial_{z}\varepsilon(z,\bar{z}).\end{aligned}
\end{equation}
所以前面定义的守恒荷其实就是对径认同的Large gauge transformation的生成元,自然就是一个守恒荷!前面的讨论一直忽略了流耦合项$S_M$,加入后对前面的结论不会有影响,注意到$Q_S$与物质场对易,根据Noether定理:
\begin{equation}
	\left[j_{u}^{(2)}(u',w,\bar{w}),\Phi_{k}(u,z,\bar{z})\right]=-Q_{k}\Phi_{k}(u,z,\bar{z})\gamma^{z\bar{z}}\delta^{2}(z-w)\delta(u-u')
\end{equation}
这意味着:
\begin{equation}\label{23.36}
	\left[Q_\varepsilon^+,\Phi_k(u,z,\bar{z})\right]=\left[\int_{\mathcal{I}^+}\varepsilon*j,\Phi_k(u,z,\bar{z})\right]=-Q_k\varepsilon(z,\bar{z})\Phi_k(u,z,\bar{z})\equiv i\delta_\varepsilon\Phi_k(u,z,\bar{z})
\end{equation}
所以$Q_S$生成规范场的规范变换,$Q_H$生成费米场规范变换,合起来$Q_\varepsilon$生成$\mathcal{I}^+$上的local规范变换。
\section{Ward indentity = Soft theorem}
\subsection{Ward indentity of $\mathcal{S}$\mbox{-}Matrix}
量子力学里面就知道守恒意味着与哈密顿量对易,而$\mathcal{S}\sim \exp{iHT}$,所以前面的守恒荷会和$\mathcal{S}$对易:\sn{初末态基底选取平面波。}
\begin{equation}\label{eq:24.1}
	\boxed{\langle\mathrm{out}|\left(Q_\varepsilon^+\mathcal{S}-\mathcal{S}Q_\varepsilon^-\right)|\mathrm{in}\rangle=0}
\end{equation}
这样一个简单的式子就是$\mathcal{S}$\mbox{-}Matrix的Ward identity。利用\ref{eq:23.21}和\ref{eq:23.21.2}得到:
\begin{equation}
	\begin{aligned}
		Q_\varepsilon^-|\mathrm{in}\rangle&=-2\int d^2z\partial_{\bar{z}}\varepsilon\partial_zN^-(z,\bar{z})|\mathrm{in}\rangle+\sum_{k=1}^mQ_k^\mathrm{in}\varepsilon(z_k^\mathrm{in},\bar{z}_k^\mathrm{in})|\mathrm{in}\rangle \\
		\langle\mathrm{out}|Q_\varepsilon^+&=2\int d^2z\partial_z\partial_{\bar{z}}\varepsilon\langle\mathrm{out}|N(z,\bar{z})+\sum_{k=1}^nQ_k^\mathrm{out}\varepsilon(z_k^\mathrm{out},\bar{z}_k^\mathrm{out})\langle\mathrm{out}|
	\end{aligned}
\end{equation}
第一项没啥好说的,后面会讨论其物理含义,第二项的计算稍微提一下。首先我们假设入射态是$m$个hard particles,而且来自于天球上的点$z_k^{\mathrm{in}}$,\ref{23.36}给出了守恒荷与物质场之间的对易关系,实际上,不难证明守恒荷和产生湮灭算符也满足同样的对易关系\sn{比如标量场\cite{srednicki}$$a^{\dagger}(\mathbf{k})=-i\int d^{3}xe^{ikx}\stackrel{\leftrightarrow}{\partial_{0}}\varphi(x)$$}。以两个入射粒子为例:
\begin{equation}
	\begin{aligned}
		Q^-_{\varepsilon}\ket{\text{in}}&\sim Q^-_{\varepsilon}a_1^\dagger a_2^\dagger\ket{0}\\
		&\sim \left[Q^-_{\varepsilon},a_1^\dagger a_2^\dagger\right]\ket{0}+\cancelto{0}{a_1^\dagger a_2^\dagger Q^-_{\varepsilon}\ket{0}}\\
		&\sim  \left[Q^-_{\varepsilon},a_1^\dagger \right]a_2^\dagger\ket{0}+a_1^\dagger\left[Q^-_{\varepsilon},a_2^\dagger \right]\ket{0}\\
		&\sim Q_1\varepsilon_1\ket{\text{in}}+Q_2\varepsilon_2\ket{\text{in}}
	\end{aligned}
\end{equation}
第二行我们利用了真空态一定是没荷的。现在可以把\ref{eq:24.1}写为:
\begin{equation}
	\begin{aligned}
		2\int d^2z\partial_z\partial_{\bar{z}}\varepsilon\langle\text{out}|&\left(N(z,\bar{z})\mathcal{S}-\mathcal{S}N^-(z,\bar{z})\right)|\text{in}\rangle\\&=\left[\sum_{k=1}^mQ_k^\text{in}\varepsilon(z_k^\text{in},\bar{z}_k^\text{in})-\sum_{k=1}^nQ_k^\text{out}\varepsilon(z_k^\text{out},{\bar{z}}_k^\text{out})\right]\langle\text{out}|\mathcal{S}|\text{in}\rangle
	\end{aligned}
\end{equation}
取$\varepsilon(z,\bar z)=\frac{1}{w-z}$,则上式可以写为如下形式:\sn{$$\partial_{\bar{z}}\frac{1}{z-w}=2\pi\delta^2(z-w)$$}
\begin{equation}\label{eq:24.5}
	4\pi\langle\mathrm{out}|\left(\partial_zN\mathcal{S}-\mathcal{S}\partial_zN^-\right)|\mathrm{in}\rangle=\left[\sum_{k=1}^m\frac{Q_k^\mathrm{in}}{z-z_k^\mathrm{in}}-\sum_{k=1}^n\frac{Q_k^\mathrm{out}}{z-z_k^\mathrm{out}}\right]\langle\mathrm{out}|\mathcal{S}|\mathrm{in}\rangle 
\end{equation}
这个等式其实就是带有U(1) K$\breve{\text{a}}$c-Moody current的$\text{CFT}_2$的Ward恒等式\cite{Blumenhagen:2009zz,He:2015zea,Strominger:2013lka,Nande:2017dba}。
\subsection{Mode expansion}
本部分的终极目标是证明Ward恒等式可以联系渐近对称性和软定理,前文从相空间角度给出了Ward恒等式,在动量空间给出了软定理,是时候在动量空间重写Ward恒等式了,重点就是将场算符用平面波展开,也就是正则量子化的过程。free mode expansion 弯曲时空量子力学领域已经不少人算过了,U(1)规范理论结果如下:
\begin{equation}
	A_{\nu}(x)=e\sum_{\alpha=\pm}\int\frac{d^3q}{\left(2\pi\right)^3}\frac{1}{2\omega_q}\left[\varepsilon_{\nu}^{*\alpha}(\vec{q})a_{\alpha}^{\mathrm{out}}(\vec{q})e^{iq\cdot x}+\varepsilon_{\nu}^{\alpha}(\vec{q})a_{\alpha}^{\mathrm{out}}(\vec{q})^\dagger e^{-iq\cdot x}\right]
\end{equation}
$q$是平面波在壳动量$q^2=0$,$\varepsilon^\pm_\mu$是极化矢量,常常取为:
\begin{equation}
	\varepsilon^{+\mu}(\vec{q})=\frac1{\sqrt{2}}\left(\bar{z},1,-i,-\bar{z}\right),\quad\varepsilon^{-\mu}(\vec{q})=\frac1{\sqrt{2}}\left(z,1,i,-z\right),\quad q_{\mu}\varepsilon^{\pm\mu}(\vec{q})=0,\quad\varepsilon_{\alpha}^{\mu}\varepsilon_{\beta\mu}^{*}=\delta_{\alpha\beta}
\end{equation}
产生湮灭算符之间满足对易关系:
\begin{equation}
	\left[a_\alpha^{\mathrm{out}}(\vec q),a_\beta^{\mathrm{out}}(\vec q')^\dagger\right]=\delta_{\alpha\beta}(2\pi)^3(2\omega_q)\delta^3\left(\vec q-\vec q'\right)
\end{equation}
定义内积:
\begin{equation}
	(A,A')=-i\int d\Sigma^{\mu}[A^{\nu}(\nabla_{\mu}{A'}_{\nu}^{*}-\nabla_{\nu}{A'}_{\mu}^{*})-(A\leftrightarrow {A'}^{*})]
\end{equation}
这样定义的内积与类空超曲面$\Sigma$的选取是无关的。这样一来产生湮灭算符可以写成:
\begin{equation}
	a_\pm(q)=i({A},(\epsilon^\pm e^{iq\cdot x})^*)
\end{equation}
对于$\mathcal{I}^-$类似讨论,结果只是把out改成in。下面计算$A_z^{(0)}(u,z,\bar{z})\equiv\lim_{r\to\infty}A_z(u,r,z,\bar{z})$的模式展开:\sn{这里$\hat q,\hat x$都是指其空间部分}
\begin{equation}
	\begin{aligned}
		A_{\mu}(x) =&e\sum_{\alpha=\pm}\int\frac{d^3q}{\left(2\pi\right)^3}\frac{1}{2\omega_q}\left[\varepsilon_{\mu}^{*\alpha}(\vec{q})a_{\alpha}(\vec{q})e^{-i\omega_qu-i\omega_qr(1-\hat{q}\cdot\hat{x})}\right. \\
		&\left.+\varepsilon_{\mu}^{\alpha}(\vec{q})a_{\alpha}^{\dagger}(\vec{q})e^{i\omega_{q}u+i\omega_{q}r(1-\hat{q}\cdot\hat{x})} \right]\\
		=&\frac e{8\pi^2}\sum_{\alpha=\pm}\int_0^\infty d\omega_q\omega_q\int_0^\pi d\theta\sin\theta\left[\varepsilon_\mu^{*\alpha}(\vec{q})a_\alpha(\vec{q})e^{-i\omega_qu-i\omega_qr(1-\cos\theta)}\right. \\
		&\left.+\varepsilon_{\mu}^{\alpha}(\vec{q})a_{\alpha}^{\dagger}(\vec{q})e^{i\omega_{q}u+i\omega_{q}r(1-\cos\theta)}\right].
	\end{aligned}
\end{equation}
这里$\theta$是$\hat q$和$\hat x$之间的夹角,利用数学公式\sn{来源是鞍点近似}:
\begin{equation}
	\lim_{r\to\infty}\sin\theta e^{i\omega_qr(1-\cos\theta)}=\frac{i}{\omega_qr}\delta(\theta)+\mathcal{O}((r)^{-2})
\end{equation}
代入得到:
\begin{equation}
	A_\mu(x)=-\frac{ie}{8\pi^{2}r}\sum_{\alpha=\pm}\int_{0}^{\infty}d\omega_{q}\left[\varepsilon_{\mu}^{*\alpha}(\omega_{q}\hat{x})a_{\alpha}(\omega_{q}\hat{x})e^{-i\omega_{q}u}-c.c.\right]+\mathcal{O}(r^{-2})
\end{equation}
我们要求的玩意儿是$A_{z}=\partial_{z}x^{\mu}A_{\mu}$,注意到:
\[
	\partial_{z}x^{\mu}\varepsilon_{\mu}^{+}(\omega_{q}\hat{x})=0,\quad\partial_{z}x^{\mu}\varepsilon_{\mu}^{-}(\omega_{q}\hat{x})=\frac{\sqrt{2}r}{1+z\bar{z}}
\]
最终求得:\sn{下文中的$\omega$指$\omega_q$,即$q^0$}
\begin{equation}
	A_{z}^{(0)}(u,z,\bar{z})=-\frac{i}{8\pi^2}\frac{\sqrt{2}e}{1+z\bar{z}}\int_{0}^{\infty}d\omega\left[a_{+}^{\mathrm{out}}(\omega\hat{x})e^{-i\omega u}-a_{-}^{\mathrm{out}}(\omega\hat{x})^\dagger e^{i\omega u}\right]
\end{equation}
把定义式\ref{eq:28.18}改写为厄米的形式:
\begin{equation}
	\partial_zN=\dfrac{1}{2e^2}\lim_{\omega\to0^+}\int_{-\infty}^{\infty}du\left(e^{i\omega u}+e^{-i\omega u}\right)F_{uz}^{(0)}
\end{equation}
利用$A_z^{(0)}$重写上式为:
\begin{equation}
	\partial_zN=-\frac{1}{8\pi e}\frac{\sqrt{2}}{1+z\bar{z}}\lim_{\omega\to0^+}\left[\omega a_+^{\mathrm{out}}(\omega\hat{x})+\omega a_-^{\mathrm{out}}(\omega\hat{x})^\dagger\right]
\end{equation}
在$\mathcal{I}^-$上也有类似式子,只需要把$N\to N^-$,out$\to$in 就好。现在可以看出来为啥要叫soft photon mode 了。根据前面的铺垫,\ref{eq:24.5}终于可以写成如下形式:
\begin{equation}
	\begin{aligned}
		\operatorname*{lim}_{\omega\rightarrow0}\left\lfloor\omega\langle\mathrm{out}|\left(a_{+}^{\mathrm{out}}(\omega\hat{x})\mathcal{S}\right.\right.& \left.-\mathcal{S}a_{-}^{\mathrm{in}}(\omega\hat{x})^{\dagger}\right)|\mathrm{in}\rangle   \\
		&=\sqrt{2}e(1+z\bar{z})\left[\sum_{k=1}^{n}\frac{Q_{k}^\mathrm{out}}{z-z_{k}^\mathrm{out}}-\sum_{k=1}^{m}\frac{Q_{k}^\mathrm{in}}{z-z_{k}^\mathrm{in}}\right]\langle\mathrm{out}|\mathcal{S}|\mathrm{in}\rangle 
	\end{aligned}
\end{equation}
\subsection{Soft photon \& graviton}
前面用费曼图导出的软定理可以用$\mathcal{S}$矩阵写成如下形式:
\begin{equation}
	\lim\limits_{\omega\to0}\left[\omega\langle\text{out}|a_+^\text{out}(\vec{q})\mathcal{S}|\text{in}\rangle\right]=e\lim\limits_{\omega\to0}\left[\sum\limits_{k=1}^m\frac{\omega Q_k^\text{out}p_k^\text{out}\cdot\varepsilon^+}{p_k^\text{out}\cdot q}-\sum\limits_{k=1}^n\frac{\omega Q_k^\text{in}p_k^\text{in}\cdot\varepsilon^+}{p_k^\text{in}\cdot q}\right]\langle\text{out}|\mathcal{S}|\text{in}\rangle 
\end{equation}
这里出射和入射态都是平面波为基,而且我们把元电荷作为公因子提出,$Q\in\mathbb{Z}$,出入射动量用Embedding的形式写出来,前面第三部分写过,这里再写一遍:
\begin{equation}
	\begin{aligned}
		&q^{\mu} =\frac\omega{1+z\bar{z}}\left(1+z\bar{z},z+\bar{z},-i(z-\bar{z}),1-z\bar{z}\right),  \\
		&p_{k}^{\mu} =\frac{E_{k}}{1+z_{k}\bar{z}_{k}}\left(1+z_{k}\bar{z}_{k},z_{k}+\bar{z}_{k},-i(z_{k}-\bar{z}_{k}),1-z_{k}\bar{z}_{k}\right) 
	\end{aligned}
\end{equation}
注意到:
\begin{equation}
	\varepsilon_{+}^{\mu}(q)=\frac{1}{\sqrt{2}\omega}\partial_{z}\left[\left(1+z\bar{z}\right)q^{\mu}\right],\quad\varepsilon_{-}^{\mu}(q)=\frac{1}{\sqrt{2}\omega}\partial_{\bar{z}}\left[\left(1+z\bar{z}\right)q^{\mu}\right]
\end{equation}
带进去一通暴力计算,软定理被我们写成了:
\begin{equation}
\lim_{\omega\to0^+}\left[\omega\left\langle\mathrm{out}\right|a_+^\mathrm{out}(\omega\hat{x})\mathcal{S}\left|\mathrm{in}\right\rangle\right]\\=\frac{e}{\sqrt{2}}\left(1+z\bar{z}\right)\left[\sum_{k\in\mathrm{out}}\frac{Q_k}{z-z_k}-\sum_{k\in\mathrm{in}}\frac{Q_k}{z-z_k}\right]\left\langle\mathrm{out}\right|\mathcal{S}\left|\mathrm{in}\right\rangle
\end{equation}
根据${\mathcal{CPT}}$不变性:\sn{?}
\begin{equation}
\langle\text{out}|a_+^\text{out}(\vec{q})\mathcal{S}|\text{in}\rangle=-\langle\mathrm{out}|\mathcal{S}a_{-}^{\mathrm{in}\dagger}(\vec{q})|\mathrm{in}\rangle
\end{equation}
结合上面两个式子就证明了Ward恒等式和软定理是一回事,或者说通过Ward恒等式,软定理和 Large gauge symmetry 是一回事。这其实反过来说明了我们找到的 Large gauge symmetry 是non-trivial的,确实得看作是一个渐近对称性。这种non-trivial从记忆效应也能体现出来,后面会详细讨论。

回到引力这边,这边的证明思路上差不多,但是有很多比较微妙的点。首先就是守恒荷的量子化,虽然也可以按照QED那边的方法一样做,但是技术细节上微妙许多。不过我们可以猜测它们应当有如下形式:
\begin{equation}
	\boxed{
		\left[Q_f^+,\cdots\right]=i\delta f,\quad \left[Q_f^-,\cdots\right]=i\delta Y
	}
\end{equation}
至于$\delta_f$和$\delta_Y$的形式前面已经算过了,而且它们都是和哈密顿量对易的,这也是利用Ward恒等式的基础。第一个式子证明相对简单,可以看文献\cite{He:2014laa},第二个等式要复杂许多,首先是需要下面几个式子\cite{Ashtekar1987AsymptoticQ,Ashtekar:1978zz,Ashtekar:1981bq,PhysRevLett.46.573}:
\begin{equation}
	\begin{aligned}
		\begin{bmatrix}N_{\bar{z}\bar{z}}(u,z,\bar{z}),C_{ww}(u',w,\bar{w})\end{bmatrix}&=16\pi Gi\gamma_{z\bar{z}}\delta^2(z-w)\delta(u-u')\\
		\left[Q_{S}^{+},C_{zz}\right]&=-iuD_{z}^{3}Y^{z},\\\left[Q_{H}^{+},C_{zz}\right]&=\frac{iu}{2}D\cdot YN_{zz}+iY\cdot DC_{zz}-\frac{i}{2}D\cdot YC_{zz}+2iD_{z}Y^{z}C_{zz}
	\end{aligned}
\end{equation}
而且Superrotation对应的相空间也更加复杂\cite{Strominger:2016wns}。这些都只能先argue一下,细说太麻烦。同样也可以给出模式展开,这个时候$g^{\mu\nu}=\eta^{\mu\nu}+\kappa h^{\mu\nu}$,$h^{\mu\nu}$满足linearized Einstein equation:
\begin{equation}
	\partial_{\sigma}\partial_{\nu}h^{\sigma}{}_{\mu}+\partial_{\sigma}\partial_{\mu}h^{\sigma}{}_{\nu}-\partial_{\mu}\partial_{\nu}h-\Box h_{\mu\nu}=0
\end{equation}
Mode expansion为:
\begin{equation}
	{h}_{\mu\nu}(x)=\kappa\sum_{\alpha\in\pm}\int\frac{d^3k}{(2\pi)^3}\frac{1}{2k^0}\left[\epsilon_{\mu\nu}^{\alpha*}a_{\alpha}e^{ik\cdot x}+\epsilon_{\mu\nu}^{\alpha}a_{\alpha}^{\dagger}e^{-ik\cdot x}\right]
\end{equation}
若$h=0$,内积定义为:
\begin{equation}
	(h,h')=-i\int d\Sigma^\rho[h^{\mu\nu}(\nabla_\rho {h'}_{\mu\nu}^*-2\nabla_\mu {h'}_{\rho\nu}^*)-(h\leftrightarrow {h'}^*)]
\end{equation}
同样也可以用鞍点近似去求度规里的那些参数的模式展开,然后去证明软引力子定理,详细的计算移步至文献\cite{Kapec:2014opa,He:2014laa}。

\subsection{Asymptotic analysis on QED}
这一节的主要目的是讲一下历史进程,文献最早是通过类似于BMS的渐近分析得到守恒荷这些\cite{Strominger:2013lka,He:2014cra}。分析渐近对称性的第一步就是指定边界条件,从而找到允许的对称性。由于$T_{uu}$是能动张量$T_{00}$和$T_{0i}$的线性组合,所以它现在代表能流,由于天球半径按照$r^2$增大,所以为了让能量有限大,$T_{uu}\sim\mathcal{O}\left(\frac{1}{r^2}\right)$,根据Noether定理:
\begin{equation}
	T^{\mu\nu}=\frac{1}{16\pi}F^{\alpha\beta}F_{\alpha\beta}\eta^{\mu\nu}-\frac{1}{4\pi}F^{\mu\rho}\partial^{\nu}A_{\rho}
\end{equation}
进行对称化操作得到:\sn{这样子做最大的好处是现在不显含$A$,与规范无关了}
\begin{equation}
	T_{S}^{\mu\nu}\equiv T^{\mu\nu}-\frac{1}{4\pi}\partial_{\rho}\left(F^{\rho\mu}A^{\nu}\right)=\frac{1}{16\pi}F^{\alpha\beta}F_{\alpha\beta}\eta^{\mu\nu}-\frac{1}{4\pi}F^{\mu\rho}{F^{\nu}}_{\rho}
\end{equation}
改到Bondi坐标下:
\begin{equation}
	T_{uu}\sim F_{uz}F_{u\bar{z}}\frac{\gamma^{z\bar{z}}}{r^2}+\ldots 
\end{equation}
这个式子意味着$F_{uz}\sim\mathcal{O}(1)$,再根据$F_{ru},F_{rz}$是电磁场分量,所以渐近行为均为$\mathcal{O}(1/r^2)$。下面的一组$A$的渐近行为选取就满足这几个条件:\sn{这一节我们并不取定规范}\sn{这个条件充分但不必要,但是根据文献\cite{Campiglia:2016hvg,Conde:2016csj,Conde:2016rom}的讨论,选别的理论会变得不是很自然。}
\begin{equation}
	A_z\sim\mathcal{O}(1),\quad A_r\sim\mathcal{O}\left(\frac{1}{r^2}\right),\quad A_u\sim\mathcal{O}\left(\frac{1}{r}\right)
\end{equation}
而满足这一边界条件的规范变换只能是:
\begin{equation}
	\varepsilon=\varepsilon(z,\bar{z})+\mathcal{O}\left(\frac1r\right)
\end{equation}
无穷远(天球)处不归零,就是前面说的Large gauge transformations。类似于引力散射问题的分析,QED这边散射问题为了well-define,也必须把对径认同的要求看作是散射问题定义的一部分。
\section{Massive QED}

\section{Further Progress}
\subsection{Subleading order}
\subsection{Non-Abelian gauge theory}
\subsection{Higher dimensions}



