\part{Conformal transformations}
% 计数器清零,每个part都要引用,除了part1
\setcounter{theorem}{0}
\setcounter{definition}{0}
\setcounter{lemma}{0}
\setcounter{sidenote}{1}

\section{Pull back, push forward \& Lie derivative}
考虑光滑映射$\phi:\mathcal{M}\to\mathcal{N}$,可以定义拉回映射为:
\begin{margintable}\footnotesize 
	\begin{tabularx}{\marginparwidth}{|X}
		$C^{\infty}(\mathcal{M})$表示$\mathcal{M}$上的光滑标量场\\
		$\mathscr{X}(\mathcal{M})$表示$\mathcal{M}$上某点处的切矢空间,相应的$\mathscr{X}^*(\mathcal{M})$表示余切丛空间\\
		$\mathscr{T}_\mathcal{M}(k,l)$表示$\mathcal{M}$上的$(k,l)$型张量场
	\end{tabularx}
\end{margintable}
\begin{definition}[pull back]
	$\phi^*: C^{\infty}(\mathcal{N})\to C^{\infty}(\mathcal{M}),f\mapsto \phi^*f$
	其中$\phi^*f\equiv f\circ\phi$,也即$(\phi^*f)|_p=f|_{\phi(p)}$,这个定义可以自然延拓到$\phi^*:\mathscr{T}_\mathcal{N}(0,l)\to\mathscr{T}_\mathcal{M}(0,l)$,其中:
	\begin{equation*}
		(\phi^*T)_{a_1\cdots a_l}|_p(v_1)^{a_1}\cdots (v_l)^{a_l}\equiv
		T_{a_1\cdots a_l}|_{\phi(p)}(\phi_*v_1)^{a_1}\cdots (\phi_*v_l)^{a_l}
	\end{equation*}
	对于$\forall p\in\mathcal{M},v_1,\ldots,v_l\in\mathscr{X}_p(\mathcal{M})$恒成立。
\end{definition}
类似的可以定义推前映射概念:
\begin{definition}[push forward]
	$\phi_*: \mathscr{X}_p(\mathcal{M})\to \mathscr{X}_{\phi(p)}(\mathcal{N}),X^a\mapsto (\phi_*X)^a$
	其中
	
	\begin{equation*}
	\eqnmarkbox[blue]{node1}{(\phi_*X)}(f)=\eqnmarkbox[red]{node2}{X}(\phi^*f),\quad\forall f\in C^\infty(\mathcal{N})	
	\end{equation*}
	\annotate[yshift=-0.5em]{below,left,label below}{node1}{at $\phi(p)$}
	\annotate[yshift=-0.5em]{below,label below}{node2}{at $p$}
	
	同样也可以进行延拓$\phi_*:\mathscr{T}_\mathcal{M}(k,0)\to\mathscr{T}_\mathcal{N}(k,0)$,其中
	\[
		(\phi_*T)^{a_1\cdots a_k}|_{q}(w^1)_{a_1}\cdots (w^k)_{a_l}\equiv
		T^{a_1\cdots a_k}|_{\phi^{-1}(q)}(\phi^*w_1)_{a_1}\cdots (\phi^*v_l)_{a_l}
	\]
	对于$\forall q\in\mathcal{N},w_1,\ldots,w_l\in\mathscr{X}^*_p(\mathcal{N})$恒成立。
\end{definition}

如果$\phi$是一个微分同胚,那可以进一步延拓到$\mathscr{T}_\mathcal{M}(k,0)\leftrightarrow\mathscr{T}_\mathcal{N}(k,0)$之间的推前和拉回映射。
\begin{definition}
	以$(1,1)$型张量的推前映射为例:
	\[
		(\phi_*T)^a_b|_qw_av^b\equiv T^a_b|_{\phi^{-1}(q)}(\phi^*w)_a(\phi^*v)^b
	\]
	对于任意的$q\in\mathcal{N},w_a\in\mathscr{X}_q^*(\mathcal{N}),v^b\in\mathscr{X}_q(\mathcal{N})$成立,其中$(\phi^*v)^b$理解为$(\phi^{-1}_*v)^b$。其它类型张量,以及拉回映射可类似定义,而且$\phi^*=\phi^{-1}_*$
\end{definition}

\begin{remark}
	现在我们有必要澄清一下关于映射的主动和被动观点。首先注意到微分同胚$\phi$其实很自然地定义了一个$\mathcal{M}$坐标变换$x\mapsto x^\prime$,其中$x$是$\mathcal{M}$上老坐标,$y$是$\mathcal{N}$上坐标,则:$$x^\prime(p)\equiv y(\phi(p))$$反过来,坐标变换也可以确定一个微分同胚映射。这让我们可以用两种方法去看待这个微分同胚:
	\begin{itemize}
		\item \textbf{主动观点}:老老实实看作是$p\in\mathcal{M}\mapsto \phi(p)\in\mathcal{N}$,然后在$\mathcal{N}$上确定了一个新的张量场,由原先的张量场“认同”后得来,也就是$T|p\mapsto\phi_*T|_{\phi(p)}$。
		\item \textbf{被动观点}:还是在原先的$\mathcal{M}$,点和张量也没有变换,而是现在在新的坐标系$\{x^\mu\}$下考虑问题。
	\end{itemize}
	这两种观点是等价的,关键就在于下面的这个等式:
	
	\begin{equation}\label{eq:equiv}
		\eqnmarkbox[blue]{node1}{(\phi_*T)}\eqnmarkbox[red]{node2}{{{}^{\mu_1\cdots\mu_k}}_{\nu_1\cdots\nu_l}}\eqnmarkbox[yellow]{node3}{\bigg|_{\phi(p)}}=\eqnmarkbox[blue]{node4}{T}\eqnmarkbox[red]{node5}{{{}^{\prime\mu_1\cdots\mu_k}}_{\nu_1\cdots\nu_l}}\eqnmarkbox[yellow]{node6}{\bigg|_{p}}
	\end{equation}
	\annotate[yshift=-1em]{below,left,label below}{node1}{New tensor}
	\annotate[yshift=0.7em]{left}{node2}{in Old coordinate $\{y^\mu\}$}
	\annotate[yshift=-0.5em]{below,label below}{node3}{at New point}
	\annotate[yshift=0.7em]{left}{node4}{Old tensor}
	\annotate[yshift=-1em]{below,label below}{node5}{in New coordinate $\{x^{\prime\mu}\}$}
	\annotate[yshift=1em]{}{node6}{at Old point}
	\vspace{0em}\\
	
	更多关于等价性的论述见梁灿彬\cite{lcb}第四章相关内容,后面会直接作为结论直接进行引述。
\end{remark}

\section{Killing \& Conformal Killing vector field}
下面我们考虑$\mathcal{M}=\mathcal{N},\{x^\mu\}=\{y^\mu\}$。对于矢量场$\xi^a$,其积分曲线诱导了一个微分同胚(点沿着积分曲线流动),在被动观点下看就是诱导了一个无穷小坐标变换$x^\mu\mapsto x^\mu+\xi^\mu t$,其中$t\to0$,在主动观点下看就是诱导了流形上点的变换和张量的变换,但是坐标系仍然不变。首先给出对$\xi$方向的李导数的定义:
\begin{definition}[李导数]
	李导数$\mathscr{L}_\xi$定义为
	\[\mathscr{L}_{\xi}T^{{a_1\cdots a_k}}_{b_1\cdots b_l}\equiv\lim_{t\to0}\frac{1}{t}\left(\phi_t^*{T^{a_1\cdots a_k}}_{b_1\cdots b_l}-{T^{a_1\cdots a_k}}_{b_1\cdots b_l}\right)\]
	可以利用下面的式子计算其在某一坐标系下分量:
	\begin{equation}
		\begin{aligned}
			\mathscr{L}_{\xi}{T^{\mu_1\cdots \mu_k}}_{\nu_1\cdots\nu_l}=&\xi^\lambda\nabla_\lambda{T^{\mu_1\cdots \mu_k}}_{\nu_1\cdots\nu_l}-{T^{\lambda\cdots \mu_k}}_{\nu_1\cdots\nu_l}\nabla_\lambda\xi^{\mu_1}-\cdots-{T^{\mu_1\cdots \lambda}}_{\nu_1\cdots\nu_l}\nabla_\lambda\xi^{\mu_k}\\
			&+{T^{\mu_1\cdots \mu_k}}_{\lambda\cdots\nu_l}\nabla_{\nu_1}\xi^{\lambda}+\cdots+{T^{\mu_1\cdots \mu_k}}_{\nu_1\cdots\lambda}\nabla_{\nu_l}\xi^{\lambda}
		\end{aligned}		
	\end{equation}
\end{definition}
比如度规的李导数:\sn{这里我们用到了度规平行移动的性质,$\forall X\in \mathscr{X}(\mathcal{M}),\nabla_X g=0$}
\begin{equation}
	\mathscr{L}_\mathbf{\xi} g_{\mu\nu}=\nabla_\mu\xi_\nu+\nabla_\mu\xi_\nu	
\end{equation}
\begin{definition}[Killing]
	矢量场$\xi^a$诱导单参数微分同胚群$\phi_t:\mathcal{M}\to\mathcal{M}$,如果其诱导的度规变换满足:
	\begin{equation}
		\phi^*g_{ab}=\Omega^2g_{ab},\quad \forall p\in\mathcal{M}
	\end{equation}
	其中$\Omega^2\in C^\infty(\mathcal{M})$且正定。我们就称向量场为\textbf{共形Killing向量场},对应的微分同胚称为\textbf{共形映射(变换)}。从李导数的观点来看就是要求:
	\begin{equation}
		\mathscr{L}_\xi g_{\mu\nu}=\omega(t)g_{\mu\nu}
	\end{equation}
	其中$\omega(t)\in C^\infty(\mathcal{M})$,有关系$\Omega^2=1+\omega(t)t+\mathcal{O}(t^2)$。特殊的,如果$\Omega^2=1$也即$\omega(t)=0$,那我们就称$\xi^a$为\textbf{Killing向量场},对应的微分同胚为\textbf{等度规映射}。
\end{definition}
前面的Lorentz变换其实就是在找在Minkowski时空内由Killing场诱导的变换,这要求:
\[\mathscr{L}_\xi \eta_{\mu\nu}\nabla_\mu\xi_\nu+\nabla_\mu\xi_\nu=\partial_\mu\xi_\nu+\partial_\nu\xi_\mu=0\]
称为\textbf{Killing方程}。这与前面直接从坐标变换导出的式子是一致的\sn{前面的式子实际上是把无穷小因子$t$吸收进了$\xi$中}。实际上,前面用坐标变换那一套就是在玩被动观点,可以证明,$\xi^a$是(共形)Killing场的充要条件是其生成的坐标变换使得:\sn{注意张量分量作为坐标的函数在何处取值,以及偏导数在何处取值,只要想清楚这些函数的自变量是什么、方程两边各自在哪个坐标系,以及方程作为张量等式都是在流形上同一点取值即可。}
\begin{equation}
	g^\prime_{\mu\nu}(x^\prime)=\frac{\partial x^\sigma}{\partial x^{\prime\mu}}(x^\prime)\frac{\partial x^\sigma}{\partial x^{\prime\nu}}(x^\prime)g_{\sigma\rho}(x)=\Omega^2(x)g_{\mu\nu}(x)
\end{equation}
在场论中我们更多使用坐标变换的被动观点来看问题。这个等式的证明关键就是使用\ref{eq:equiv}。

根据前面的论证,对于四维闵氏时空,Killing方程共有$10=\frac{4\times(4+1)}{2}$个独立解,也就是说存在$10$个独立的Killing场。实际上可以证明,对于任意$n$维时空,其最多具有$\frac{n(n+1)}{2}$个独立的Killing场,然而闵氏时空的$Poincar\'e$变换正好取到不等式上界。由于Killing场诱导的是等度规映射,所以也成为时空的对称性,根据上面的分析,平直闵氏时空具有最大的对称性。
\begin{remark}
	在GR的语境下提到共形变换更多的其实是指Wald invariant,它的定义和共形变换很像,但是不等价。Wald invariant的定义不需要微分同胚,或者说我们直接取同胚为$\mathrm{id}_\mathcal{M}$,这样流形上的点、张量和坐标系都不变,但是我们直接把流形上的度规结构改变,变成:
	\[\tilde{g}_{ab}=\Omega^2g_{ab}\]
	还要求物理不变,这就是Wald invariant,定义与共形变换非常类似,但是不能混淆两者,两者之间的微妙区别会体现在弦论中共形反常的消去上。\cite{Blumenhagen}
\end{remark}

度规在流形上定义了长度$\|v\|\equiv g(v,v)$,相应的可以定义角度为:
\[\cos\theta\equiv\frac{g(v,w)}{\|v\|\cdot\|w\|}\]
显然,共形变换是不改变两曲线交点处切矢之间角度的变换,也常被称为\textbf{保角变换}。
\section{Conformal transformations in $d>2$}
	现在假设背景时空是平直时空\footnote{后面的推导对于Minkowski时空和Euclide时空都适用},利用共形Killing方程,共形Killing场满足:
	\begin{equation}\label{11.6}
		\partial_\mu\xi_\nu+\partial_\nu\xi_\mu = \omega(x)g_{\mu\nu}
	\end{equation}
	式子两边同时取迹得到:
	\begin{equation}\label{11.7}
		\omega(x)=\frac{2}{d}\partial^\mu\xi_\mu
	\end{equation}
	\ref{11.6}两边同时微分$\partial_\rho$得到:
	\begin{equation}
		2\partial_\rho\partial_{{\{\mu}}\xi_{{\nu\}}}=\partial_\rho\omega g_{\mu\nu}
	\end{equation}
	上式对三个指标进行轮换,每次轮换改变符号然后相加得到:
	\begin{equation}\label{11.9}
		-\partial_\rho \omega g_{\mu\nu}+\partial_\mu \omega g_{\nu\mu}=2\partial_\mu\partial\nu \xi_\rho
	\end{equation}
	再次与$g^{\mu\nu}$缩并得到:
	\begin{equation}\label{11.10}
		\partial^\u\partial_\mu\xi_\rho=\frac{2-d}{2}\partial_\rho\omega
	\end{equation}
	\ref{11.6}作用上$\partial^\rho\partial_\rho$,再由\ref{11.10}得到:
	\begin{equation}\label{11.11}
		\left[\partial^\rho\partial_\rho g_{\mu\nu}+(d-2)\partial_\mu\partial_\nu\right]\omega(x)
	\end{equation}
	上式两边求迹得到:
	\begin{equation}
		(d-1)\partial^\mu\partial_\mu \omega(x)=0
	\end{equation}
	$d=1$,上式恒成立,也就是说任意变换都是共形变换,这是由于一维无法定义角度导致的,$d\geq 2$时,满足拉普拉斯方程:
	\begin{equation}
		\square^2 \omega(x)=0
	\end{equation}
	$d>2$则根据\ref{11.11}还进一步要求:
	\begin{equation}
		\partial_\mu\partial_\nu\omega(x)=0
	\end{equation}
	这说明$\omega$形式上只能为:
	\begin{equation}
		\omega(x)=A+B_\mu x^\mu
	\end{equation}
	代入 ref{11.10}得到
	\begin{equation}
		\partial_{\mu} \partial_{\nu} \xi_{\rho}=\frac{1}{2}\left(-B_{\rho} g_{\mu \nu}+B_{\mu} g_{\nu \rho}+B_{\nu} g_{\rho \mu}\right)
	\end{equation}
	右边是常向量。因此, $\xi_\mu $是 $x^\mu $的二次函数,可展开成
	\begin{equation}
		\xi_{\mu}(x)=a_{\mu}+b_{\mu \nu} x^{\nu}+c_{\mu \nu \rho} x^{\nu} x^{\rho}
	\end{equation}
	
	这里, $a_\mu,b_{\mu\nu},c_{\mu\nu\rho}$ 是常数, $c_{\mu\nu\rho} $关于后两指标对称: $c_{\mu \nu \rho}=c_{\mu \rho \nu} $。将上式代入 \ref{11.7} ,得到
	\begin{equation}
		\omega(x)=\frac{2}{d}\left(b^{\mu}{}_{\mu}+2 c^{\mu}{}_{\mu \rho} x^{\rho}\right)
	\end{equation}
	因此, $\omega $的展开式 (1.30) 中的系数 $A,B $同 $b,c $的关系是
	\begin{equation}
		A=\frac{2}{d} b^{\mu}{}_{\mu}, \quad B_{\mu}=\frac{4}{d} c^{\nu}{}_{\nu \mu}
	\end{equation}
	那么 $A,B$ 由 $a,b,c $确定了, (1.32) 代入 (1.31) 和 (1.18) ,可进一步限制 $b,c$ 的形式。事实上,代入后得到
	\begin{align} &2 c_{\rho \mu \nu}=\frac{1}{2}\left(-B_{\rho} g_{\mu \nu}+B_{\mu} g_{\nu \rho}+B_{\nu} g_{\rho \mu}\right)\\ &b_{\mu \nu}+b_{\nu \mu}+2\left(c_{\mu \nu \rho}+c_{\nu \mu \rho}\right) x^{\rho}=\left(A+B_{\rho} x^{\rho}\right) g_{\mu \nu} \end{align}
	于是
	\begin{align} &b_{\mu \nu}+b_{\nu \mu}=A g_{\mu \nu}\\ &c_{\mu \nu \rho}=\frac{1}{4}\left(-B_{\mu} g_{\nu \rho}+B_{\nu} g_{\rho \mu}+B_{\rho} g_{\mu \nu}\right) \end{align}
	
	由此,知道共形因子后就可以写出共形变换。最终得到无穷小变换可分成以下几类:\sn{SCT: Special Conformal Transformation}
	\begin{equation}
		\boxed{
		\begin{aligned}
				&\text{Translation}&&x^{\prime \mu}=x^{\mu}-a^{\mu}\\
				&\text{Rotation}&&x^{\prime \mu}=x^{\mu}-b^{A \mu \nu} x_{\nu}, \quad b^{A \mu \nu}=-b^{A \nu \mu}\\
				&\text{Dilation}&&x^{\prime \mu}=x^{\mu}-\frac{A}{2} x^{\mu}\\
				&\text{SCT}&&x^{\prime \mu}=x^{\mu}-\frac{1}{4}\left(-B^{\mu} x^{2}+2 x^{\mu} B^{\nu} x_{\nu}\right)
		\end{aligned}
	}
	\end{equation}
	
	对应的无穷小变换的生成元可表示为
	\begin{equation}
		\boxed{
		\begin{aligned} &P_{\mu}=-i \partial_{\mu} \\ &M_{\mu \nu}=i\left(x_{\mu} \partial_{\nu}-x_{\nu} \partial_{\mu}\right) \\\ &D=-i x^{\mu} \partial_{\mu} \\ &K_{\mu}=-i\left(2 x_{\mu} x^{\nu} \partial_{\nu}-x^{2} \partial_{\mu}\right)  
		\end{aligned}
	}
	\end{equation}
	它们之间的对易关系为:
	\begin{equation}
		\boxed{
				\begin{aligned} 
					&[P_\mu,P_\nu]=0,\quad[P_\rho,M_{\mu\nu}]=i(\eta_{\mu\rho}P_\nu-\eta_{\nu\rho} P_\mu)\\
					&[M_{\mu\nu},M_{\rho\sigma}=i\left(\eta_{\mu\rho}M_{\nu\sigma}-\eta_{\mu\sigma}M_{\nu\rho}-\eta_{\nu\rho}M_{\nu\rho}+\eta_{\nu\sigma}M_{\mu\rho}\right)]\\
					&\left[D, P_{\mu}\right]=i P_{\mu}\\ &\left[D, K_{\mu}\right]=-i K_{\mu} \\ &\left[K_{\mu}, P_{\nu}\right]=2 i\left(g_{\mu \nu} D-M_{\mu \nu}\right) \\ &\left[K_{\rho}, M_{\mu \nu}\right]=i\left(g_{\rho \mu} K_{\nu}-g_{\rho \nu} K_{\mu}\right) \\ &\left[P_{\rho}, M_{\mu \nu}\right]=i\left(g_{\rho \mu} P_{\nu}-g_{\rho \nu} P_{\mu}\right) 
			\end{aligned}
		
		}
	\end{equation}
	
	可以看到其中一部分就是Poincar\'e代数\sn{$g_{\mu\nu}=\delta_{\mu\nu}$时为Euclide代数},这也说明了等度规变换是共形变换的特殊情况。这个Lie代数称为\textbf{$ d $维共形代数}。
	
	重定义生成元 $J_{ab} (a,b=-1,0,\cdots,d )$:
	\begin{align} &J_{\mu \nu}=M_{\mu \nu}\\ &J_{-1 \mu}=\frac{1}{2}\left(P_{\mu}-K_{\mu}\right) \\ &J_{-10}=D \\ &J_{0 \mu}=\frac{1}{2}\left(P_{\mu}+K_{\mu}\right)  \end{align}
	从共形代数的对易关系,可以得到 $J_{ab}$ 满足对易关系
	\begin{equation}
		\left[J_{a b}, J_{c d}\right]=i\left(g_{a d} J_{b c}+g_{b c} J_{a d}-g_{a c} J_{b d}-g_{b d} J_{a c}\right)
	\end{equation}
	当共形变换是Euclide空间中的变换时, $g_{ab} $是号差为 $(-,+, \cdots,+)$ 的Minkowski度规\sn{否则有两个负号}。也就是说$d$维(Euclide空间)中的共形代数,同构于Lorentz代数$\mathfrak{so}(d+1,1) $。
	
	有限共形变换可有由无穷小共形变换的叠加构成,形式为:
	\begin{equation}
		\boxed{
			\begin{aligned}
				&\text{Translation}&&x^{\prime \mu}=x^{\mu}-a^{\mu}\\
				&\text{Rotation(Boost)}&&x^{\prime \mu}=\Lambda_{\nu}^{\mu} x^{\nu}\\
				&\text{Dilation}&&x^{\prime \mu}=\alpha x^{\mu}\\
				&\text{SCT}&&x^{\prime \mu}=\frac{x^{\mu}-b^{\mu} x^{2}}{1-2 b\cdot x+b^{2} x^{2}}
			\end{aligned}
		}
	\end{equation}
	其中最后一个变换是由平移和反演变换的组合得到的:
		\begin{equation}
		x^{\mu} \rightarrow x^{\prime \mu}=\frac{x^{\mu}}{x^{2}} \rightarrow x^{\prime \prime \mu}=x^{\prime \mu}-b^{\mu} \rightarrow x^{\prime \prime \prime \mu}=\frac{x^{\prime \prime \mu}}{x^{\prime \prime 2}}
	\end{equation}
	而反演变换是离散的,所以SCT并不能用无穷小变换生成。根据前面的讨论,这些变化构成$SO(d+1,1)$群。
\section{Conformal transformations in $d=2$}
	上一节利用李导数看待问题,也就是所谓主动观点,现在用坐标变换的观点来进行推演。
	
	在二维情况下,使用复数作为参数非常方便,定义复变量:\sn{导数这样定义是为了$\partial_z z=\partial_{\bar z}\bar z=1$}
	\[z=x^1+ix^2,\quad \partial_z\equiv\frac{1}{2}(\partial_1-i\partial_2),\quad \partial_{\bar z}\equiv\frac{1}{2}(\partial_1+i\partial_2)\]
	Euclide空间中度规可以写成:
	\begin{equation}
		g_z = dzd\bar z
	\end{equation}
	现在考虑坐标变换$z\mapsto z^\prime(z,\bar z)$,度规相应变为:
	\begin{equation}
		\begin{aligned}
			g_z\mapsto g^'_z&=dz^\prime d{\bar z}^\prime\\
			&=\left(\frac{\partial z^\prime}{\partial z}dz+\frac{\partial z^\prime}{\partial\bar z}d\bar z\right)
			\left(\frac{\partial \bar z^\prime}{\partial z}dz+\frac{\partial \bar z^\prime}{\partial\bar z}d\bar z\right)\\
			&=\frac{\partial z^\prime}{\partial z}\frac{\partial \bar z^\prime}{\partial z}dz^2+\frac{\partial z^\prime}{\partial \bar z}\frac{\partial \bar z^\prime}{\partial \bar z}{dz^\prime}^2+\left(\frac{\partial \bar z^\prime}{\partial  z}\frac{\partial  z^\prime}{\partial \bar z}+\frac{\partial z^\prime}{\partial z}\frac{\partial \bar z^\prime}{\partial \bar z}\right)dzd\bar z
		\end{aligned}
	\end{equation}
	这导致了下面的Cauchy-Riemann条件:
	\begin{equation}
		\frac{\partial z^\prime}{\partial z}=0\quad \text{or}\quad \frac{\partial z^\prime}{\partial \bar z}=0
	\end{equation}
	而且前面的推导只要求度规形式为$\Omega(z,\bar z)dzd\bar z$。也就是说,变换$z^\prime$为全纯\sn{数学人叫法,物理人喜欢称为解析函数}或者反全纯函数,而反全纯函数会将右手系变为左手系,后面主要考虑全纯情况,反全纯情况只需要全部取复共轭即可。
	
	考虑局部的无穷小共形变换,变换形式可以写成:
	\[z\mapsto z+\xi(z),\quad \bar z\mapsto \bar z+\bar\xi(\bar z)\]
	其中$\xi$可以进行Laurent展开为:
	\[\xi(z)=\sum_{n=-\infty}^{+\infty}\xi_n z^{n+1} ,\quad \bar\xi(\bar z)=\sum_{n=-\infty}^{+\infty}\bar\xi_n \bar z^{n+1}\]
	对应于$\xi_n,\bar\xi_n$的生成元为$L_n,\bar L_n$,显然这些生成元的个数是无限个!生成了一个无限维李代数Witt代数:
	\begin{equation}
		L_n=-z^{n+1}\partial_z,\quad \bar L_n=-\bar w_{n+1}\partial^{\bar w}
	\end{equation}
	\begin{equation}
		[L_m,L_n]=(m-n)L_{m+n},\quad [\bar L_m,\bar L_n]=(m-n)\bar L_{m+n},\quad [L_n,\bar L_m]=0
	\end{equation}
	二维共形场论在局域变换下具有无穷多的对称性!但是全局的共形变换要求$z^\prime(z)$没有极点且只有一个非简并零点(否则变换不是单的),这限制了$z^\prime(z)$的形式为:
	\begin{equation}
		z^\prime(z)=a z+b,\quad a,b\in\mathbb{C}, a\neq 0
	\end{equation}
	这里$a\neq 0$是为了让$z\mapsto z^\prime$为$\mathbb{C}\to \mathbb{C}$的满映射。
	
	$d$维共性代数与$\mathfrak{so}(d+1,1)$之间的同构可以显式的构造出来\cite{Rychkov:2016iqz,Hugh}。首先注意到由于Lorentz变换保$ds^2$,所以对于顶点在原点处的光锥,Lorentz变换是在光锥上的同胚。那我们可以把光锥看作是嵌入\sn{Embedding formalism}在$\mathbb{R}^{d+1,1}$中的子流形,这实际上是将$\mathbb{R}^{d}$通过下面的方式嵌入到$\mathbb{R}^{d+1,1}$中:
	\begin{equation}
		q^{\mu}=(1+x^2,2x^A,1-x^2),\quad x^A\in\mathbb{R}^d
	\end{equation}
	这样,Lorentz变换作用在光锥上可以看作是$\mathbb{R}^d$上的一个变换:
	\begin{equation}
		q^\mu\mapsto q^{\prime\mu}=\frac{(\Lambda q)^\mu}{(\Lambda q)^+}
	\end{equation}
	其中分母是为了将$q^+\equiv\frac{1}{2}\left(q^0+q^{d+1}\right)=1$归一化。而且,可以验证这个变换还是一个共形变换!特别地,当$d=2$时,定义$w=x^1+ix^2$,这个同构变成:
	\begin{equation}
		q^\mu(w,\bar w)=(1+w\bar w,w+\bar w,i(\bar w-w),1-w\bar w)
	\end{equation}
\section{Conformal transformations on the Riemann sphere}
本节的关键在于下面的式子:
\begin{equation}
	\mathcal{S}^2\cong\mathbb{C}\cup\{z=\infty\}
\end{equation}
这实际上是在对复平面进行\textbf{一点紧化},导致的球面我们称为Riemann球面,这个同胚可以利用球极投影显式构造出来:
\begin{equation}
	(x_1,x_2,x_3)\mapsto(x_1^{\prime},x_2^{\prime},0):\begin{cases}
		x_1^{\prime}=\frac{rx_1}{r+x_3}\\
		x_2^{\prime}=\frac{rx_2}{r+x_3}
	\end{cases}
\end{equation}
利用复坐标可以写为:
\begin{equation}
	(x_1,x_2,x_3)\mapsto z\equiv\frac{x_1^{\prime}+ix_2^{\prime}}{r}=\frac{x_1+ix_2}{r+x_3}\iff z=\mathrm{e}^{i\phi}\tan\frac{\theta}{2}
\end{equation}
这里我们是按照南极为极点进行投影,投影到赤道平面,以北极为投影点只需要把上式中的$r+x_3$替换为$r-x_3$。更一般的,我们还可以给出高维的球极投影:
\begin{equation}
	\mathcal{S}^{n+1}\cong{\mathbb{E}^{n}\cup\{\infty\}}
	\quad(x_1, x_2, \cdots, x_n,0) \mapsto\frac{(2x_1, 2x_2, \cdots, 2x_n, |x|^2 - 1)}{|x|^2 + 1}
\end{equation}

将$\mathcal{S}^2$作为$\mathbb{R}^3$的子流形,不难在复坐标下写出对应度规为:
\begin{equation}
	g_z=\frac{4r^2}{(1+z\bar z)^2}dzd\bar z
\end{equation}
根据之前的分析,$\mathcal{S}^2$上的共形变换由全纯或反全纯函数诱导,后者将左右手系互换。复平面一点紧化为球面之后,全局共形变换函数必然有一个极点,负责映射到$\{\infty\}$,所以\sn{下面只对全纯进行讨论,反全纯只需取复共轭}
\[z'(z)=\frac{\operatorname{Poly}(z)}{\operatorname{Poly}(z)}\]
依旧根据变换的单射性质,要求分子分母都必须只能线性依赖于$z$,而且由于满性,分子分母零点不能相同,所以这要求全局共形变换形式只能为:
\begin{equation}
	z'(z)=\frac{a z+b}{c z+d},\quad \begin{vmatrix}
		a&b \\
		c&d
	\end{vmatrix}=1
\end{equation}
注意到这里对行列式进行了归一化选取。一般谈及二维共形变换我们都是在加入无穷远点后进行讨论,在数学上这种变换称为\textbf{M\"obius变换}。由于整体相差一个负号代表的是同一个变换,所以变换群为$SL(2,\mathbb{C})/\mathbb{Z}_2$,将反全纯部分一并考虑进来后扩充为$SL(2,\mathbb{C})$。

Witt代数中$n=\{-1,0,1\}$的部分张成了$\mathfrak{sl}(2,\mathbb{C})$子代数,也就是那些全局共形变换的生成元。它们实际上与Lorentz群生成元可以直接由下式联系:
\begin{equation}
	\begin{aligned}
		L_0&=\frac{1}{2}\left(J_3-iK_3\right),
		&L_{-1}&=\frac{1}{2}\left(-J_1+iJ_2+iK_1+K_2\right),
		&L_{1}&=\frac{1}{2}\left(J_1+iJ_2-iK_1+K_2\right),\\
		L_0&=\frac{1}{2}\left(-J_3-iK_3\right),
		&L_{-1}&=\frac{1}{2}\left(J_1+iJ_2+iK_1-K_2\right),
		&L_{1}&=\frac{1}{2}\left(-J_1+iJ_2-iK_1-K_2\right).
	\end{aligned}
\end{equation}
其中
\[J_i=\frac{1}{2}\epsilon_{ijk}M^{jk},\quad K_i=M_{i0},\quad i\in\{1,2,3\}\]