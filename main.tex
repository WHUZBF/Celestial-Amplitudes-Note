\documentclass[10pt]{ctexart}

\usepackage{NotesTeXV3,lipsum}
%\usepackage{showframe}
\usepackage{amsmath}
\usepackage{amsfonts,amssymb}
\usepackage{mathrsfs}
\usepackage{tikz}
\usepackage{annotate-equations} % 公式标注美化
%%%%%%%%%%%%%%%%%%%%%%%%%%%%%%%%%%%%%%%%%%%%%%%%%%%%%%%%%%%%%%%%%%%%%%%%
\usepackage{xfp} % higher precision (16 digits?)
\usepackage[outline]{contour} % glow around text
\usetikzlibrary{decorations.markings,decorations.pathmorphing}
\usetikzlibrary{angles,quotes} % for pic (angle labels)
\usetikzlibrary{arrows.meta} % for arrow size
\contourlength{1.4pt}
% tikz绘图一系列设定
\tikzset{>=latex} % for LaTeX arrow head
\colorlet{myred}{red!80!black}
\colorlet{myblue}{blue!80!black}
\colorlet{mygreen}{green!80!black}
\colorlet{mydarkred}{red!50!black}
\colorlet{mydarkblue}{blue!50!black}
\colorlet{mylightblue}{mydarkblue!6}
\colorlet{mypurple}{blue!40!red!80!black}
\colorlet{mydarkpurple}{blue!40!red!50!black}
\colorlet{mylightpurple}{mydarkpurple!80!red!6}
\colorlet{myorange}{orange!40!yellow!95!black}
\tikzstyle{cone}=[mydarkblue,line width=0.2,top color=blue!60!black!30,
bottom color=blue!60!black!50!red!30,shading angle=60,fill opacity=0.9]
\tikzstyle{cone back}=[mydarkblue,line width=0.1,dash pattern=on 1pt off 1pt]
\tikzstyle{world line}=[myblue!60,line width=0.4]
\tikzstyle{world line t}=[mypurple!60,line width=0.4]
\tikzstyle{particle}=[mygreen,line width=0.5]
\tikzstyle{photon}=[-{Latex[length=4,width=3]},myorange,line width=0.4,decorate,
decoration={snake,amplitude=0.9,segment length=4,post length=3.8}]
\tikzstyle{singularity}=[myred,line width=0.6,decorate,
decoration={zigzag,amplitude=2,segment length=6.17}]
\tikzset{declare function={%
		penrose(\x,\c)  = {\fpeval{2/pi*atan( (sqrt((1+tan(\x)^2)^2+4*\c*\c*tan(\x)^2)-1-tan(\x)^2) /(2*\c*tan(\x)^2) )}};%
		penroseu(\x,\t) = {\fpeval{atan(\x+\t)/pi+atan(\x-\t)/pi}};%
		penrosev(\x,\t) = {\fpeval{atan(\x+\t)/pi-atan(\x-\t)/pi}};%
		kruskal(\x,\c)  = {\fpeval{asin( \c*sin(2*\x) )*2/pi}};% Penrose coordinates for Kruskal
}}
\def\tick#1#2{\draw[thick] (#1) ++ (#2:0.04) --++ (#2-180:0.08)}
\def\Nsamples{40} % number samples in plot

% LIGHTCONE
\def\R{0.08} % size lightcone
\def\e{0.08} % vertical scale
\def\ang{45} % angle light cone
\def\angb{acos(sqrt(\e)*sin(\ang))} % angle ellipse center to point of tangency
\def\a{\R*sin(\ang)*sqrt(1-\e*sin(\ang)^2)/(1-\e*sin(\ang)^2)} % vertical radius
\def\b{\R*sqrt(\e)*sin(\ang)*cos(\ang)/(1-\e*sin(\ang)^2)} % horizontal radius
\def\coneback#1{ % light cone part to be drawn behind world lines
	\draw[cone back] % dashed line back
	(#1)++(-45:\R) arc({90-\angb}:{90+\angb}:{\a} and {\b});
	\draw[cone,shading angle=-60] % top edge & inside
	(#1)++(0,{\R*cos(\ang)/(1-\e*sin(\ang)^2)}) ellipse({\a} and {\b});
}
\def\conefront#1{ % light cone part to be drawn over world lines
	\draw[cone] % light cone outside
	(#1) --++ (45:\R) arc({\angb-90}:{-90-\angb}:{\a} and {\b})
	--++ (-45:2*\R) arc({90-\angb}:{-270+\angb}:{\a} and {\b}) -- cycle;
}
%%%%%%%%%%%%%%%%%%%%%%%%%%%%%%%%%%%%%%%%%%%%%%%%%%%%%%%%%%%%%%%%%%%%%%%%

\bibliographystyle{JHEP}

\begin{document}
	\title{{天球振幅科研笔记}\\{\normalsize{\itshape Celestial Amplitudes}}}
	\author{Bufan Zheng}
	\affiliation{
	Undergraduate Student at the Wuhan University\\
%	\href{https://adhumunt.github.io/}{Website}\\
%	\href{https://inspirehep.net/authors/1669979}{Inspire-HEP}\\
%	\href{https://www.linkedin.com/in/aditya-dhumuntarao-80a0b8112/}{LinkedIn}\\
	\href{https://github.com/WHUZBF}{GitHub}\\
	}
	\emailAdd{whuzbf@qq.com}
	\maketitle
	\newpage
	\pagestyle{fancynotes}
	\part{A quick review on SR \& GR}
\section{Basics conceptions in SR}\label{sec:1}
我们生活的空间是一个四维局部平坦的\textbf{Lorentz流形},也就是一个四维微分流形配备一个非正定、非退化的度规$g=g_{\mu\nu}dx^\mu\otimes dx^\nu$,这是一个$(0,2)$张量。局部平坦意思是说任何一点处都可以选取一个坐标系\sn{这个坐标系称为\textbf{局部惯性系},由于可以从指数映射结合测地线来构造这个惯性系,所以也称为\textbf{自由下落参考系}}使得$g_{\mu\nu}=\eta_{\mu\nu},\nabla_\rho g_{\mu\nu}$\sn{这里符号约定为$\eta_{\mu\nu}=(-,+,+,+)$}。而Lorentz体现在$\eta$有一个指标是负数,而且根据惯性定理,无论你选取什么坐标系将度规对角化,最终负数的个数都是一样的,这样一来我们便可以严格的区分时间和空间\sn{如果这里把$\eta$中的$-1$变成$+1$,我们称为\textbf{Riemann流形}}。

参数化流形后,时空上的每一点(事件)都将对应一个坐标$x^\mu$,时空中的曲线(世界线)可以参数化为$x^\mu(\tau)$,其可以看作是由矢量场$X=X^\mu\partial_\mu=\frac{x^\mu(\tau)}{d\tau}\partial_\mu$诱导的。考虑世界线上相邻的两点,我们可以定义线长为
\[dl=\sqrt{|g(X,X)|}=\int d\tau\sqrt{g_{\mu\nu}\frac{x^\mu}{d\tau}\frac{x^\nu}{d\tau}}\]
很多时候也把线长记为$ds^2=g_{\mu\nu}dx^\mu dx^\nu$但是在微分几何的严格意义下,$dx^\mu$是对偶矢量,并不是初等微积分中的微分,所以这个式子只能理解为$ds^2=g_{\mu\nu}dx^\mu\otimes dx^\nu$,也就是说$ds^2$只是张量$g$的另一个叫法而已!后面我们为了方便可能牺牲严谨性,使用$ds^2$表示世界线长。

GR中最重要的基本假设便是在坐标变换下物理定律是不变的,这说明作用量必须是标量,几乎唯一确定了真空引力场作用量为:
\begin{margintable}\footnotesize 
	\begin{tabularx}{\marginparwidth}{|X}
		$\Lambda$是宇宙学常数\\
		$g\equiv\det g_{\mu\nu}$\\
		$R$是Ricci标量\\
		取自然单位制$c=\hbar=1$
	\end{tabularx}
\end{margintable}
\begin{equation}
	S_{\mathrm{EH}}=\frac{1}{16\pi G}\int \mathrm{d}^4x\sqrt{-g} (R-2\Lambda)
\end{equation}
由于度规是张量,所以其在坐标变换下分量变换为:\sn{注意两边对应的自变量,因为张量都是关于流形上点的场,所以这里坐标变换后流形上某点对应的坐标也变了。}
\[\tilde{g}_{\mu\nu}(\tilde{x})=\frac{\partial \tilde{x}^\rho}{\partial x^\mu}\frac{\partial \tilde{x}^\sigma}{\partial x^\nu}g_{\rho\sigma}(x)\]
在SR中我们仅研究平直的时空,或者说只研究惯性系之间的变换,这些惯性系中的变换满足$\tilde{\eta}=\eta$,可以一般的记为:
\[\tilde{x}^\mu=\Lambda^{\mu}_{\ \nu}x^\nu+a^\mu\]
那么$\Lambda$满足:
\begin{equation}
	\Lambda^{\operatorname{T}}\eta\Lambda=\eta
\end{equation}
本笔记主要考虑SR时空。
\section{Poincar\'e group and Lorentz group}
显然所有的$\Lambda$构成了一个群,称之为\textbf{Lorentz群}:
\begin{equation}
	L\equiv O(3,1)\equiv\left\{\Lambda\in M(4,\mathbb{R})|	\Lambda^{\operatorname{T}}\eta\Lambda=\eta\right\}
\end{equation}
而所有的保度规变换还要加入$a^mu$,构成\textbf{Poincar\'e群}:$O(3,1)\ltimes \mathbb{R}$,群乘法为:
\begin{equation}
	(\Lambda,a)\cdot(\Lambda',a')=(\Lambda\cdot\Lambda',a+\Lambda\cdot a')
\end{equation}
利用Poincar\'e群的不等价不可约表示可以对基本粒子进行分类,见本部分末尾。后面我们将主要关注Lorentz群。
不难验证$\det \Lambda=\pm 1$,它将Lorentz群分成两个分支,其中$\det \Lambda=1$的部分含有单位元,构成子群\textbf{正规Lorentz群}。记为$SO(3,1)$或$L_+$

另外$(\Lambda^0_{\ 0})^2\geq 1$也将Lorentz群分成两个分支,其中$\Lambda^0_{\ 0}\geq1$的部分含有单位元,构成子群\textbf{正时Lorentz群}。记为$O(3,1)^\uparrow$或$L^\uparrow$。

最后$L^\uparrow_+\equiv L^\uparrow\cap L_+$也构成了$L$的一个子群。这些子群之间可以用时间反演和空间反演算符相联系:
\begin{equation}
	\mathcal{T}=\begin{pmatrix}
		-1&  &  & \\
		& 1 &  & \\
		&  & 1 & \\
		&  &  &1
	\end{pmatrix},\quad\quad\mathcal{P}=\begin{pmatrix}
	1&  &  & \\
	& -1 &  & \\
	&  & -1 & \\
	&  &  &-1
	\end{pmatrix}
\end{equation}
显然$L_+=\{L_+^\uparrow,\mathcal{T}\},L^\uparrow=\{L_+^\uparrow,\mathcal{P}\},L=\{L_+^\uparrow,\mathcal{T},\mathcal{P}\}$

\subsection{Poincar\'e algebra}
现在考虑群的局部性质,考虑无穷小坐标变换$x^\mu\mapsto x^\mu+\xi^\mu$,保度规条件为:
\begin{equation}\label{lie_eta}
	\mathscr{L}_\xi\eta_{\mu\nu}=\partial_\mu\xi_\nu+\partial_\mu\xi_\nu=\tilde{\eta}_{\mu\nu}(\tilde x)-\eta_{\mu\nu}(x)=0
\end{equation}
\begin{remark}
	这里$\mathscr{L}_\xi$表示李导数。李导数的严格定义涉及到拉回映射,首先对于某个矢量场$\xi$,其给出了一个单参微分同胚群$\phi_t$(可以想象矢量场生成了一簇积分曲线,流形上每一点根据积分曲线移动便生成了单参微分同胚群),利用它我们可以构造拉回映射$\phi_t^*$,从而构造李导数为:
	\[\mathscr{L}_{\xi}T^{{a_1\cdots a_k}}_{b_1\cdots b_l}\equiv\lim_{t\to0}\frac{1}{t}\left(\phi_t^*{T^{a_1\cdots a_k}}_{b_1\cdots b_l}-{T^{a_1\cdots a_k}}_{b_1\cdots b_l}\right)\]
	这里$a,b$是抽象指标,只表示张量类型,不表示分量。李导数可以利用下式计算其分量:
	\begin{equation}
		\begin{aligned}
			\mathscr{L}_{\xi}{T^{\mu_1\cdots \mu_k}}_{\nu_1\cdots\nu_l}=&\xi^\lambda\nabla_\lambda{T^{\mu_1\cdots \mu_k}}_{\nu_1\cdots\nu_l}-{T^{\lambda\cdots \mu_k}}_{\nu_1\cdots\nu_l}\nabla_\lambda\xi^{\mu_1}-\cdots-{T^{\mu_1\cdots \lambda}}_{\nu_1\cdots\nu_l}\nabla_\lambda\xi^{\mu_k}\\
			&+{T^{\mu_1\cdots \mu_k}}_{\lambda\cdots\nu_l}\nabla_{\nu_1}\xi^{\lambda}+\cdots+{T^{\mu_1\cdots \mu_k}}_{\nu_1\cdots\lambda}\nabla_{\nu_l}\xi^{\lambda}
		\end{aligned}		
	\end{equation}
	利用上式便可证明\ref{lie_eta}的正确性。
\end{remark}
$\xi^\mu$可以用$\omega_{\ \nu}^\mu$和$b^\mu$两个无穷小参数标记:
\[\xi^\mu={\omega^\mu}_\nu x^\nu+b^\mu,\quad \omega_{\mu\nu}=-\omega_{\nu\mu}\]
平移生成元为:
\[P_\mu=-i\partial_\mu\Rightarrow T(b)=\exp(-i b^\mu P_\mu)\]
boost和转动生成元为:
\[M_{\mu\nu}=i\left(x_\mu\partial_\nu-x_\nu\partial_\mu\right)\Rightarrow \Lambda(\omega)=\exp(-\frac{i}{2} \omega^{\mu\nu} M_{\mu\nu})\]
生成共同构成Poincar\'e代数:
\begin{equation}
	\begin{aligned}
		&[P_\mu,P_\nu]=0,\quad[P_\rho,M_{\mu\nu}]=i(\eta_{\mu\rho}P_\nu-\eta_{\nu\rho} P_\mu)\\
		&[M_{\mu\nu},M_{\rho\sigma}=i\left(\eta_{\mu\rho}M_{\nu\sigma}-\eta_{\mu\sigma}M_{\nu\rho}-\eta_{\nu\rho}M_{\nu\rho}+\eta_{\nu\sigma}M_{\mu\rho}\right)]
	\end{aligned}
\end{equation}

对于更一般的度规,如果矢量场满足\sn{这里我们用到了度规平行移动的性质,$\forall X\in \mathscr{X}(\mathcal{M}),\nabla_X g=0$}
\[\mathscr{L}_\mathbf{\xi} g_{\mu\nu}=\nabla_\mu\xi_\nu+\nabla_\mu\xi_\nu=0\]
那我们称$\mathbf{\xi}$为Killing矢量场,如果
\[\mathscr{L}_\mathbf{\xi} g=0\]
则称为共形Killing矢量场,后面共形变换部分会对这一点做深入讨论。
\begin{remark}
	注意,当我们用矢量场来看待这个问题的时候,我们其实是在用所谓坐标变换的主动观点,也就是坐标变换诱导矢量场,矢量场再诱导流形同胚从而导致张量变换,因为在被动的坐标变换观点下,张量场不会有任何改变。更多关于主动被动观点的讨论见\cite{lcb}。
\end{remark}
\section{Boost and Rapidity}
我们把Lorentz群记为$O(3,1)$强烈暗示了其与$O(n)$群的类似性,其实Lorentz变换完全可以看作是四维时空中的旋转。三维空间旋转有三个自由度,分别是绕着$x,y,z$轴的旋转,这些轴都是由另外两个轴张成的平面所确定的,总数为$C_3^2=3$。那么对于高维空间旋转,比如四维空间可以推广为共$C_4^2=6$个自由度。其中有3个是单纯的$\mathbb{R}^3$中的旋转,还有三个是混合了时间轴的旋转,也就是初等SR介绍中的两个相对速度为$v$的惯性系之间的变换,称为\textbf{boost}。比如$x$方向上的boost就可以显式表达出来为:
\begin{margintable}\footnotesize 
	\begin{tabularx}{\marginparwidth}{|X}
		$\gamma(v)=1/\sqrt{(1-\beta(v)^2)}$\\
		$\beta(v)=v/c$
	\end{tabularx}
\end{margintable}
\begin{equation}
	\Lambda(v)=\begin{pmatrix}
		\gamma(v)&-\gamma(v)\beta(v)  &0  & 0\\
		-\gamma(v)\beta(v) & \gamma(v) &0  &0 \\
		0&0  & 1 &0 \\
		0&  0&  0&1
	\end{pmatrix}
\end{equation}

$\Lambda(v)$是一个boost,或者说四维转动,类比三维转动$R_x(\theta)R_x(\phi)R_x(\theta+\phi)$,很容易想到$\Lambda(v)\Lambda(w)\overset{?}{=}\Lambda(v+w)$,即绕着某个轴的转动是一个单参数Abel子群。但实际上以$v$为参数并不能看出这一点。可以定义\sn{$\chi\in(-\infty,+\infty)$,所以是非紧致的Lie群}:
\begin{equation}
	\chi(v)\equiv\operatorname{arctanh}(\frac{v}{c})
\end{equation}
称为\textbf{rapidity}这样便有$\Lambda(\chi_2)\Lambda(\chi_1)=\Lambda(\chi_2+\chi_1)$。

rapidity其实有非常明显的物理含义,回忆一下速度的定义:
\[\text{velocity}=\frac{\text{displacement}}{\text{time}}\]
由于右边的分式分子分母都是依赖于参考系的,所以如果B相对于A运动\sn{方便起见假设沿$x$轴作直线运动,但不要求匀速},实际上可以对于B定义三种不同的速度。
\begin{definition}
	$v=\frac{dx}{dt}$,这里$x,t$都是在A的参考系下测得的。
\end{definition}
\begin{definition}
	$u=\frac{dx}{d\tau}$,这里$x$是在A的参考系下测的,$\tau$是B的固有时。
\end{definition}
这个定义是关于参考系协变的,也就是通常的$\mathfrak{4}$\mbox{-}速度的定义。
\begin{definition}
	$\tilde(v)=\frac{dx_B}{d\tau}$,分子分母都是B自己测得的。
\end{definition}
但是这个定义有个很大的问题,B自己测量时间没问题,但是B自己测量自己的位移始终是0,所以上面这个定义必须重新审视。首先我们看如何对应B的加速度,假设B在固有时$\tau$的时刻相对于地面系的速度\sn{第一个定义}为$v$,这个时候考虑一个与B速度相同的瞬时惯性系,也就是说过一段时间$d\tau$之后B相对于这个瞬时惯性系会有个速度$d\tilde{v}$,加速度也便定义为$d\tilde{v}/d\tau$。假设这段时间内,相对于地面系B速度增加了$dv$,那么根据速度叠加法则:
\begin{equation}
	\frac{v+d\tilde{v}}{1+vd\tilde{v}/c^2}=v+dv\Rightarrow d\tilde{v}\left(1+v^2/c^2\right)dv
\end{equation}
现在对加速度进行积分:
\begin{equation}
		I(\tau)=\int^\tau_0 d\tau\frac{d\tilde{v}}{d\tau}=\int^{\tilde{v}}_0d\tilde{v}=\int^{v}_0\frac{dv}{1+v^2/c^2}=c\cdot\operatorname{arctanh}\frac{v(\tau)}{c}=c\cdot\chi(v(\tau))
\end{equation}
所以在这个速度的定义下,自然导出了rapidity,如果取自然单位制$c=1$,那么两者完全一致。
\section{Connected components of Lorentz group}
对于任何正规且正时的Lorentz群中的元素都可以做标准分解:
\begin{theorem}
	$\forall \Lambda\in L_+^\uparrow, \exists R_1,R_2\in SO(3)$,使得
	\begin{equation}\label{eq:4.1}
		\Lambda = R_1 L_x(\chi) R_2
	\end{equation}
\end{theorem}
而Lorentz群只需要再加上$\mathcal{T}$和$\mathcal{P}$即可,而且这种分解对于$d>2$维时空都是适用的。从物理上很好理解,一般的Lorentz变换无非就是绕着任意轴的boost,我们都可以先进行转动,将boost方向转为$x$轴,进行boost之后再转回原来的方向。

任何一个Lie群实际上都是一个微分流形,而连通性这个概念正是建立在此之上从拓扑观点来看的。作为一个流形,$G$不一定是连通的,可以有很多个连通分支,其中只有含有$e$的连通分支才能构成子群,我们记为$G_e$,有下面的定理成立:
\begin{theorem}
	$G_e\lhd G$且所有连通分支构成商群$G/G_e$.
\end{theorem}
这里不做严格证明,下面我们将此定理用于Lorentz群。根据\ref{eq:4.1},由于其中的每个因子都与$e$道路连通,所以$L^\uparrow_+\subseteq L_e$,而$L$中的其它群元为了与$e$相连,必须通过离散变换$\mathcal{T},\mathcal{P}$,所以实际上$L^\uparrow_+=L_e$,那么连通分支构成商群$L/L^\uparrow_+\cong \mathbb{Z}_2\times\mathbb{Z}_2$,这里同构成立是因为商群实际上由$\{\mathcal{T},\mathcal{P}\}$生成。
\begin{margintable}\footnotesize 
	\begin{tabularx}{\marginparwidth}{|X}
		后面我们都用$\cong$表示同构,$\simeq$表示同态
	\end{tabularx}
\end{margintable}

可见Lorentz群确实包含了4个连通分支,可以根据$\Lambda^0_{\ 0}$以及$\det \Lambda$的符号进行分类。
\section{Poincar\'e group and particles}
这一部分是最为精妙的部分,我们将会利用Poincar\'e群的不可约表示对场和粒子进行分类,本节论述主要参考Weinberg\cite{Weinberg}和董无极\cite{WKT}


to be continue\ldots


	\part{Lorentz group \& special linear groups}
% 计数器清零,每个part都要引用,除了part1
\setcounter{theorem}{0}
\setcounter{definition}{0}
\setcounter{lemma}{0}
\setcounter{sidenote}{1}

本部分我们的目的是建立一系列同构关系,基本思路就是先找到一个同态,然后利用同态核定理构造同构
\begin{theorem}[同态核定理]
	如果$f:G\to H$是一个群同态,那么有$\ker(f)\lhd G$且$G/\ker(f)\cong\Im(f)$
\end{theorem}
而且本部分会充分利用李群是微分流形这一拓扑性质进行说明,很多证明没有数学上的严谨,重在直观的感性认知。
\section{$SO(3)\cong SU(2)/\mathbb{Z}_2$}
$SU(2)$是所有行列式为1的酉矩阵构成的群,其群元素可以一般的写为:
\[
U=
\begin{pmatrix}
	\alpha & \beta\\
	-\beta^* &\alpha^*
\end{pmatrix},\quad |\alpha|^2+|\beta|^2=1
\]
这意味着描述一个群元需要四个实参,且满足$\alpha_1^2+\alpha_2^2+\beta_1^2+\beta_2^2=1$,显然这意味着$SU(2)$的拓扑结构为$\mathcal{S}^3$。另外一个需要用到的概念是\textbf{群中心},也就是与所有群元都对易的群元\sn{注意与Casimir算符的区别}。不难看出$SU(2)$的群中心构成子群$\mathbb{Z}^2$:
\[
\mathbb{I}_2=\begin{pmatrix}
	1 & 0\\
	0 &1
\end{pmatrix},\quad
-\mathbb{I}_2=\begin{pmatrix}
	-1 &0\\
	0 &\-1
\end{pmatrix}
\]
\begin{theorem}
$
		SO(3)\cong SU(2)/\mathbb{Z}_2
$
\end{theorem}

\begin{proof}
	首先考虑$2\times2$无迹厄米矩阵构成的线性空间$\mathbb{V}$,显然其中任意一个元素都可以写为$X=x^i\sigma_i$,其中$\sigma_i$是三个Pauli矩阵,这实际上建立起了同构$\mathbb{V}\cong\mathbb{R}^3$。$\mathfrak{su}(2)$李代数作为线性空间显然是与$\mathbb{V}$同构的,而$SU(2)$在李代数\footnote{所谓代数,就是线性空间赋予一个封闭的乘法结构}上诱导出一个所谓伴随表示:
	\[
	\mathcal{U}(U)X=UXU^\dagger=Ux^i\sigma_i U^\dagger\equiv f(U)^i_j x^j\sigma_i
	\]
	容易验证这个表示是保范数$\left\|x\right\|$的,那么$f(U)\in O(3)$,也就是说我们建立了一个同态:
	\[f:SU(2)\to O(3),U\mapsto f(U)\]
	
	
	\setlength\parindent{2em}为了利用同态核定理,首先计算$\Im(f)$。由于$f$是个连续映射,而且$SU(2)$单连通\footnote{$\mathcal{S}^n$的基本群在$n=1$时为自由群$\mathbb{Z}$,其它时候都为平凡群},所以$f(U)$也应当包含在$O(3)$的单连通子群中,即$\Im(f)\subseteq SO(3)$。反过来$SO(3)\subseteq \Im(f)$也成立,可以看作是Euler角和Caylay\mbox{-}Klein参数之间的对应,所以$\Im(f)=SO(3)$
	
	
	\setlength\parindent{2em}现在来计算$\ker(f)$,$f(U)=\mathbb{I}_{3\times 3}$说明$\forall X\in \mathbb{V}$,都有$UXU^\dagger=X$,也就是说要找的$U$与任意$X$对易,那么其也与任意的$e^{i X}$对易,然而$\text{Lie Group}=\mathrm{e}^{\textbf{Lie Algebra}}$,所以$U$就是群中心的元素,所以$\ker(f)\cong\mathbb{Z}_2$,根据同态核定理便有$SO(3)\cong SU(2)/\mathbb{Z}_2$。
\end{proof}
\begin{remark}
	这其实说明了$SU(2)$是$SO(3)$群的双覆盖,$SO(3)$群对应流形是对径认同实心球$\mathbb{R}\mathbf{P}^2\times [0,\pi]$,从流形上也能感受一下。最后我们显式给出这个同态:
	\[
	f\left[\begin{pmatrix}
		a& b\\
		c &f
	\end{pmatrix}\right]=\begin{pmatrix}
	\Re(\bar a d+\bar b c) & \Im(a\bar  d-b\bar c)  &\Re(\bar a c-\bar bd) \\
	\Im(\bar a d+\bar b c) & \Re(a\bar  d-b\bar c)   & \Im(\bar a c-\bar bd)\\
	\Re(a\bar b-c\bar d) & \Im(a\bar b-c\bar d) &\frac{1}{2}\left(|a|^2-|b|^2-|c|^2+|d|^2\right)
	\end{pmatrix}
	\]
	上式直接从$U,-U$对应同一个$SO(3)$中元素也可看出双覆盖性。
\end{remark}

\section{$SO(2,1)^\uparrow\cong SL(2,\mathbb{R})/\mathbb{Z}_2$}
\begin{lemma}[QR分解]
	任意复矩阵都可以分解为一个酉矩阵$Q$和一个上三角矩阵$R$的乘积,且$R$主对角元全为正数。如果这个矩阵是实矩阵,那么$Q$为正交矩阵。如矩阵可逆,则分解唯一。
\end{lemma}
\begin{lemma}
	$SL(2,\mathbb{R})$的拓扑结构为$\mathcal{S}$,基本群为$\mathbb{Z}$
\end{lemma}
\begin{proof}
	根据QR分解,以及$det S=1\neq 0$,任意$SL(2,\mathbb{R})$中的矩阵都可以唯一的分解为$S=QR$,而且要求$|Q|\cdot|R|=1$,而$R$主对角元全为正数以及$|Q|=\pm 1$实际上给出$Q\in SO(3)$且$R$的对角线上元素有$a\cdot b=1$且为正数的限制,而另一个元素不做限制。这其实就是在对$SL(2,\mathbb{R})$做直积分解,由于$SO(2)$对应的流形为$\mathcal{S}^1$,所以$SL(2,\mathbb{R})$对应的流形为$\mathcal{S}^1\times \mathbb{R}\times\mathbb{R}^+$,后两者基本群平凡,所以$SL(2,\mathbb{R})$对应的基本群为$\mathbb{Z}$。\footnote{这里用了乘积空间基本群为各自基本群的直积。}
\end{proof}
了解了$SL(2)$的拓扑性质后就可以开始证明本节的核心结论。
\begin{theorem}
	$SO(2,1)^\uparrow\cong SL(2,\mathbb{R})/\mathbb{Z}_2$
\end{theorem}
\begin{proof}
	与上一节同样,我们先构造$\mathfrak{sl}(2,\mathbb{C})$李代数,其由二维实无迹矩阵构成,生成元为:
	\[t_0=\begin{pmatrix}
		0 &1 \\
		-1&0
	\end{pmatrix},\quad t_1=\begin{pmatrix}
	0 &1 \\
	1&0
	\end{pmatrix},\quad t_2=\begin{pmatrix}
	1 &0 \\
	0&-1
	\end{pmatrix} \]
	张成的线性空间中任一元素可以表达为$X=x^\mu t_\mu$,显然$\det X=-\eta_{\mu\nu}x^\mu x^\nu=-x^2$,同样我们对$S\in SL(2,\mathbb{C})$构造伴随表示$X\mapsto SXS^{-1}$,其保事件间隔不变,所以诱导了一个同态:
	\[f:SL(2,\mathbb{R})\to O(2,1),S\mapsto f(S)\]
	其中
	\[S t_\mu x^\mu S^{-1}=t_\mu {f(S)^\mu}_\nu x^\nu,\forall x^\mu \in\mathbb{R}^3\iff S t_\mu  S^{-1}=t_\nu {f(S)^\nu}_\mu \]
	根据$f$连续,从拓扑上得知$\Im f\in SO(2,1)^\uparrow$,反过来,要论证任何$\Lambda \in SO(2,1)^\uparrow$都可以用$f(S)$表示,根据$\Lambda=R_1 L(\chi) R_2$,我们只需要找到$S_1,S_2,S(\chi)$使得
	\[f(S_1)=R_1,f(S_1)=R_2,f(S(\chi))=L(\chi)\]
	不难验证前两个等式只需要选取
	\[S=\begin{pmatrix}
		\cos\theta &\sin\theta \\
		-\sin\theta&\cos\theta
	\end{pmatrix}\in SO(2)\subseteq SL(2,\mathbb{R})\]
	并合适选取$\theta$参数即可,而后面一个只需选取:
	\[S(\chi)=\begin{pmatrix}
		e^{-\chi/2} &0 \\
		0&e^{\chi/2} 
	\end{pmatrix}\]
	最后证明$\ker f\cong \mathbb{Z}_2$的方法就和上一节一样了。
\end{proof}
\begin{remark}
	下面显式给出同态:
	\begin{equation}
		f\left[\begin{pmatrix}
			a &b \\
			c&d
		\end{pmatrix}\right]=\begin{pmatrix}
		\frac{1}{2}\left(a^2+b^2+c^2+d^2\right) &\frac{1}{2}\left(a^2-b^2+c^2-d^2\right) &-ab-cd\\
		\frac{1}{2}\left(a^2+b^2-c^2-d^2\right)&\frac{1}{2}\left(a^2-b^2-c^2-d^2\right)&-ab+cd\\
		-ac-bd&bd-ac&ad+bc
		\end{pmatrix}
	\end{equation}
\end{remark}
\section{$SO(3,1)^\uparrow\cong SL(2,\mathbb{C})/\mathbb{Z}_2$}
这一节的证明与上一节非常类似,证明细节会适当省略。还是首先关注一下$SL(2,\mathbb{C})$$SL(2,\mathbb{C})$的拓扑性质。
\begin{lemma}
	$SL(2,\mathbb{C})$单连通
\end{lemma}
\begin{proof}
	证明依旧是使用QR分解,现在$R$需要一个正实数和一个复数来描述,所以对应流形为$\mathbb{R}^+\times\mathbb{C}$,基本群平凡。而$Q\in SU(2)$对应流形为$\mathcal{S}^3$,基本群也平凡,所以$SL(2,\mathbb{C})$基本群平凡,即单连通。
\end{proof}
下面证明本节核心定理:
\begin{theorem}
	$SO(3,1)^\uparrow\cong SL(2,\mathbb{C})/\mathbb{Z}_2$
\end{theorem}
\begin{proof}
	证明完全仿造上一节,只是$t\to\tau$,$\tau_0=\mathbb{I}_{2\times 2}$,$\tau_1=-\sigma_1$\footnote{这里符号约定上比一般定义多了个负号,是为了后文处理天球符号更自洽。},$\tau_2=\sigma_2$,$\tau_3=\sigma_3$。后面的证明也是用伴随表示诱导同态后计算$\Im f$,这里根据$SL(2,\mathbb{C})\supseteq SU(2)/\mathbb{Z}_2\cong SO(3)$可以得到$f(S)=R$,剩下的一个只用取:
	\[S(\chi)=\begin{pmatrix}
		\cosh\frac{\chi}{2} &\sinh\frac{\chi}{2} \\
		\sinh\frac{\chi}{2} &\cosh\frac{\chi}{2} 
	\end{pmatrix}\]
	最后计算$\ker f$也是同样的思路说明$\ker f\cong\mathbb{Z}_2$
\end{proof}
\begin{remark}
	下面显式给出同态:
	\begin{equation}\label{eq:8.1}
		\begin{aligned}
		&f\left[\begin{pmatrix}
			a &b \\
			c&d
		\end{pmatrix}\right]=\\
		&\begin{pmatrix}
			\frac{1}{2}\left(|a|^2+|b|^2+|c|^2+|d|^2\right) &-\Re(a\bar b +c\bar d) &\Im(a\bar b +c\bar d)&\frac{1}{2}\left(|a|^2-|b|^2+|c|^2-|d|^2\right)\\
			-\Re(\bar ac +\bar bd)&\Re(\bar ad +\bar bc)&-\Im(a\bar d- b\bar c)&-\Re(\bar ac -\bar bd)\\
			\Im(\bar ac +\bar bd)&-\Im(\bar ad +\bar bc)&\Re(a\bar d- b\bar c)&\Im(\bar ac -\bar bd)\\
			\frac{1}{2}\left(|a|^2+|b|^2-|c|^2-|d|^2\right)&-\Re(a\bar b -c\bar d)&\Im(a\bar b -c\bar d)&\frac{1}{2}\left(|a|^2-|b|^2-|c|^2+|d|^2\right)
		\end{pmatrix}
		\end{aligned}
	\end{equation}
\end{remark}
下面给出两个例子:
\begin{example}
	$z$轴旋转
	\begin{equation}
		\begin{pmatrix}
			1 & 0 &0  &0 \\
			0&\cos\theta  &-\sin\theta  & 0\\
			0&-\sin\theta  & \cos\theta &0 \\
			0&  0&  0&1
		\end{pmatrix}\sim\pm\begin{pmatrix}
		e^{-i\theta/2}&0\\
		0&e^{i\theta/2}
		\end{pmatrix}
	\end{equation}
	$z$方向boost
	\begin{equation}
		\begin{pmatrix}
			\cosh\chi& 0 &0  &\sinh\chi \\
			0&1  &0  & 0\\
			0&0 & 1 &0 \\
			\sinh\chi&  0&  0&\cosh\chi
		\end{pmatrix}\sim\pm\begin{pmatrix}
			e^{-\chi/2}&0\\
			0&e^{\chi/2}
		\end{pmatrix}
	\end{equation}
\end{example}
\section{Higher dimensions}
\begin{definition}[赋范可除代数]
	首先考虑$\mathbb{R}$或者$\mathbb{C}$上的线性空间我们可以赋予乘法结构将其提升为代数,如果除了$0$元其它元素都有逆元,我们称为\textbf{可除代数},进一步我们可以赋予范数,而且要求范数满足:
	\[\left\|xy\right\|=\left\|x\right\|\left\|x\right\|,\forall x,y\in V\]
	即\textbf{赋范可除代数}。
\end{definition}
\begin{theorem}[Hurwitz]
	任何赋范可除代数都同构于$\mathbb{R},\mathbb{C},\mathbb{H},\mathbb{O}$中的一种。其中$\mathbb{H}$是四元数,$\mathbb{O}$是八元数。
\end{theorem}
这是一个非常漂亮的结论,告诉我们为什么历史上发现复数之后寻找三元数必然是失败的,而哈密顿的四元数会成功。

不难猜测,对于更高维时空的Lorentz群,会对应四元数和八元数,实际上有:
\begin{equation}
	SO(5,1)^\uparrow\cong SL(2,\mathbb{H})/\mathbb{Z}_2\quad\quad
	SO(9,1)^\uparrow\cong SL(2,\mathbb{O})/\mathbb{Z}_2
\end{equation}
另外,八元数实际上构成的不是一个结合代数,所以$SL(2,\mathbb{O})$的存在并非显然的,这里并不深入讨论。但是实际上这些同构关系式蕴含着很深刻的物理,暗示着最小超对称理论只能构建在3,4,6,10维时空中。更有意思的是,可除代数与超弦理论之间有着非常深刻的联系。\cite{Baez1,Baez2}
\section*{Interlude: Project representation}
{\color{red}\hrule}
\hspace*{\fill} \\

下面对Lorentz群表示的本身做更加细致的考量,主要是为了引进一些必要的数学概念,最终结论主要就是将一些讨论更加数学严格化,不用过分在意。本节讨论主要参照Weinberg\cite{Weinberg}。

对于一个对称群$G$,在考虑量子力学关于这个群的对称性时我们其实是在考虑其表示对Hilbert空间的作用。群表示自然是个同态,所以我们想到去考虑:
\begin{equation}
	U(T)U(\bar T)=U(T\cdot \bar T)
\end{equation}
但是量子力学的希尔伯特空间实际上是一个射影空间,两个相差全局相位的量子态视作等价,所以我们实际上应该去考虑群的\textbf{射影表示}:
\begin{equation}
	U(T)U(\bar T)=e^{i\phi(T,\bar T)}U(T\cdot \bar T)	
\end{equation}
这里利用线性可以证明$\phi(T,\bar T)$与所作用的量子态本身无关,但前提条件是这两个量子态是可加的。比如具有半整数自旋和整数自旋的量子态就是不可加的,最多只能制备出这俩态的直积态。

但是射影表示用起来很麻烦,如果相位具有下面的特殊结构:
\[\phi(T,\bar T)=\alpha(T\bar T)-\alpha(T)-\alpha(\bar T)\]
那我们可以对群表示后的算符重定义:
\begin{equation}\label{redefine1}
	\tilde U(T)=U(T)\exp\left(i\alpha(T)\right)
\end{equation}
这样我们就可以继续考虑普通的表示而不是射影表示了。现在我们必须严格考虑一个群是否存在不能通过重定义消去的射影表示,我们称为\textbf{内禀射影表示}。

既然考虑的是李群,那我们可以把群元用$\theta^a$参数化,并定义:
\[T(\bar\theta)T(\theta)=T\left(f(\bar\theta,\theta)\right)\]
根据$f(0,\theta)=f(0,\theta)=\theta$,我们得到单位元附近群元表示的展开:
\begin{equation}\label{eq:9.3}
	U\left(T(\theta)\right)=1+i\theta^a t_a+\frac{1}{2}\theta^b\theta^c t_{bc}+\mathcal{O}(\theta^3)
\end{equation}
这里$t_a$就是常说的生成元,在这个符号约定下是厄米的,$t_{ab}$关于指标对称,表示更高阶的项。另外$f$有展开:
\begin{equation}\label{eq:9.4}
	f^a(\bar\theta,\theta)=\theta^a+\theta^a+{f^a}_{bc}\bar\theta^b\theta^c+\mathcal{O}(\theta^3)
\end{equation}
类似地,因为$\phi(T,1)=\phi(T,1)=0$,我们有展开:
\begin{equation}\label{eq:9.5}
	\phi\left(T(\theta),T(\bar\theta)\right)=f_{ab}\theta^a\bar\theta^b
\end{equation}
结合\ref{eq:9.3},\ref{eq:9.4}和\ref{eq:9.5}我们得到:
\begin{equation}
	t_{bc}=-t_bt_c-i{f^a}_{bc}t_a-if_{bc}
\end{equation}
再根据$t_{bc}$的对称性有:
\begin{equation}
	[t_b,t_c]=i\eqnmarkbox[blue]{node1}{\left({f^a}_{cb}-{f^a}_{bc}\right)}t_a+i\eqnmarkbox[red]{node2}{\left(f_{cb}-f_{bc}\right)}\cdot\mathbb{I}
\end{equation}
\annotate[yshift=1em]{left}{node1}{denoted by ${C^a}_{bc}$}
\annotate[yshift=-1em]{below,label below}{node2}{denoted by $C_{bc}$}

在相位不为0时,生成元的对易关系之间多了一项$iC_{bc}\cdots\mathbb{I}$,称为\textbf{中心荷}。根据Jacobi恒等式,中心荷要满足方程:
\begin{equation}
	\begin{aligned}
		{C^a}_{bc}{C^e}_{ad}+{C^a}_{cd}{C^e}_{ab}+{C^a}_{db}{C^e}_{ac}&=0\\
		{C^a}_{bc}{C}_{ad}+{C^a}_{cd}{C}_{ab}+{C^a}_{db}{C}_{ac}&=0
	\end{aligned}
\end{equation}
这是与李代数具体结构无关的约束,给出一类特解:
\[C_{ab}={C^e}_{ab}\phi_e,\quad\phi_e\in\mathbb{R}\]
这类解到底存不存在要看李代数具体结构,但如果说李代数恰好是这样的解,那么我们可以通过重定义生成元:
\begin{equation}\label{redefine2}
	\tilde t_a=t_a+\phi_a\Rightarrow [\tilde t_b,\tilde t_c]=i{C^a}_{bc}\tilde t_a
\end{equation}
来消除中心荷。这引出了李群是否存在内禀射影表示的定理。
\begin{theorem}
	如果李群满足下面两个条件:
	\begin{itemize}
		\item[$\bullet$] 可以类似\ref{redefine2}重定义生成元消去所有中心荷。
		\item[$\bullet$] 群的拓扑结构单连通。
	\end{itemize}
	那么我么总可以类似\ref{redefine1}一样令相位为。
\end{theorem}
证明比较复杂,我们重点看在Poincar\'e群上的应用,另外,这个定理告诉我们只有两种方式产生内禀投影表示,一种是代数的,一种是拓扑的。

\begin{theorem}[V.Bargmann\cite{V.Bargmann}]
	半单Lie代数都可以通过重定义生成元消去中心荷
\end{theorem}
很幸运,齐次Lorentz群,也就是$M_{\mu\nu}$张成的代数是半单的,但不幸的是Poincar\'e代数不是半单的,不过更加幸运的是中心荷依旧可以被消除。

前面我们说明Lorentz群的拓扑结构使用了$QR$分解,实际上,使用另一种稍微不同的分解方式——极分解——可以证明拓扑结构其实同胚于$\mathbb{R}^3\times\mathcal{S}^3\times\mathbb{Z}_2$,Poincar\'e群多出来的那一部分,也就是$\mathbb{R}^4$是平凡的,重点在于$\mathcal{S}^3\times\mathbb{Z}_2$,这其实是个双连通结构。也就是说基本群为$\mathbb{Z}_2$。直观但不严谨的说就是初始点固定,转两圈总共回到初始点两次的“双圈”可以连续收缩到一点,但是单圈被分成两种,一种能收缩到一点,另一种必须再重复自己以此才能收缩到一点。

这么来看Lorentz群必须得用射影表示,我们看一下这个射影表示的特点,核心在于双圈可以收缩到单位元,所以$1\to\Lambda\to\Lambda\bar\Lambda\to1$的路径走两次等于单位元:
\begin{equation}
	\left[U(\Lambda)U(\bar\Lambda)U^{-1}(\Lambda\bar\Lambda)\right]^2=1\Rightarrow U(\Lambda)U(\bar\Lambda)=\pm U(\Lambda\bar\Lambda)
\end{equation} 

这里的正负号完全可以解释成自旋!整数自旋取正号,半整数取负号。而完整的描述应该是这个射影表示加上所谓“超选择定则”。前面我们说过相位是不依赖于态的,但前提条件是这些态是可加的,所以我们可以认为整数和半整数自旋存在超选择定则,也就是说它们不可加,那么相位就依赖于作用的态的自旋,我们实验上也确实发现了这种不可加性。这就构成了整个Lorentz群的表示(同样的推理也可以扩张到Poincar\'e群,毕竟它们拓扑结构一致)。

但是这样还是比较繁琐,但是从数学上看似乎引进射影表示是必然的,那我们能否从物理上把超选择定则给去掉呢?其实,我们完全可以把大自然真正的对称群取为$SL(2,\mathbb{C})$而不是Lorentz群$SL(2,\mathbb{C})/\mathbb{Z}$,这个更大的对称群只有普通表示,射影表示里的正负号随着表示本身的不同就自然的冒出来了。就像是$SU(2)$,看作是对$SO(3)$对称性的扩张,奇数维表示是简并表示,会出现负号,而偶数维是忠实表示,就只留下正号了。当然这一做法有代价,那就是抛弃了超选择定则\sn{总之除了超选择定则,其它完全一样。},这样不同自旋态的不可加性就不能从Lorentz不变性直接导出,但并不用担心这一点,毕竟实验上从未制备出这样的叠加态。

所以,对于任何对称群,如果存在中心荷,可以干脆扩张这个李代数,把与一切生成元都对易的生成元包含进来,这样就不存在中心荷了\sn{伽利略群就存在质量$M$这个中心荷,我们可以进行扩张,见A.Zee\cite{A.Zee}},但是同样也会丢弃超选择定则;如果李群不是单连通的,我们可以将其表示成$C/H$,其中$C$单连通,称为$G$的通用覆盖群\sn{类似的论述可以在\cite{PFS}对应章节找到}。然后把对称群取为$C$而不是$G$,这样扩张群之后就可以不用担心射影表示问题,也不用引入超选择定则。
	\part{Conformal transformations}
\section{Pull back, push forward \& Lie derivative}
考虑光滑映射$\phi:\mathcal{M}\to\mathcal{N}$,可以定义拉回映射为:
\begin{margintable}\footnotesize 
	\begin{tabularx}{\marginparwidth}{|X}
		$C^{\infty}(\mathcal{M})$表示$\mathcal{M}$上的光滑标量场\\
		$\mathscr{X}(\mathcal{M})$表示$\mathcal{M}$上某点处的切矢空间,相应的$\mathscr{X}^*(\mathcal{M})$表示余切丛空间\\
		$\mathscr{T}_\mathcal{M}(k,l)$表示$\mathcal{M}$上的$(k,l)$型张量场
	\end{tabularx}
\end{margintable}
\begin{definition}[pull back]
	$\phi^*: C^{\infty}(\mathcal{N})\to C^{\infty}(\mathcal{M}),f\mapsto \phi^*f$
	其中$\phi^*f\equiv f\circ\phi$,也即$(\phi^*f)|_p=f|_{\phi(p)}$,这个定义可以自然延拓到$\phi^*:\mathscr{T}_\mathcal{N}(0,l)\to\mathscr{T}_\mathcal{M}(0,l)$,其中:
	\begin{equation*}
		(\phi^*T)_{a_1\cdots a_l}|_p(v_1)^{a_1}\cdots (v_l)^{a_l}\equiv
		T_{a_1\cdots a_l}|_{\phi(p)}(\phi_*v_1)^{a_1}\cdots (\phi_*v_l)^{a_l}
	\end{equation*}
	对于$\forall p\in\mathcal{M},v_1,\ldots,v_l\in\mathscr{X}_p(\mathcal{M})$恒成立。
\end{definition}
类似的可以定义推前映射概念:
\begin{definition}[push forward]
	$\phi_*: \mathscr{X}_p(\mathcal{M})\to \mathscr{X}_{\phi(p)}(\mathcal{N}),X^a\mapsto (\phi_*X)^a$
	其中
	
	\begin{equation*}
	\eqnmarkbox[blue]{node1}{(\phi_*X)}(f)=\eqnmarkbox[red]{node2}{X}(\phi^*f),\quad\forall f\in C^\infty(\mathcal{N})	
	\end{equation*}
	\annotate[yshift=-0.5em]{below,left,label below}{node1}{at $\phi(p)$}
	\annotate[yshift=-0.5em]{below,label below}{node2}{at $p$}
	
	同样也可以进行延拓$\phi_*:\mathscr{T}_\mathcal{M}(k,0)\to\mathscr{T}_\mathcal{N}(k,0)$,其中
	\[
		(\phi_*T)^{a_1\cdots a_k}|_{q}(w^1)_{a_1}\cdots (w^k)_{a_l}\equiv
		T^{a_1\cdots a_k}|_{\phi^{-1}(q)}(\phi^*w_1)_{a_1}\cdots (\phi^*v_l)_{a_l}
	\]
	对于$\forall q\in\mathcal{N},w_1,\ldots,w_l\in\mathscr{X}^*_p(\mathcal{N})$恒成立。
\end{definition}

如果$\phi$是一个微分同胚,那可以进一步延拓到$\mathscr{T}_\mathcal{M}(k,0)\leftrightarrow\mathscr{T}_\mathcal{N}(k,0)$之间的推前和拉回映射。
\begin{definition}
	以$(1,1)$型张量的推前映射为例:
	\[
		(\phi_*T)^a_b|_qw_av^b\equiv T^a_b|_{\phi^{-1}(q)}(\phi^*w)_a(\phi^*v)^b
	\]
	对于任意的$q\in\mathcal{N},w_a\in\mathscr{X}_q^*(\mathcal{N}),v^b\in\mathscr{X}_q(\mathcal{N})$成立,其中$(\phi^*v)^b$理解为$(\phi^{-1}_*v)^b$。其它类型张量,以及拉回映射可类似定义,而且$\phi^*=\phi^{-1}_*$
\end{definition}

\begin{remark}
	现在我们有必要澄清一下关于映射的主动和被动观点。首先注意到微分同胚$\phi$其实很自然地定义了一个$\mathcal{M}$坐标变换$x\mapsto x^\prime$,其中$x$是$\mathcal{M}$上老坐标,$y$是$\mathcal{N}$上坐标,则:$$x^\prime(p)\equiv y(\phi(p))$$反过来,坐标变换也可以确定一个微分同胚映射。这让我们可以用两种方法去看待这个微分同胚:
	\begin{itemize}
		\item \textbf{主动观点}:老老实实看作是$p\in\mathcal{M}\mapsto \phi(p)\in\mathcal{N}$,然后在$\mathcal{N}$上确定了一个新的张量场,由原先的张量场“认同”后得来,也就是$T|p\mapsto\phi_*T|_{\phi(p)}$。
		\item \textbf{被动观点}:还是在原先的$\mathcal{M}$,点和张量也没有变换,而是现在在新的坐标系$\{x^\mu\}$下考虑问题。
	\end{itemize}
	这两种观点是等价的,关键就在于下面的这个等式:
	
	\begin{equation}\label{eq:equiv}
		\eqnmarkbox[blue]{node1}{(\phi_*T)}\eqnmarkbox[red]{node2}{{{}^{\mu_1\cdots\mu_k}}_{\nu_1\cdots\nu_l}}\eqnmarkbox[yellow]{node3}{\bigg|_{\phi(p)}}=\eqnmarkbox[blue]{node4}{T}\eqnmarkbox[red]{node5}{{{}^{\prime\mu_1\cdots\mu_k}}_{\nu_1\cdots\nu_l}}\eqnmarkbox[yellow]{node6}{\bigg|_{p}}
	\end{equation}
	\annotate[yshift=-1em]{below,left,label below}{node1}{New tensor}
	\annotate[yshift=0.7em]{left}{node2}{in Old coordinate $\{y^\mu\}$}
	\annotate[yshift=-0.5em]{below,label below}{node3}{at New point}
	\annotate[yshift=0.7em]{left}{node4}{Old tensor}
	\annotate[yshift=-1em]{below,label below}{node5}{in New coordinate $\{x^{\prime\mu}\}$}
	\annotate[yshift=1em]{}{node6}{at Old point}
	\vspace{0em}\\
	
	更多关于等价性的论述见梁灿彬\cite{lcb}第四章相关内容,后面会直接作为结论直接进行引述。
\end{remark}

\section{Killing \& Conformal Killing vector field}
下面我们考虑$\mathcal{M}=\mathcal{N},\{x^\mu\}=\{y^\mu\}$。对于矢量场$\xi^a$,其积分曲线诱导了一个微分同胚(点沿着积分曲线流动),在被动观点下看就是诱导了一个无穷小坐标变换$x^\mu\mapsto x^\mu+\xi^\mu t$,其中$t\to0$,在主动观点下看就是诱导了流形上点的变换和张量的变换,但是坐标系仍然不变。首先给出对$\xi$方向的李导数的定义:
\begin{definition}[李导数]
	李导数$\mathscr{L}_\xi$定义为
	\[\mathscr{L}_{\xi}T^{{a_1\cdots a_k}}_{b_1\cdots b_l}\equiv\lim_{t\to0}\frac{1}{t}\left(\phi_t^*{T^{a_1\cdots a_k}}_{b_1\cdots b_l}-{T^{a_1\cdots a_k}}_{b_1\cdots b_l}\right)\]
	可以利用下面的式子计算其在某一坐标系下分量:
	\begin{equation}
		\begin{aligned}
			\mathscr{L}_{\xi}{T^{\mu_1\cdots \mu_k}}_{\nu_1\cdots\nu_l}=&\xi^\lambda\nabla_\lambda{T^{\mu_1\cdots \mu_k}}_{\nu_1\cdots\nu_l}-{T^{\lambda\cdots \mu_k}}_{\nu_1\cdots\nu_l}\nabla_\lambda\xi^{\mu_1}-\cdots-{T^{\mu_1\cdots \lambda}}_{\nu_1\cdots\nu_l}\nabla_\lambda\xi^{\mu_k}\\
			&+{T^{\mu_1\cdots \mu_k}}_{\lambda\cdots\nu_l}\nabla_{\nu_1}\xi^{\lambda}+\cdots+{T^{\mu_1\cdots \mu_k}}_{\nu_1\cdots\lambda}\nabla_{\nu_l}\xi^{\lambda}
		\end{aligned}		
	\end{equation}
\end{definition}
比如度规的李导数:\sn{这里我们用到了度规平行移动的性质,$\forall X\in \mathscr{X}(\mathcal{M}),\nabla_X g=0$}
\begin{equation}
	\mathscr{L}_\mathbf{\xi} g_{\mu\nu}=\nabla_\mu\xi_\nu+\nabla_\mu\xi_\nu	
\end{equation}
\begin{definition}[Killing]
	矢量场$\xi^a$诱导单参数微分同胚群$\phi_t:\mathcal{M}\to\mathcal{M}$,如果其诱导的度规变换满足:
	\begin{equation}
		\phi^*g_{ab}=\Omega^2g_{ab},\quad \forall p\in\mathcal{M}
	\end{equation}
	其中$\Omega^2\in C^\infty(\mathcal{M})$且正定。我们就称向量场为\textbf{共形Killing向量场},对应的微分同胚称为\textbf{共形映射(变换)}。从李导数的观点来看就是要求:
	\begin{equation}
		\mathscr{L}_\xi g_{\mu\nu}=\omega(t)g_{\mu\nu}
	\end{equation}
	其中$\omega(t)\in C^\infty(\mathcal{M})$,有关系$\Omega^2=1+\omega(t)t+\mathcal{O}(t^2)$。特殊的,如果$\Omega^2=1$也即$\omega(t)=0$,那我们就称$\xi^a$为\textbf{Killing向量场},对应的微分同胚为\textbf{等度规映射}。
\end{definition}
前面的Lorentz变换其实就是在找在Minkowski时空内由Killing场诱导的变换,这要求:
\[\mathscr{L}_\xi \eta_{\mu\nu}\nabla_\mu\xi_\nu+\nabla_\mu\xi_\nu=\partial_\mu\xi_\nu+\partial_\nu\xi_\mu=0\]
称为\textbf{Killing方程}。这与前面直接从坐标变换导出的式子是一致的\sn{前面的式子实际上是把无穷小因子$t$吸收进了$\xi$中}。实际上,前面用坐标变换那一套就是在玩被动观点,可以证明,$\xi^a$是(共形)Killing场的充要条件是其生成的坐标变换使得:\sn{注意张量分量作为坐标的函数在何处取值,以及偏导数在何处取值,只要想清楚这些函数的自变量是什么、方程两边各自在哪个坐标系,以及方程作为张量等式都是在流形上同一点取值即可。}
\begin{equation}
	g^\prime_{\mu\nu}(x^\prime)=\frac{\partial x^\sigma}{\partial x^{\prime\mu}}(x^\prime)\frac{\partial x^\sigma}{\partial x^{\prime\nu}}(x^\prime)g_{\sigma\rho}(x)=\Omega^2(x)g_{\mu\nu}(x)
\end{equation}
在场论中我们更多使用坐标变换的被动观点来看问题。这个等式的证明关键就是使用\ref{eq:equiv}。

根据前面的论证,对于四维闵氏时空,Killing方程共有$10=\frac{4\times(4+1)}{2}$个独立解,也就是说存在$10$个独立的Killing场。实际上可以证明,对于任意$n$维时空,其最多具有$\frac{n(n+1)}{2}$个独立的Killing场,然而闵氏时空的$Poincar\'e$变换正好取到不等式上界。由于Killing场诱导的是等度规映射,所以也成为时空的对称性,根据上面的分析,平直闵氏时空具有最大的对称性。
\begin{remark}
	在GR的语境下提到共形变换更多的其实是指Wald invariant,它的定义和共形变换很像,但是不等价。Wald invariant的定义不需要微分同胚,或者说我们直接取同胚为$\mathrm{id}_\mathcal{M}$,这样流形上的点、张量和坐标系都不变,但是我们直接把流形上的度规结构改变,变成:
	\[\tilde{g}_{ab}=\Omega^2g_{ab}\]
	还要求物理不变,这就是Wald invariant,定义与共形变换非常类似,但是不能混淆两者,两者之间的微妙区别会体现在弦论中共形反常的消去上。\cite{Blumenhagen}
\end{remark}

度规在流形上定义了长度$\|v\|\equiv g(v,v)$,相应的可以定义角度为:
\[\cos\theta\equiv\frac{g(v,w)}{\|v\|\cdot\|w\|}\]
显然,共形变换是不改变两曲线交点处切矢之间角度的变换,也常被称为\textbf{保角变换}。
\section{Conformal transformations in $d>2$}
	现在假设背景时空是平直时空\footnote{后面的推导对于Minkowski时空和Euclide时空都适用},利用共形Killing方程,共形Killing场满足:
	\begin{equation}\label{11.6}
		\partial_\mu\xi_\nu+\partial_\nu\xi_\mu = \omega(x)g_{\mu\nu}
	\end{equation}
	式子两边同时取迹得到:
	\begin{equation}\label{11.7}
		\omega(x)=\frac{2}{d}\partial^\mu\xi_\mu
	\end{equation}
	\ref{11.6}两边同时微分$\partial_\rho$得到:
	\begin{equation}
		2\partial_\rho\partial_{{\{\mu}}\xi_{{\nu\}}}=\partial_\rho\omega g_{\mu\nu}
	\end{equation}
	上式对三个指标进行轮换,每次轮换改变符号然后相加得到:
	\begin{equation}\label{11.9}
		-\partial_\rho \omega g_{\mu\nu}+\partial_\mu \omega g_{\nu\mu}=2\partial_\mu\partial\nu \xi_\rho
	\end{equation}
	再次与$g^{\mu\nu}$缩并得到:
	\begin{equation}\label{11.10}
		\partial^\u\partial_\mu\xi_\rho=\frac{2-d}{2}\partial_\rho\omega
	\end{equation}
	\ref{11.6}作用上$\partial^\rho\partial_\rho$,再由\ref{11.10}得到:
	\begin{equation}\label{11.11}
		\left[\partial^\rho\partial_\rho g_{\mu\nu}+(d-2)\partial_\mu\partial_\nu\right]\omega(x)
	\end{equation}
	上式两边求迹得到:
	\begin{equation}
		(d-1)\partial^\mu\partial_\mu \omega(x)=0
	\end{equation}
	$d=1$,上式恒成立,也就是说任意变换都是共形变换,这是由于一维无法定义角度导致的,$d\geq 2$时,满足拉普拉斯方程:
	\begin{equation}
		\square^2 \omega(x)=0
	\end{equation}
	$d>2$则根据\ref{11.11}还进一步要求:
	\begin{equation}
		\partial_\mu\partial_\nu\omega(x)=0
	\end{equation}
	这说明$\omega$形式上只能为:
	\begin{equation}
		\omega(x)=A+B_\mu x^\mu
	\end{equation}
	代入 ref{11.10}得到
	\begin{equation}
		\partial_{\mu} \partial_{\nu} \xi_{\rho}=\frac{1}{2}\left(-B_{\rho} g_{\mu \nu}+B_{\mu} g_{\nu \rho}+B_{\nu} g_{\rho \mu}\right)
	\end{equation}
	右边是常向量。因此, $\xi_\mu $是 $x^\mu $的二次函数,可展开成
	\begin{equation}
		\xi_{\mu}(x)=a_{\mu}+b_{\mu \nu} x^{\nu}+c_{\mu \nu \rho} x^{\nu} x^{\rho}
	\end{equation}
	
	这里, $a_\mu,b_{\mu\nu},c_{\mu\nu\rho}$ 是常数, $c_{\mu\nu\rho} $关于后两指标对称: $c_{\mu \nu \rho}=c_{\mu \rho \nu} $。将上式代入 \ref{11.7} ,得到
	\begin{equation}
		\omega(x)=\frac{2}{d}\left(b^{\mu}{}_{\mu}+2 c^{\mu}{}_{\mu \rho} x^{\rho}\right)
	\end{equation}
	因此, $\omega $的展开式 (1.30) 中的系数 $A,B $同 $b,c $的关系是
	\begin{equation}
		A=\frac{2}{d} b^{\mu}{}_{\mu}, \quad B_{\mu}=\frac{4}{d} c^{\nu}{}_{\nu \mu}
	\end{equation}
	那么 $A,B$ 由 $a,b,c $确定了, (1.32) 代入 (1.31) 和 (1.18) ,可进一步限制 $b,c$ 的形式。事实上,代入后得到
	\begin{align} &2 c_{\rho \mu \nu}=\frac{1}{2}\left(-B_{\rho} g_{\mu \nu}+B_{\mu} g_{\nu \rho}+B_{\nu} g_{\rho \mu}\right)\\ &b_{\mu \nu}+b_{\nu \mu}+2\left(c_{\mu \nu \rho}+c_{\nu \mu \rho}\right) x^{\rho}=\left(A+B_{\rho} x^{\rho}\right) g_{\mu \nu} \end{align}
	于是
	\begin{align} &b_{\mu \nu}+b_{\nu \mu}=A g_{\mu \nu}\\ &c_{\mu \nu \rho}=\frac{1}{4}\left(-B_{\mu} g_{\nu \rho}+B_{\nu} g_{\rho \mu}+B_{\rho} g_{\mu \nu}\right) \end{align}
	
	由此,知道共形因子后就可以写出共形变换。最终得到无穷小变换可分成以下几类:\sn{SCT: Special Conformal Transformation}
	\begin{equation}
		\boxed{
		\begin{aligned}
				&\text{Translation}&&x^{\prime \mu}=x^{\mu}-a^{\mu}\\
				&\text{Rotation}&&x^{\prime \mu}=x^{\mu}-b^{A \mu \nu} x_{\nu}, \quad b^{A \mu \nu}=-b^{A \nu \mu}\\
				&\text{Dilation}&&x^{\prime \mu}=x^{\mu}-\frac{A}{2} x^{\mu}\\
				&\text{SCT}&&x^{\prime \mu}=x^{\mu}-\frac{1}{4}\left(-B^{\mu} x^{2}+2 x^{\mu} B^{\nu} x_{\nu}\right)
		\end{aligned}
	}
	\end{equation}
	
	对应的无穷小变换的生成元可表示为
	\begin{equation}
		\boxed{
		\begin{aligned} &P_{\mu}=-i \partial_{\mu} \\ &M_{\mu \nu}=i\left(x_{\mu} \partial_{\nu}-x_{\nu} \partial_{\mu}\right) \\\ &D=-i x^{\mu} \partial_{\mu} \\ &K_{\mu}=-i\left(2 x_{\mu} x^{\nu} \partial_{\nu}-x^{2} \partial_{\mu}\right)  
		\end{aligned}
	}
	\end{equation}
	它们之间的对易关系为:
	\begin{equation}
		\boxed{
				\begin{aligned} 
					&[P_\mu,P_\nu]=0,\quad[P_\rho,M_{\mu\nu}]=i(\eta_{\mu\rho}P_\nu-\eta_{\nu\rho} P_\mu)\\
					&[M_{\mu\nu},M_{\rho\sigma}=i\left(\eta_{\mu\rho}M_{\nu\sigma}-\eta_{\mu\sigma}M_{\nu\rho}-\eta_{\nu\rho}M_{\nu\rho}+\eta_{\nu\sigma}M_{\mu\rho}\right)]\\
					&\left[D, P_{\mu}\right]=i P_{\mu}\\ &\left[D, K_{\mu}\right]=-i K_{\mu} \\ &\left[K_{\mu}, P_{\nu}\right]=2 i\left(g_{\mu \nu} D-M_{\mu \nu}\right) \\ &\left[K_{\rho}, M_{\mu \nu}\right]=i\left(g_{\rho \mu} K_{\nu}-g_{\rho \nu} K_{\mu}\right) \\ &\left[P_{\rho}, M_{\mu \nu}\right]=i\left(g_{\rho \mu} P_{\nu}-g_{\rho \nu} P_{\mu}\right) 
			\end{aligned}
		
		}
	\end{equation}
	
	可以看到其中一部分就是Poincar\'e代数\sn{$g_{\mu\nu}=\delta_{\mu\nu}$时为Euclide代数},这也说明了等度规变换是共形变换的特殊情况。这个Lie代数称为\textbf{$ d $维共形代数}。
	
	重定义生成元 $J_{ab} (a,b=-1,0,\cdots,d )$:
	\begin{align} &J_{\mu \nu}=M_{\mu \nu}\\ &J_{-1 \mu}=\frac{1}{2}\left(P_{\mu}-K_{\mu}\right) \\ &J_{-10}=D \\ &J_{0 \mu}=\frac{1}{2}\left(P_{\mu}+K_{\mu}\right)  \end{align}
	从共形代数的对易关系,可以得到 $J_{ab}$ 满足对易关系
	\begin{equation}
		\left[J_{a b}, J_{c d}\right]=i\left(g_{a d} J_{b c}+g_{b c} J_{a d}-g_{a c} J_{b d}-g_{b d} J_{a c}\right)
	\end{equation}
	当共形变换是Euclide空间中的变换时, $g_{ab} $是号差为 $(-,+, \cdots,+)$ 的Minkowski度规\sn{否则有两个负号}。也就是说$d$维(Euclide空间)中的共形代数,同构于Lorentz代数$\mathfrak{so}(d+1,1) $。
	
	有限共形变换可有由无穷小共形变换的叠加构成,形式为:
	\begin{equation}
		\boxed{
			\begin{aligned}
				&\text{Translation}&&x^{\prime \mu}=x^{\mu}-a^{\mu}\\
				&\text{Rotation(Boost)}&&x^{\prime \mu}=\Lambda_{\nu}^{\mu} x^{\nu}\\
				&\text{Dilation}&&x^{\prime \mu}=\alpha x^{\mu}\\
				&\text{SCT}&&x^{\prime \mu}=\frac{x^{\mu}-b^{\mu} x^{2}}{1-2 b\cdot x+b^{2} x^{2}}
			\end{aligned}
		}
	\end{equation}
	其中最后一个变换是由平移和反演变换的组合得到的:
		\begin{equation}
		x^{\mu} \rightarrow x^{\prime \mu}=\frac{x^{\mu}}{x^{2}} \rightarrow x^{\prime \prime \mu}=x^{\prime \mu}-b^{\mu} \rightarrow x^{\prime \prime \prime \mu}=\frac{x^{\prime \prime \mu}}{x^{\prime \prime 2}}
	\end{equation}
	而反演变换是离散的,所以SCT并不能用无穷小变换生成。根据前面的讨论,这些变化构成$SO(d+1,1)$群。
\section{Conformal transformations in $d=2$}
	上一节利用李导数看待问题,也就是所谓主动观点,现在用坐标变换的观点来进行推演。
	
	在二维情况下,使用复数作为参数非常方便,定义复变量:\sn{导数这样定义是为了$\partial_z z=\partial_{\bar z}\bar z=1$}
	\[z=x^1+ix^2,\quad \partial_z\equiv\frac{1}{2}(\partial_1-i\partial_2),\quad \partial_{\bar z}\equiv\frac{1}{2}(\partial_1+i\partial_2)\]
	Euclide空间中度规可以写成:
	\begin{equation}
		g_z = dzd\bar z
	\end{equation}
	现在考虑坐标变换$z\mapsto z^\prime(z,\bar z)$,度规相应变为:
	\begin{equation}
		\begin{aligned}
			g_z\mapsto g^'_z&=dz^\prime d{\bar z}^\prime\\
			&=\left(\frac{\partial z^\prime}{\partial z}dz+\frac{\partial z^\prime}{\partial\bar z}d\bar z\right)
			\left(\frac{\partial \bar z^\prime}{\partial z}dz+\frac{\partial \bar z^\prime}{\partial\bar z}d\bar z\right)\\
			&=\frac{\partial z^\prime}{\partial z}\frac{\partial \bar z^\prime}{\partial z}dz^2+\frac{\partial z^\prime}{\partial \bar z}\frac{\partial \bar z^\prime}{\partial \bar z}{dz^\prime}^2+\left(\frac{\partial \bar z^\prime}{\partial  z}\frac{\partial  z^\prime}{\partial \bar z}+\frac{\partial z^\prime}{\partial z}\frac{\partial \bar z^\prime}{\partial \bar z}\right)dzd\bar z
		\end{aligned}
	\end{equation}
	这导致了下面的Cauchy-Riemann条件:
	\begin{equation}
		\frac{\partial z^\prime}{\partial z}=0\quad \text{or} \frac{\partial z^\prime}{\partial \bar z}=0
	\end{equation}
	而且前面的推导只要求度规形式为$\Omega(z,\bar z)dzd\bar z$。也就是说,变换$z^\prime$为全纯或者反全纯函数,而反全纯函数会将右手系变为左手系,后面主要考虑全纯情况,反全纯情况只需要全部取复共轭即可。
	
	考虑局部的无穷小共形变换,变换形式可以写成:
	\[z\mapsto z+\xi(z),\quad \bar z\mapsto \bar z+\bar\xi(\bar z)\]
	其中$\xi$可以进行Laurent展开为:
	\[\xi(z)=\sum_{n=-\infty}^{+\infty}\xi_n z^{n+1} ,\quad \bar\xi(\bar z)=\sum_{n=-\infty}^{+\infty}\bar\xi_n \bar z^{n+1}\]
	对应于
	
	
	
	
	
	$d$维共性代数与$\mathfrak{so}(d+1,1)$之间的同构可以显式的构造出来。首先注意到由于Lorentz变换保$ds^2$,所以对于顶点在原点处的光锥,Lorentz变换是在光锥上的同胚。那我们可以把光锥看作是嵌入在$\mathbb{R}^{d+1,1}$中的子流形,这实际上是将$\mathbb{R}^{d}$通过下面的方式嵌入到$\mathbb{R}^{d+1,1}$中:
	\begin{equation}
		q^{\mu}=(1+x^2,2x^A,1-x^2),\quad x^A\in\mathbb{R}^d
	\end{equation}
	这样,Lorentz变换作用在光锥上可以看作是$\mathbb{R}^d$上的一个变换:
	\begin{equation}
		q^\mu\mapsto q^{\prime\mu}=\frac{(\Lambda q)^\mu}{(\Lambda q)^+}
	\end{equation}
	其中分母是为了将$q^+\equiv\frac{1}{2}\left(q^0+q^{d+1}\right)=1$归一化。而且,可以验证这个变换还是一个共形变换!特别地,当$d=2$时,定义$w=x^1+ix^2$,这个同构变成:
	\begin{equation}
		q^\mu=()
	\end{equation}
\section{Conformal transformations on the Riemann sphere}


	\part{Celestial Sphere}
to be continue$\ldots$
	
%	\begin{thebibliography}{99}
%	
%	\end{thebibliography}
	\nocite{*}
	\addcontentsline{toc}{part}{参考文献}
	\bibliographystyle{JHEP}
	\bibliography{ref.bib}
\end{document}